\begin{comment}
\documentclass[10pt]{article}
\usepackage{fullpage, graphicx, url}
\setlength{\parskip}{1ex}
\setlength{\parindent}{0ex}
\title{r Statement (Repeat Statement)}
\begin{document}


\begin{tabular}{ccc}
The Alternative Csound Reference Manual & & \\
Previous & &Next

\end{tabular}

%\hline 
\end{comment}
\section{r Statement (Repeat Statement)}
r�--� Starts a repeated section. \subsection*{Description}


  Starts a repeated section, which lasts until the next \emph{s}
, \emph{r}
 or \emph{e statement}
. 
\subsection*{Syntax}


 \textbf{r}
 p1 p2
\subsection*{Initialization}


 \emph{p1}
 -- Number of times to repeat the section. 


 \emph{p2}
 -- Macro(name) to advance with each repetition (optional). 
\subsection*{Performance}


  In order that the sections may be more flexible than simple editing, the macro named p2 is given the value of 1 for the first time through the section, 2 for the second, and 3 for the third. This can be used to change p-field parameters, or ignored. 


 


\begin{tabular}{cc}
Warning &\textbf{Warning}
 \\
� &

  Because of serious problems of interaction with macro expansion, sections must start and end in the same file, and not in a macro. 


\end{tabular}

\subsection*{Examples}


  Here is an example of the r statement. It uses the files \emph{r.orc}
 and \emph{r.sco}
. 


 \textbf{Example 1. Example of the r statement.}

\begin{lstlisting}
/* r.orc */
; Initialize the global variables.
sr = 44100
kr = 4410
ksmps = 10
nchnls = 1

; Instrument #1.
instr 1
  ; The score's p4 parameter has the number of repeats.
  kreps = p4
  ; The score's p5 parameter has our note's frequency.
  kcps = p5
  
  ; Print the number of repeats.
  printks "Repeated %i time(s).\\n", 1, kreps

  ; Generate a nice beep.
  a1 oscil 20000, kcps, 1
  out a1
endin
/* r.orc */
        
\end{lstlisting}
\begin{lstlisting}
/* r.sco */
; Table #1, a sine wave.
f 1 0 16384 10 1

; We'll repeat this section 6 times. Each time it 
; is repeated, its macro REPS_MACRO is incremented. 
r6 REPS_MACRO

; Play Instrument #1.
; p4 = the r statement's macro, REPS_MACRO.
; p5 = the frequency in cycles per second.
i 1 00.10 00.10 $REPS_MACRO 1760
i 1 00.30 00.10 $REPS_MACRO 880
i 1 00.50 00.10 $REPS_MACRO 440
i 1 00.70 00.10 $REPS_MACRO 220

; Marks the end of the section.
s

e
/* r.sco */
        
\end{lstlisting}
\subsection*{Credits}


 


 


\begin{tabular}{cccc}
Author: John ffitch &University of Bath/Codemist Ltd. &Bath, UK &April, 1998

\end{tabular}



 


 New in Csound version 3.48


 Example written by Kevin Conder
%\hline 


\begin{comment}
\begin{tabular}{lcr}
Previous &Home &Next \\
q Statement &Up &s Statement

\end{tabular}


\end{document}
\end{comment}
