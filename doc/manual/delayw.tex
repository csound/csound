\begin{comment}
\documentclass[10pt]{article}
\usepackage{fullpage, graphicx, url}
\setlength{\parskip}{1ex}
\setlength{\parindent}{0ex}
\title{delayw}
\begin{document}


\begin{tabular}{ccc}
The Alternative Csound Reference Manual & & \\
Previous & &Next

\end{tabular}

%\hline 
\end{comment}
\section{delayw}
delayw�--� Writes the audio signal to a digital delay line. \subsection*{Description}


  Writes the audio signal to a digital delay line. 
\subsection*{Syntax}


 \textbf{delayw}
 asig
\subsection*{Performance}


 \emph{delayw}
 writes \emph{asig}
 into the delay area established by the preceding \emph{delayr}
 unit. Viewed as a pair, these two units permit the formation of modified feedback loops, etc. However, there is a lower bound on the value of \emph{idlt}
, which must be at least 1 control period (or 1/\emph{kr}
). 
\subsection*{Examples}


  Here is an example of the delayw opcode. It uses the files \emph{delayw.orc}
 and \emph{delayw.sco}
. 


 \textbf{Example 1. Example of the delayw opcode.}

\begin{lstlisting}
/* delayw.orc */
; Initialize the global variables.
sr = 44100
kr = 4410
ksmps = 10
nchnls = 2

; Instrument #1 -- Delayed beeps.
instr 1
  ; Make a basic sound.
  abeep vco 20000, 440, 1

  ; Set up a delay line.
  idlt = 0.1
  adel delayr idlt

  ; Write the beep to the delay line.
  delayw abeep

  ; Send the beep to the left speaker and
  ; the delayed beep to the right speaker.
  outs abeep, adel
endin
/* delayw.orc */
        
\end{lstlisting}
\begin{lstlisting}
/* delayw.sco */
; Table #1, a sine wave.
f 1 0 16384 10 1

; Keep the score running for 2 seconds.
f 0 2

; Play Instrument #1.
i 1 0.0 0.2
i 1 0.5 0.2
e
/* delayw.sco */
        
\end{lstlisting}
\subsection*{See Also}


 \emph{delay}
, \emph{delay1}
, \emph{delayr}

\subsection*{Credits}


 Example written by Kevin Conder.
%\hline 


\begin{comment}
\begin{tabular}{lcr}
Previous &Home &Next \\
delayr &Up &deltap

\end{tabular}


\end{document}
\end{comment}
