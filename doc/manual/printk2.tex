\begin{comment}
\documentclass[10pt]{article}
\usepackage{fullpage, graphicx, url}
\setlength{\parskip}{1ex}
\setlength{\parindent}{0ex}
\title{printk2}
\begin{document}


\begin{tabular}{ccc}
The Alternative Csound Reference Manual & & \\
Previous & &Next

\end{tabular}

%\hline 
\end{comment}
\section{printk2}
printk2�--� Prints a new value every time a control variable changes. \subsection*{Description}


  Prints a new value every time a control variable changes. 
\subsection*{Syntax}


 \textbf{printk2}
 kvar [, inumspaces]
\subsection*{Initialization}


 \emph{inumspaces}
 (optional, default=0) -- number of space characters printed before the value of \emph{kvar}

\subsection*{Performance}


 \emph{kvar}
 -- signal to be printed 


  Derived from Robin Whittle's \emph{printk}
, prints a new value of \emph{kvar}
 each time \emph{kvar}
 changes. Useful for monitoring MIDI control changes when using sliders. 


 


\begin{tabular}{cc}
Warning &

 \emph{WARNING!}
 Don't use this opcode with normal, continuously variant k-signals, because it can hang the computer, as the rate of printing is too fast. 


\end{tabular}

\subsection*{Examples}


  Here is an example of the printk2 opcode. It uses the files \emph{printk2.orc}
 and \emph{printk2.sco}
. 


 \textbf{Example 1. Example of the printk2 opcode.}

\begin{lstlisting}
/* printk2.orc */
; Initialize the global variables.
sr = 44100
kr = 44100
ksmps = 1
nchnls = 1

; Instrument #1.
instr 1
  ; Change a value linearly from 0 to 10,
  ; over the period defined by p3.
  kval1 line 0, p3, 10

  ; If kval1 is greater than or equal to 5, 
  ; then kval=2, else kval=1.
  kval2 = (kval1 >= 5 ? 2 : 1)

  ; Print the value of kval2 when it changes.
  printk2 kval2
endin
/* printk2.orc */
        
\end{lstlisting}
\begin{lstlisting}
/* printk2.sco */
; Play Instrument #1 for 5 seconds.
i 1 0 5
e
/* printk2.sco */
        
\end{lstlisting}
 Its output should include a line like this: \begin{lstlisting}
 i1     1.00000
 i1     2.00000
      
\end{lstlisting}
\subsection*{See Also}


 \emph{printk}
 and \emph{printks}

\subsection*{Credits}


 


 


\begin{tabular}{ccc}
Author: Gabriel Maldonado &Italy &1998

\end{tabular}



 


 Example written by Kevin Conder.


 New in Csound version 3.48
%\hline 


\begin{comment}
\begin{tabular}{lcr}
Previous &Home &Next \\
printk &Up &printks

\end{tabular}


\end{document}
\end{comment}
