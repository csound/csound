\begin{comment}
\documentclass[10pt]{article}
\usepackage{fullpage, graphicx, url}
\setlength{\parskip}{1ex}
\setlength{\parindent}{0ex}
\title{comb}
\begin{document}


\begin{tabular}{ccc}
The Alternative Csound Reference Manual & & \\
Previous & &Next

\end{tabular}

%\hline 
\end{comment}
\section{comb}
comb�--� Reverberates an input signal with a ``colored'' frequency response. \subsection*{Description}


  Reverberates an input signal with a ``colored'' frequency response. 
\subsection*{Syntax}


 ar \textbf{comb}
 asig, krvt, ilpt [, iskip] [, insmps]
\subsection*{Initialization}


 \emph{ilpt}
 -- loop time in seconds, which determines the ``echo density'' of the reverberation. This in turn characterizes the ``color'' of the \emph{comb}
 filter whose frequency response curve will contain \emph{ilpt}
 * \emph{sr}
/2 peaks spaced evenly between 0 and \emph{sr}
/2 (the Nyquist frequency). Loop time can be as large as available memory will permit. The space required for an \emph{n}
 second loop is 4\emph{n}
*\emph{sr}
 bytes. Delay space is allocated and returned as in \emph{delay}
. 


 \emph{iskip}
 (optional, default=0) -- initial disposition of delay-loop data space (cf. \emph{reson}
). The default value is 0. 


 \emph{insmps}
 (optional, default=0) -- delay amount, as a number of samples. 
\subsection*{Performance}


 \emph{krvt}
 -- the reverberation time (defined as the time in seconds for a signal to decay to 1/1000, or 60dB down from its original amplitude). 


  This filter reiterates input with an echo density determined by loop time \emph{ilpt}
. The attenuation rate is independent and is determined by \emph{krvt}
, the reverberation time (defined as the time in seconds for a signal to decay to 1/1000, or 60dB down from its original amplitude). Output from a comb filter will appear only after \emph{ilpt}
 seconds. 
\subsection*{Examples}


  Here is an example of the comb opcode. It uses the files \emph{comb.orc}
 and \emph{comb.sco}
. 


 \textbf{Example 1. Example of the comb opcode.}

\begin{lstlisting}
/* comb.orc */
; Initialize the global variables.
sr = 44100
kr = 4410
ksmps = 10
nchnls = 1

; Initialize the audio mixer.
gamix init 0 

; Instrument #1.
instr 1 
  ; Generate a source signal.
  a1 oscili 30000, cpspch(p4), 1
  ; Output the direct sound.
  out a1  

  ; Add the source signal to the audio mixer.
  gamix = gamix + a1 
endin

; Instrument #99 (highest instr number executed last)
instr 99 
  krvt = 1.5
  ilpt = 0.1

  ; Comb-filter the mixed signal.
  a99 comb gamix, krvt, ilpt
  ; Output the result.
  out a99 

  ; Empty the mixer for the next pass.
  gamix = 0 
endin
/* comb.orc */
        
\end{lstlisting}
\begin{lstlisting}
/* comb.sco */
; Table #1, a sine wave.
f 1 0 128 10 1

; p4 = frequency (in a pitch-class)
; Play Instrument #1 for a tenth of a second, p4=7.00
i 1 0 0.1 7.00
; Play Instrument #1 for a tenth of a second, p4=7.02
i 1 1 0.1 7.02
; Play Instrument #1 for a tenth of a second, p4=7.04
i 1 2 0.1 7.04
; Play Instrument #1 for a tenth of a second, p4=7.06
i 1 3 0.1 7.06

; Make sure the comb-filter remains active.
i 99 0 5
e
/* comb.sco */
        
\end{lstlisting}
\subsection*{See Also}


 \emph{alpass}
, \emph{reverb}
, \emph{valpass}
, \emph{vcomb}

\subsection*{Credits}


 


 


\begin{tabular}{cccc}
Author: William ``Pete'' Moss (\emph{vcomb}
 and \emph{valpass}
) &University of Texas at Austin &Austin, Texas USA &January 2002

\end{tabular}



 


 Example written by Kevin Conder.
%\hline 


\begin{comment}
\begin{tabular}{lcr}
Previous &Home &Next \\
cngoto &Up &control

\end{tabular}


\end{document}
\end{comment}
