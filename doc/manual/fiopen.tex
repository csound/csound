\begin{comment}
\documentclass[10pt]{article}
\usepackage{fullpage, graphicx, url}
\setlength{\parskip}{1ex}
\setlength{\parindent}{0ex}
\title{fiopen}
\begin{document}


\begin{tabular}{ccc}
The Alternative Csound Reference Manual & & \\
Previous & &Next

\end{tabular}

%\hline 
\end{comment}
\section{fiopen}
fiopen�--� Opens a file in a specific mode. \subsection*{Description}


 \emph{fiopen}
 can be used to open a file in one of the specified modes. 
\subsection*{Syntax}


 ihandle \textbf{fiopen}
 ifilename, imode
\subsection*{Initialization}


 \emph{ihandle}
 -- a number which specifies this file. 


 \emph{ifilename}
 -- the output file's name (in double-quotes). 


 \emph{imode}
 -- choose the mode of opening the file. \emph{imode}
 can be a value chosen among the following: 


 
\begin{itemize}
\item 

 0 - open a text file for writing

\item 

 1 - open a text file for reading

\item 

 2 - open a binary file for writing

\item 

 3 - open a binary file for reading


\end{itemize}
\subsection*{Performance}


 \emph{fiopen}
 opens a file to be used by the \emph{fout}
 family of opcodes. It must be defined in the header section, external to any instruments. It returns a number, \emph{ihandle}
, which unequivocally refers to the opened file. 


  Notice that \emph{fout}
 and \emph{foutk}
 can use either a string containing a file pathname, or a handle-number generated by \emph{fiopen}
. Whereas, with \emph{fouti}
 and \emph{foutir}
, the target file can be only specified by means of a handle-number. 
\subsection*{See Also}


 \emph{fout}
, \emph{fouti}
, \emph{foutir}
, \emph{foutk}

\subsection*{Credits}


 


 


\begin{tabular}{ccc}
Author: Gabriel Maldonado &Italy &1999

\end{tabular}



 


 New in Csound version 3.56
%\hline 


\begin{comment}
\begin{tabular}{lcr}
Previous &Home &Next \\
fink &Up &flanger

\end{tabular}


\end{document}
\end{comment}
