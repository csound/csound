\begin{comment}
\documentclass[10pt]{article}
\usepackage{fullpage, graphicx, url}
\setlength{\parskip}{1ex}
\setlength{\parindent}{0ex}
\title{osciln}
\begin{document}


\begin{tabular}{ccc}
The Alternative Csound Reference Manual & & \\
Previous & &Next

\end{tabular}

%\hline 
\end{comment}
\section{osciln}
osciln�--� Accesses table values at a user-defined frequency. \subsection*{Description}


  Accesses table values at a user-defined frequency. This opcode can also be written as \emph{oscilx}
. 
\subsection*{Syntax}


 ar \textbf{osciln}
 kamp, ifrq, ifn, itimes
\subsection*{Initialization}


 \emph{ifrq, itimes}
 -- rate and number of times through the stored table. 


 \emph{ifn}
 -- function table number. 
\subsection*{Performance}


 \emph{kamp}
 -- amplitude factor 


 \emph{osciln}
 will sample several times through the stored table at a rate of \emph{ifrq}
 times per second, after which it will output zeros. Generates audio signals only, with output values scaled by \emph{kamp.}

\subsection*{See Also}


 \emph{table}
, \emph{tablei}
, \emph{table3}
, \emph{oscil1}
, \emph{oscil1i}

%\hline 


\begin{comment}
\begin{tabular}{lcr}
Previous &Home &Next \\
oscilikts &Up &oscils

\end{tabular}


\end{document}
\end{comment}
