\begin{comment}
\documentclass[10pt]{article}
\usepackage{fullpage, graphicx, url}
\setlength{\parskip}{1ex}
\setlength{\parindent}{0ex}
\title{duserrnd}
\begin{document}


\begin{tabular}{ccc}
The Alternative Csound Reference Manual & & \\
Previous & &Next

\end{tabular}

%\hline 
\end{comment}
\section{duserrnd}
duserrnd�--� Discrete USER-defined-distribution RaNDom generator. \subsection*{Description}


  Discrete USER-defined-distribution RaNDom generator. 
\subsection*{Syntax}


 aout \textbf{duserrnd}
 ktableNum


 iout \textbf{duserrnd}
 itableNum


 kout \textbf{duserrnd}
 ktableNum
\subsection*{Initialization}


 \emph{itableNum}
 -- number of table containing the random-distribution function. Such table is generated by the user. See GEN40, GEN41, and GEN42. The table length does not need to be a power of 2 
\subsection*{Performance}


 \emph{ktableNum}
 -- number of table containing the random-distribution function. Such table is generated by the user. See GEN40, GEN41, and GEN42. The table length does not need to be a power of 2 


 \emph{duserrnd}
 (discrete user-defined-distribution random generator) generates random values according to a discrete random distribution created by the user. The user can create the discrete distribution histogram by using GEN41. In order to create that table, the user has to define an arbitrary amount of number pairs, the first number of each pair representing a value and the second representing its probability (see GEN41 for more details). 


  When used as a function, the rate of generation depends by the rate type of input variable XtableNum. In this case it can be embedded into any formula. Table number can be varied at k-rate, allowing to change the distribution histogram during the performance of a single note. \emph{duserrnd}
 is designed be used in algorithmic music generation. 


 \emph{duserrnd}
 can also be used to generate values following a set of ranges of probabilities by using distribution functions generated by GEN42 (See GEN42 for more details). In this case, in order to simulate continuous ranges, the length of table XtableNum should be reasonably big, as \emph{duserrnd}
 does not interpolate between table elements. 


  For a tutorial about random distribution histograms and functions see: 


 
\begin{itemize}
\item 

  D. Lorrain. ``A panoply of stochastic cannons''. In C. Roads, ed. 1989. Music machine. Cambridge, Massachusetts: MIT press, pp. 351 - 379. 


\end{itemize}
\subsection*{See Also}


 \emph{cuserrnd}
, \emph{urd}

\subsection*{Credits}


 Author: Gabriel Maldonado


 New in Version 4.16
%\hline 


\begin{comment}
\begin{tabular}{lcr}
Previous &Home &Next \\
dumpk4 &Up &else

\end{tabular}


\end{document}
\end{comment}
