\begin{comment}
\documentclass[10pt]{article}
\usepackage{fullpage, graphicx, url}
\setlength{\parskip}{1ex}
\setlength{\parindent}{0ex}
\title{mrtmsg}
\begin{document}


\begin{tabular}{ccc}
The Alternative Csound Reference Manual & & \\
Previous & &Next

\end{tabular}

%\hline 
\end{comment}
\section{mrtmsg}
mrtmsg�--� Send system real-time messages to the MIDI OUT port. \subsection*{Description}


  Send system real-time messages to the MIDI OUT port. 
\subsection*{Syntax}


 \textbf{mrtmsg}
 imsgtype
\subsection*{Initialization}


 \emph{imsgtype}
 -- type of real-time message: 


 
\begin{itemize}
\item 

 1 sends a START message (0xFA);

\item 

 2 sends a CONTINUE message (0xFB);

\item 

 0 sends a STOP message (0xFC);

\item 

 -1 sends a SYSTEM RESET message (0xFF);

\item 

 -2 sends an ACTIVE SENSING message (0xFE)


\end{itemize}
\subsection*{Performance}


  Sends a real-time message once, in init stage of current instrument. \emph{imsgtype}
 parameter is a flag to indicate the message type. 
\subsection*{See Also}


 \emph{mclock}

\subsection*{Credits}


 


 


\begin{tabular}{cc}
Author: Gabriel Maldonado &Italy

\end{tabular}



 


 New in Csound version 3.47
%\hline 


\begin{comment}
\begin{tabular}{lcr}
Previous &Home &Next \\
mpulse &Up &multitap

\end{tabular}


\end{document}
\end{comment}
