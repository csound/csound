\begin{comment}
\documentclass[10pt]{article}
\usepackage{fullpage, graphicx, url}
\setlength{\parskip}{1ex}
\setlength{\parindent}{0ex}
\title{clockon}
\begin{document}


\begin{tabular}{ccc}
The Alternative Csound Reference Manual & & \\
Previous & &Next

\end{tabular}

%\hline 
\end{comment}
\section{clockon}
clockon�--� Starts one of a number of internal clocks. \subsection*{Description}


  Starts one of a number of internal clocks. 
\subsection*{Syntax}


 \textbf{clockon}
 inum
\subsection*{Initialization}


 \emph{inum}
 -- the number of a clock. There are 32 clocks numbered 0 through 31. All other values are mapped to clock number 32. 
\subsection*{Performance}


  Between a \emph{clockon}
 and a \emph{clockoff}
 opcode, the CPU time used is accumulated in the clock. The precision is machine dependent but is the millisecond range on UNIX and Windows systems. The \emph{readclock}
 opcode reads the current value of a clock at initialization time. 
\subsection*{Examples}


  See the \emph{readclock}
 opcode for an example. 
\subsection*{See Also}


 \emph{clockoff}
, \emph{readclock}

\subsection*{Credits}


 


 


\begin{tabular}{cccc}
Author: John ffitch &University of Bath/Codemist Ltd. &Bath, UK &July, 1999

\end{tabular}



 


 New in Csound version 3.56
%\hline 


\begin{comment}
\begin{tabular}{lcr}
Previous &Home &Next \\
clockoff &Up &cngoto

\end{tabular}


\end{document}
\end{comment}
