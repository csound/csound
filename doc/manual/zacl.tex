\begin{comment}
\documentclass[10pt]{article}
\usepackage{fullpage, graphicx, url}
\setlength{\parskip}{1ex}
\setlength{\parindent}{0ex}
\title{zacl}
\begin{document}


\begin{tabular}{ccc}
The Alternative Csound Reference Manual & & \\
Previous & &Next

\end{tabular}

%\hline 
\end{comment}
\section{zacl}
zacl�--� Clears one or more variables in the za space. \subsection*{Description}


  Clears one or more variables in the za space. 
\subsection*{Syntax}


 \textbf{zacl}
 kfirst, klast
\subsection*{Performance}


 \emph{kfirst}
 -- first zk or za location in the range to clear. 


 \emph{klast}
 -- last zk or za location in the range to clear. 


 \emph{zacl}
 clears one or more variables in the za space. This is useful for those variables which are used as accumulators for mixing a-rate signals at each cycle, but which must be cleared before the next set of calculations. 
\subsection*{Examples}


  Here is an example of the zacl opcode. It uses the files \emph{zacl.orc}
 and \emph{zacl.sco}
. 


 \textbf{Example 1. Example of the zacl opcode.}

\begin{lstlisting}
/* zacl.orc */
; Initialize the global variables.
sr = 44100
kr = 4410
ksmps = 10
nchnls = 1

; Initialize the ZAK space.
; Create 1 a-rate variable and 1 k-rate variable.
zakinit 1, 1

; Instrument #1 -- a simple waveform.
instr 1
  ; Generate a simple sine waveform.
  asin oscil 20000, 440, 1

  ; Send the sine waveform to za variable #1.
  zaw asin, 1
endin

; Instrument #2 -- generates audio output.
instr 2
  ; Read za variable #1.
  a1 zar 1

  ; Generate the audio output.
  out a1

  ; Clear the za variables, get them ready for 
  ; another pass.
  zacl 0, 1
endin
/* zacl.orc */
        
\end{lstlisting}
\begin{lstlisting}
/* zacl.sco */
; Table #1, a sine wave.
f 1 0 16384 10 1

; Play Instrument #1 for one second.
i 1 0 1
; Play Instrument #2 for one second.
i 2 0 1
e
/* zacl.sco */
        
\end{lstlisting}
\subsection*{See Also}


 \emph{zamod}
, \emph{zar}
, \emph{zaw}
, \emph{zawm}
, \emph{ziw}
, \emph{ziwm}

\subsection*{Credits}


 


 


\begin{tabular}{ccc}
Author: Robin Whittle &Australia &May 1997

\end{tabular}



 


 Example written by Kevin Conder.
%\hline 


\begin{comment}
\begin{tabular}{lcr}
Previous &Home &Next \\
xyin &Up &zakinit

\end{tabular}


\end{document}
\end{comment}
