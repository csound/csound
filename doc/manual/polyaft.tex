\begin{comment}
\documentclass[10pt]{article}
\usepackage{fullpage, graphicx, url}
\setlength{\parskip}{1ex}
\setlength{\parindent}{0ex}
\title{polyaft}
\begin{document}


\begin{tabular}{ccc}
The Alternative Csound Reference Manual & & \\
Previous & &Next

\end{tabular}

%\hline 
\end{comment}
\section{polyaft}
polyaft�--� Returns the polyphonic after-touch pressure of the selected note number. \subsection*{Description}


 \emph{polyaft}
 returns the polyphonic pressure of the selected note number, optionally mapped to an user-specified range. 
\subsection*{Syntax}


 ir \textbf{polyaft}
 inote [, ilow] [, ihigh]


 kr \textbf{polyaft}
 inote [, ilow] [, ihigh]
\subsection*{Initialization}


 \emph{inote}
 -- note number. Normally set to the value returned by \emph{notnum}



 \emph{ilow}
 (optional, default: 0) -- lowest output value 


 \emph{ihigh}
 (optional, default: 127) -- highest output value 
\subsection*{Performance}


 \emph{kr}
 -- polyphonic pressure (aftertouch). 
\subsection*{Examples}


  Here is an example of the polyaft opcode. It uses the files \emph{polyaft.mid}
, \emph{polyaft.orc}
 and \emph{polyaft.sco}
. 


  Don't forget that you must include the \emph{-F flag}
 when using an external MIDI file like ``polyaft.mid''. 


 


 \textbf{Example 1. Example of the polyaft opcode.}

\begin{lstlisting}
/* polyaft.orc - written by Istvan Varga */
sr	=  44100
ksmps	=  10
nchnls	=  1

	massign 1, 1
itmp	ftgen 1, 0, 1024, 10, 1		; sine wave

	instr 1

kcps	cpsmidib 2		; note frequency
inote	notnum			; note number
kaft	polyaft inote, 0, 127	; aftertouch
; interpolate aftertouch to eliminate clicks
ktmp	phasor 40
ktmp	trigger 1 - ktmp, 0.5, 0
kaft	tlineto kaft, 0.025, ktmp
; map to sine curve for crossfade
kaft	=  sin(kaft * 3.14159 / 254) * 22000

asnd	oscili kaft, kcps, 1

	out asnd

	endin
/* polyaft.orc - written by Istvan Varga */
        
\end{lstlisting}
\begin{lstlisting}
/* polyaft.sco - written by Istvan Varga */
t 0 120
f 0 9 2 -2 0
e
/* polyaft.sco - written by Istvan Varga */
        
\end{lstlisting}
\subsection*{Credits}


 Added thanks to an email from Istvan Varga


 New in version 4.12
%\hline 


\begin{comment}
\begin{tabular}{lcr}
Previous &Home &Next \\
poisson &Up &port

\end{tabular}


\end{document}
\end{comment}
