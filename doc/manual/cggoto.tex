\begin{comment}
\documentclass[10pt]{article}
\usepackage{fullpage, graphicx, url}
\setlength{\parskip}{1ex}
\setlength{\parindent}{0ex}
\title{cggoto}
\begin{document}


\begin{tabular}{ccc}
The Alternative Csound Reference Manual & & \\
Previous & &Next

\end{tabular}

%\hline 
\end{comment}
\section{cggoto}
cggoto�--� Conditionally transfer control on every pass. \subsection*{Description}


  Transfer control to \emph{label}
 on every pass. (Combination of \emph{cigoto}
 and \emph{ckgoto}
) 
\subsection*{Syntax}


 \textbf{cggoto}
 condition, label


  where \emph{label}
 is in the same instrument block and is not an expression, and where \emph{R}
 is one of the Relational operators (\emph{$<$}
,\emph{ =}
, \emph{$<$=}
, \emph{==}
, \emph{!=}
) (and \emph{=}
 for convenience, see also under \emph{Conditional Values}
). 
\subsection*{Examples}


  Here is an example of the cggoto opcode. It uses the files \emph{cggoto.orc}
 and \emph{cggoto.sco}
. 


 \textbf{Example 1. Example of the cggoto opcode.}

\begin{lstlisting}
/* cggoto.orc */
; Initialize the global variables.
sr = 44100
kr = 4410
ksmps = 10
nchnls = 1

; Instrument #1.
instr 1
  i1 = 1

  ; If i1 is equal to one, play a high note.
  ; Otherwise play a low note.
  cggoto (i1 == 1), highnote

lownote:
  a1 oscil 10000, 220, 1
  goto playit
  
highnote:
  a1 oscil 10000, 440, 1
  goto playit

playit:
  out a1
endin
/* cggoto.orc */
        
\end{lstlisting}
\begin{lstlisting}
/* cggoto.sco */
; Table #1: a simple sine wave.
f 1 0 32768 10 1

; Play Instrument #1 for one second.
i 1 0 1
e
/* cggoto.sco */
        
\end{lstlisting}
\subsection*{See Also}


 \emph{cigoto}
, \emph{ckgoto}
, \emph{cngoto}
, \emph{if}
, \emph{igoto}
, \emph{kgoto}
, \emph{tigoto}
, \emph{timout}

\subsection*{Credits}


 Added a note by Jim Aikin.


 Example written by Kevin Conder.
%\hline 


\begin{comment}
\begin{tabular}{lcr}
Previous &Home &Next \\
cent &Up &chanctrl

\end{tabular}


\end{document}
\end{comment}
