\begin{comment}
\documentclass[10pt]{article}
\usepackage{fullpage, graphicx, url}
\setlength{\parskip}{1ex}
\setlength{\parindent}{0ex}
\title{specsum}
\begin{document}


\begin{tabular}{ccc}
The Alternative Csound Reference Manual & & \\
Previous & &Next

\end{tabular}

%\hline 
\end{comment}
\section{specsum}
specsum�--� Sums the magnitudes across all channels of the spectrum. \subsection*{Description}


  Sums the magnitudes across all channels of the spectrum. 
\subsection*{Syntax}


 ksum \textbf{specsum}
 wsig [, interp]
\subsection*{Initialization}


 \emph{interp}
 (optional, default-0) -- if non-zero, interpolate the output signal (\emph{koct }
or \emph{ksum}
). The default value is 0 (repeat the signal value between changes). 
\subsection*{Performance}


 \emph{ksum}
 -- the output signal. 


 \emph{wsig}
 -- the input spectrum. 


  Sums the magnitudes across all channels of the spectrum. At each new frame of \emph{wsig}
, the magnitudes are summed and released as a scalar \emph{ksum}
 signal. Between frames, the output is either repeated or interpolated at the k-rate. This unit produces a k-signal summation of the magnitudes present in the spectral data, and is thereby a running measure of its moment-to-moment overall strength. 
\subsection*{Examples}


 


 
\begin{lstlisting}
  ksum     \emph{specsum}
   wsig,  1                    ; sum the spec bins, and ksmooth
           \emph{if}
        ksum < 2000   \emph{kgoto}
  zero   ; if sufficient amplitude
  koct     \emph{specptrk}
  wsig                        ;    pitch-track the signal
           \emph{kgoto}
      contin
zero:  
  koct    =     0                                ; else output zero
contin:
        
\end{lstlisting}


 
\subsection*{See Also}


 \emph{specdisp}

%\hline 


\begin{comment}
\begin{tabular}{lcr}
Previous &Home &Next \\
specscal &Up &spectrum

\end{tabular}


\end{document}
\end{comment}
