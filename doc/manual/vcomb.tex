\begin{comment}
\documentclass[10pt]{article}
\usepackage{fullpage, graphicx, url}
\setlength{\parskip}{1ex}
\setlength{\parindent}{0ex}
\title{vcomb}
\begin{document}


\begin{tabular}{ccc}
The Alternative Csound Reference Manual & & \\
Previous & &Next

\end{tabular}

%\hline 
\end{comment}
\section{vcomb}
vcomb�--� Variably reverberates an input signal with a ``colored'' frequency response. \subsection*{Description}


  Variably reverberates an input signal with a ``colored'' frequency response. 
\subsection*{Syntax}


 ar \textbf{vcomb}
 asig, krvt, xlpt, imaxlpt [, iskip] [, insmps]
\subsection*{Initialization}


 \emph{imaxlpt}
 -- maximum loop time for \emph{klpt}



 \emph{iskip}
 (optional, default=0) -- initial disposition of delay-loop data space (cf. \emph{reson}
). The default value is 0. 


 \emph{insmps}
 (optional, default=0) -- delay amount, as a number of samples. 
\subsection*{Performance}


 \emph{krvt}
 -- the reverberation time (defined as the time in seconds for a signal to decay to 1/1000, or 60dB down from its original amplitude). 


 \emph{xlpt}
 -- variable loop time in seconds, same as \emph{ilpt}
 in \emph{comb}
. Loop time can be as large as \emph{imaxlpt}
. 


  This filter reiterates input with an echo density determined by loop time \emph{ilpt}
. The attenuation rate is independent and is determined by \emph{krvt}
, the reverberation time (defined as the time in seconds for a signal to decay to 1/1000, or 60dB down from its original amplitude). Output will appear only after \emph{ilpt}
 seconds. 
\subsection*{See Also}


 \emph{alpass}
, \emph{comb}
, \emph{reverb}
, \emph{valpass}

\subsection*{Credits}


 


 


\begin{tabular}{cccc}
Author: William ``Pete'' Moss &University of Texas at Austin &Austin, Texas USA &January 2002

\end{tabular}



 
%\hline 


\begin{comment}
\begin{tabular}{lcr}
Previous &Home &Next \\
vco2init &Up &vdelay

\end{tabular}


\end{document}
\end{comment}
