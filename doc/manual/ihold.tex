\begin{comment}
\documentclass[10pt]{article}
\usepackage{fullpage, graphicx, url}
\setlength{\parskip}{1ex}
\setlength{\parindent}{0ex}
\title{ihold}
\begin{document}


\begin{tabular}{ccc}
The Alternative Csound Reference Manual & & \\
Previous & &Next

\end{tabular}

%\hline 
\end{comment}
\section{ihold}
ihold�--� Creates a held note. \subsection*{Description}


  Causes a finite-duration note to become a ``held'' note 
\subsection*{Syntax}


 \textbf{ihold}

\subsection*{Performance}


 \emph{ihold}
 -- this i-time statement causes a finite-duration note to become a ``held'' note. It thus has the same effect as a negative p3 ( see score \emph{i Statement}
), except that p3 here remains positive and the instrument reclassifies itself to being held indefinitely. The note can be turned off explicitly with \emph{turnoff}
, or its space taken over by another note of the same instrument number (i.e. it is tied into that note). Effective at i-time only; no-op during a \emph{reinit}
 pass. 
\subsection*{Examples}


  Here is an example of the ihold opcode. It uses the files \emph{ihold.orc}
 and \emph{ihold.sco}
. 


 \textbf{Example 1. Example of the ihold opcode.}

\begin{lstlisting}
/* ihold.orc */
; Initialize the global variables.
sr = 44100
kr = 4410
ksmps = 10
nchnls = 1

; Instrument #1.
instr 1
  ; A simple oscillator with its note held indefinitely.
  a1 oscil 10000, 440, 1
  ihold

  ; If p4 equals 0, turn the note off.
  if (p4 == 0) kgoto offnow
    kgoto playit

offnow:
  ; Turn the note off now.
  turnoff

playit:
  ; Play the note.
  out a1
endin
/* ihold.orc */
        
\end{lstlisting}
\begin{lstlisting}
/* ihold.sco */
; Table #1: an ordinary sine wave.
f 1 0 32768 10 1

; p4 = turn the note off (if it is equal to 0).
; Start playing Instrument #1.
i 1 0 1 1
; Turn Instrument #1 off after 3 seconds.
i 1 3 1 0
e
/* ihold.sco */
        
\end{lstlisting}
\subsection*{See Also}


 \emph{i Statement}
, \emph{turnoff}

\subsection*{Credits}


 Example written by Kevin Conder.
%\hline 


\begin{comment}
\begin{tabular}{lcr}
Previous &Home &Next \\
igoto &Up &ilinrand

\end{tabular}


\end{document}
\end{comment}
