\begin{comment}
\documentclass[10pt]{article}
\usepackage{fullpage, graphicx, url}
\setlength{\parskip}{1ex}
\setlength{\parindent}{0ex}
\title{GEN11}
\begin{document}


\begin{tabular}{ccc}
The Alternative Csound Reference Manual & & \\
Previous & &Next

\end{tabular}

%\hline 
\end{comment}
\section{GEN11}
GEN11�--� Generates an additive set of cosine partials. \subsection*{Description}


  This subroutine generates an additive set of cosine partials, in the manner of Csound generators \emph{buzz}
 and \emph{gbuzz}
. 
\subsection*{Syntax}


 \textbf{f}
 \# time size 11 nh [lh] [r]
\subsection*{Initialization}


 \emph{size}
 -- number of points in the table. Must be a power of 2 or power-of-2 plus 1 (see \emph{f statement}
). 


 \emph{nh}
 -- number of harmonics requested. Must be positive. 


 \emph{lh}
(optional) -- lowest harmonic partial present. Can be positive, zero or negative. The set of partials can begin at any partial number and proceeds upwards; if \emph{lh}
 is negative, all partials below zero will reflect in zero to produce positive partials without phase change (since cosine is an even function), and will add constructively to any positive partials in the set. The default value is 1 


 \emph{r}
(optional) -- multiplier in an amplitude coefficient series. This is a power series: if the \emph{lh}
th partial has a strength coefficient of A the (\emph{lh}
 + n)th partial will have a coefficient of A * rn, i.e. strength values trace an exponential curve. \emph{r}
 may be positive, zero or negative, and is not restricted to integers. The default value is 1. 


 


\begin{tabular}{cc}
\textbf{Note}
 \\
� &

 


 
\begin{itemize}
\item 

  This subroutine is a non-time-varying version of the CSound \emph{buzz}
and \emph{gbuzz}
 generators, and is similarly useful as a complex sound source in subtractive synthesis. With \emph{lh}
 and \emph{r}
 present it parallels \emph{gbuzz}
; with both absent or equal to 1 it reduces to the simpler \emph{buzz}
 (i.e. \emph{nh}
 equal-strength harmonic partials beginning with the fundamental). 

\item 

  Sampling the stored waveform with an oscillator is more efficient than using the dynamic buzz units. However, the spectral content is invariant and care is necessary, lest the higher partials exceed the Nyquist during sampling to produce fold-over. 


\end{itemize}


\end{tabular}

\subsection*{Examples}


  Here is a simple example of the GEN11 routine. It uses the files \emph{gen11.orc}
 and \emph{gen11.sco}
. It will generate a simple cosine wave. Here is its diagram: 


 \includegraphics[scale=1]{gen11} 


 Diagram of the waveform generated by GEN11.


 \textbf{Example 1. A simple example of the GEN11 routine.}

\begin{lstlisting}
/* gen11.orc */
; Initialize the global variables.
sr = 44100
kr = 4410
ksmps = 10
nchnls = 1

; Instrument #1.
instr 1
  kamp = 30000
  kcps = 440
  ifn = 1

  ; Play the cosine wave stored in Table #1.
  a1 oscil kamp, kcps, ifn
  out a1
endin
/* gen11.orc */
        
\end{lstlisting}
\begin{lstlisting}
/* gen11.sco */
; Table #1: a simple cosine wave (using GEN11).
f 1 0 16384 11 1 1

; Play Instrument #1 for 2 seconds.
i 1 0 2
e
/* gen11.sco */
        
\end{lstlisting}
\subsection*{See Also}


 \emph{GEN10}

\subsection*{Credits}


 Example written by Kevin Conder
%\hline 


\begin{comment}
\begin{tabular}{lcr}
Previous &Home &Next \\
GEN10 &Up &GEN12

\end{tabular}


\end{document}
\end{comment}
