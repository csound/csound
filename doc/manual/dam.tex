\begin{comment}
\documentclass[10pt]{article}
\usepackage{fullpage, graphicx, url}
\setlength{\parskip}{1ex}
\setlength{\parindent}{0ex}
\title{dam}
\begin{document}


\begin{tabular}{ccc}
The Alternative Csound Reference Manual & & \\
Previous & &Next

\end{tabular}

%\hline 
\end{comment}
\section{dam}
dam�--� A dynamic compressor/expander. \subsection*{Description}


  This opcode dynamically modifies a gain value applied to the input sound \emph{ain}
 by comparing its power level to a given threshold level. The signal will be compressed/expanded with different factors regarding that it is over or under the threshold. 
\subsection*{Syntax}


 ar \textbf{dam}
 asig, kthreshold, icomp1, icomp2, irtime, iftime
\subsection*{Initialization}


 \emph{icomp1}
 -- compression ratio for upper zone. 


 \emph{icomp2}
 -- compression ratio for lower zone 


 \emph{irtime}
 -- gain rise time in seconds. Time over which the gain factor is allowed to raise of one unit. 


 \emph{iftime}
 -- gain fall time in seconds. Time over which the gain factor is allowed to decrease of one unit. 
\subsection*{Performance}


 \emph{asig}
 -- input signal to be modified 


 \emph{kthreshold}
 -- level of input signal which acts as the threshold. Can be changed at k-time (e.g. for ducking) 


  Note on the compression factors: A compression ratio of one leaves the sound unchanged. Setting the ratio to a value smaller than one will compress the signal (reduce its volume) while setting the ratio to a value greater than one will expand the signal (augment its volume). 
\subsection*{Examples}


  Because the results of the \emph{dam}
 opcode can be subtle, I recommend looking at them in a graphical audio editor program like \emph{audacity}
. \emph{audacity}
 is available for Linux, Windows, and the MacOS and may be downloaded from \emph{\url{http://audacity.sourceforge.net}}
. 


  Here is an example of the dam opcode. It uses the files \emph{dam.orc}
, \emph{dam.sco}
, and \emph{beats.wav}
. 


 \textbf{Example 1. An example of the dam opcode compressing an audio signal.}

\begin{lstlisting}
/* dam.orc */
; Initialize the global variables.
sr = 44100
kr = 4410
ksmps = 10
nchnls = 1

; Instrument #1, uncompressed signal.
instr 1
  ; Use the "beats.wav" audio file.
  asig soundin "beats.wav"

  out asig
endin

; Instrument #2, compressed signal.
instr 2
  ; Use the "beats.wav" audio file.
  asig soundin "beats.wav"

  ; Compress the audio signal.
  kthreshold init 25000
  icomp1 = 0.5
  icomp2 = 0.763
  irtime = 0.1
  iftime = 0.1
  a1 dam asig, kthreshold, icomp1, icomp2, irtime, iftime

  out a1
endin
/* dam.orc */
        
\end{lstlisting}
\begin{lstlisting}
/* dam.sco */
; Play Instrument #1 for 2 seconds.
i 1 0 2
; Play Instrument #2 for 2 seconds.
i 2 2 2
e
/* dam.sco */
        
\end{lstlisting}
 This example compresses the audio file ``beats.wav''. You should hear a drum pattern repeat twice. The second time, the sound should be quieter (compressed) than the first. 

  Here is another example of the dam opcode. It uses the files \emph{dam\_expanded.orc}
, \emph{dam\_expanded.sco}
, and \emph{mary.wav}
. 


 \textbf{Example 2. An example of the dam opcode expanding an audio signal.}

\begin{lstlisting}
/* dam_expanded.orc */
; Initialize the global variables.
sr = 44100
kr = 4410
ksmps = 10
nchnls = 1

; Instrument #1, normal audio signal.
instr 1
  ; Use the "mary.wav" audio file.
  asig soundin "mary.wav"

  out asig
endin

; Instrument #2, expanded audio signal.
instr 2
  ; Use the "mary.wav" audio file.
  asig soundin "mary.wav"

  ; Expand the audio signal.
  kthreshold init 7500
  icomp1 = 2.25
  icomp2 = 2.25
  irtime = 0.1
  iftime = 0.6
  a1 dam asig, kthreshold, icomp1, icomp2, irtime, iftime

  out a1
endin
/* dam_expanded.orc */
        
\end{lstlisting}
\begin{lstlisting}
/* dam_expanded.sco */
; Play Instrument #1.
i 1 0.0 3.5
; Play Instrument #2.
i 2 3.5 3.5
e
/* dam_expanded.sco */
        
\end{lstlisting}
 This example expands the audio file ``mary.wav''. You should hear a melody repeat twice. The second time, the sound should be louder (expanded) than the first. \subsection*{Credits}


 


 


\begin{tabular}{ccc}
Author: Marc Resibois &Belgium &1997

\end{tabular}



 


 Examples written by Kevin Conder.
%\hline 


\begin{comment}
\begin{tabular}{lcr}
Previous &Home &Next \\
cuserrnd &Up &db

\end{tabular}


\end{document}
\end{comment}
