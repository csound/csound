\begin{comment}
\documentclass[10pt]{article}
\usepackage{fullpage, graphicx, url}
\setlength{\parskip}{1ex}
\setlength{\parindent}{0ex}
\title{spectrum}
\begin{document}


\begin{tabular}{ccc}
The Alternative Csound Reference Manual & & \\
Previous & &Next

\end{tabular}

%\hline 
\end{comment}
\section{spectrum}
spectrum�--� Generate a constant-Q, exponentially-spaced DFT. \subsection*{Description}


  Generate a constant-Q, exponentially-spaced DFT across all octaves of a multiply-downsampled control or audio input signal. 
\subsection*{Syntax}


 wsig \textbf{spectrum}
 xsig, iprd, iocts, ifrqa [, iq] [, ihann] [, idbout] [, idsprd] [, idsinrs]
\subsection*{Initialization}


 \emph{ihann}
 (optional) -- apply a Hamming or Hanning window to the input. The default is 0 (Hamming window) 


 \emph{idbout}
 (optional) -- coded conversion of the DFT output: 


 
\begin{itemize}
\item 

 0 = magnitude

\item 

 1 = dB

\item 

 2 = mag squared

\item 

 3 = root magnitude


\end{itemize}
 The default value is 0 (magnitude). 

 \emph{idisprd}
 (optional) -- if non-zero, display the composite downsampling buffer every \emph{idisprd}
 seconds. The default value is 0 (no display). 


 \emph{idsines}
 (optional) -- if non-zero, display the Hamming or Hanning windowed sinusoids used in DFT filtering. The default value is 0 (no sinusoid display). 
\subsection*{Performance}


  This unit first puts signal \emph{asig}
 or \emph{ksig}
 through \emph{iocts}
 of successive octave decimation and downsampling, and preserves a buffer of down-sampled values in each octave (optionally displayed as a composite buffer every \emph{idisprd}
 seconds). Then at every \emph{iprd}
 seconds, the preserved samples are passed through a filter bank (\emph{ifrqs}
 parallel filters per octave, exponentially spaced, with frequency/bandwidth Q of \emph{iq}
), and the output magnitudes optionally converted (\emph{idbout }
) to produce a band-limited spectrum that can be read by other units. 


  The stages in this process are computationally intensive, and computation time varies directly with \emph{iocts}
, \emph{ifrqs}
, \emph{iq}
, and inversely with \emph{iprd}
. Settings of \emph{ifrqs}
 = 12, \emph{iq}
 = 10, \emph{idbout}
 = 3, and \emph{iprd}
 = .02 will normally be adequate, but experimentation is encouraged. \emph{ifrqs}
 currently has a maximum of 120 divisions per octave. For audio input, the frequency bins are tuned to coincide with A440. 


  This unit produces a self-defining spectral datablock \emph{wsig}
, whose characteristics used (\emph{iprd}
, \emph{iocts}
, \emph{ifrqs}
, \emph{idbout}
) are passed via the data block itself to all derivative \emph{wsigs}
. There can be any number of spectrum units in an instrument or orchestra, but all \emph{wsig}
 names must be unique. 
\subsection*{Examples}


 


 
\begin{lstlisting}
asig \emph{in}
                                ; get external audio
wsig \emph{spectrum}
  asig,.02,6,12,33,0,1,1  ; downsample in 6 octs & calc a 72 pt dft (Q 33, dB out) every 20 msecs
        
\end{lstlisting}


 
%\hline 


\begin{comment}
\begin{tabular}{lcr}
Previous &Home &Next \\
specsum &Up &spsend

\end{tabular}


\end{document}
\end{comment}
