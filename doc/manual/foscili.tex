\begin{comment}
\documentclass[10pt]{article}
\usepackage{fullpage, graphicx, url}
\setlength{\parskip}{1ex}
\setlength{\parindent}{0ex}
\title{foscili}
\begin{document}


\begin{tabular}{ccc}
The Alternative Csound Reference Manual & & \\
Previous & &Next

\end{tabular}

%\hline 
\end{comment}
\section{foscili}
foscili�--� Basic frequency modulated oscillator with linear interpolation. \subsection*{Description}


  Basic frequency modulated oscillator with linear interpolation. 
\subsection*{Syntax}


 ar \textbf{foscili}
 xamp, kcps, xcar, xmod, kndx, ifn [, iphs]
\subsection*{Initialization}


 \emph{ifn}
 -- function table number. Requires a wrap-around guard point. 


 \emph{iphs}
 (optional, default=0) -- initial phase of waveform in table \emph{ifn}
, expressed as a fraction of a cycle (0 to 1). A negative value will cause phase initialization to be skipped. The default value is 0. 
\subsection*{Performance}


 \emph{xamp}
 -- the amplitude of the output signal. 


 \emph{kcps}
 -- the frequency of the output signal measured in cycles per second. 


 \emph{xcar}
 -- the carrier frequency. 


 \emph{xmod}
 -- the modulating frequency. 


 \emph{kndx}
 -- the modulation index. 


 \emph{foscili}
 differs from \emph{foscil}
 in that the standard procedure of using a truncated phase as a sampling index is here replaced by a process that interpolates between two successive lookups. Interpolating generators will produce a noticeably cleaner output signal, but they may take as much as twice as long to run. Adequate accuracy can also be gained without the time cost of interpolation by using large stored function tables of 2K, 4K or 8K points if the space is available. 
\subsection*{Examples}


  Here is an example of the foscili opcode. It uses the files \emph{foscili.orc}
 and \emph{foscili.sco}
. 


 \textbf{Example 1. Example of the foscili opcode.}

\begin{lstlisting}
/* foscili.orc */
; Initialize the global variables.
sr = 44100
kr = 4410
ksmps = 10
nchnls = 1

; Instrument #1 - a basic FM waveform.
instr 1
  kamp = 10000
  kcps = 440
  kcar = 600
  kmod = 210
  kndx = 2
  ifn = 1

  a1 foscil kamp, kcps, kcar, kmod, kndx, ifn
  out a1
endin

; Instrument #2 - the basic FM waveform with extra interpolation.
instr 2
  kamp = 10000
  kcps = 440
  kcar = 600
  kmod = 210
  kndx = 2
  ifn = 1

  a1 foscili kamp, kcps, kcar, kmod, kndx, ifn
  out a1
endin
/* foscili.orc */
        
\end{lstlisting}
\begin{lstlisting}
/* foscili.sco */
; Table #1, a sine wave table with a small amount of data.
f 1 0 4096 10 1

; Play Instrument #1, the basic FM instrument, for 
; two seconds. This should sound relatively rough.
i 1 0 2

; Play Instrument #2, the interpolated FM instrument, for
; two seconds. This should sound relatively smooth.
i 2 2 2
e
/* foscili.sco */
        
\end{lstlisting}
\subsection*{Credits}


 Example written by Kevin Conder.
%\hline 


\begin{comment}
\begin{tabular}{lcr}
Previous &Home &Next \\
foscil &Up &fout

\end{tabular}


\end{document}
\end{comment}
