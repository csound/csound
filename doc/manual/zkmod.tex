\begin{comment}
\documentclass[10pt]{article}
\usepackage{fullpage, graphicx, url}
\setlength{\parskip}{1ex}
\setlength{\parindent}{0ex}
\title{zkmod}
\begin{document}


\begin{tabular}{ccc}
The Alternative Csound Reference Manual & & \\
Previous & &Next

\end{tabular}

%\hline 
\end{comment}
\section{zkmod}
zkmod�--� Facilitates the modulation of one signal by another. \subsection*{Description}


  Facilitates the modulation of one signal by another. 
\subsection*{Syntax}


 kr \textbf{zkmod}
 ksig, kzkmod
\subsection*{Performance}


 \emph{ksig}
 -- the input signal 


 \emph{kzkmod}
 -- controls which zk variable is used for modulation. A positive value means additive modulation, a negative value means multiplicative modulation. A value of 0 means no change to \emph{ksig}
. \emph{kzkmod}
 can be i-rate or k-rate 


 \emph{zkmod}
 facilitates the modulation of one signal by another, where the modulating signal comes from a zk variable. Either additive or mulitiplicative modulation can be specified. 
\subsection*{Examples}


  Here is an example of the zkmod opcode. It uses the files \emph{zkmod.orc}
 and \emph{zkmod.sco}
. 


 \textbf{Example 1. Example of the zkmod opcode.}

\begin{lstlisting}
/* zkmod.orc */
; Initialize the global variables.
sr = 44100
kr = 4410
ksmps = 10
nchnls = 2

; Initialize the ZAK space.
; Create 2 a-rate variables and 2 k-rate variables.
zakinit 2, 2

; Instrument #1 -- a signal with jitter.
instr 1
  ; Generate a k-rate signal goes from 30 to 2,000.
  kline line 30, p3, 2000

  ; Add the signal into zk variable #1.
  zkw kline, 1
endin

; Instrument #2 -- generates audio output.
instr 2
  ; Create a k-rate signal modulated the jitter opcode.
  kamp init 20
  kcpsmin init 40
  kcpsmax init 60
  kjtr jitter kamp, kcpsmin, kcpsmax
  
  ; Get the frequency values from zk variable #1.
  kfreq zkr 1
  ; Add the the frequency values in zk variable #1 to 
  ; the jitter signal.
  kjfreq zkmod kjtr, 1

  ; Use a simple sine waveform for the left speaker.
  aleft oscil 20000, kfreq, 1
  ; Use a sine waveform with jitter for the right speaker.
  aright oscil 20000, kjfreq, 1

  ; Generate the audio output.
  outs aleft, aright

  ; Clear the zk variables, prepare them for 
  ; another pass.
  zkcl 0, 2
endin
/* zkmod.orc */
        
\end{lstlisting}
\begin{lstlisting}
/* zkmod.sco */
; Table #1, a sine wave.
f 1 0 16384 10 1

; Play Instrument #1 for 2 seconds.
i 1 0 2
; Play Instrument #2 for 2 seconds.
i 2 0 2
e
/* zkmod.sco */
        
\end{lstlisting}
\subsection*{See Also}


 \emph{zamod}
, \emph{zkcl}
, \emph{zkr}
, \emph{zkwm}
, \emph{zkw}

\subsection*{Credits}


 


 


\begin{tabular}{ccc}
Author: Robin Whittle &Australia &May 1997

\end{tabular}



 


 Example written by Kevin Conder.
%\hline 


\begin{comment}
\begin{tabular}{lcr}
Previous &Home &Next \\
zkcl &Up &zkr

\end{tabular}


\end{document}
\end{comment}
