\begin{comment}
\documentclass[10pt]{article}
\usepackage{fullpage, graphicx, url}
\setlength{\parskip}{1ex}
\setlength{\parindent}{0ex}
\title{sfilist}
\begin{document}


\begin{tabular}{ccc}
The Alternative Csound Reference Manual & & \\
Previous & &Next

\end{tabular}

%\hline 
\end{comment}
\section{sfilist}
sfilist�--� Prints a list of all instruments of a previously loaded SoundFont2 (SF2) file. \subsection*{Description}


  Prints a list of all instruments of a previously loaded SoundFont2 (SF2) sample file. These opcodes allow management the sample-structure of SF2 files. In order to understand the usage of these opcodes, the user must have some knowledge of the SF2 format, so a brief description of this format can be found in the \emph{SoundFont2 File Format Appendix}
. 
\subsection*{Syntax}


 \textbf{sfilist}
 ifilhandle
\subsection*{Initialization}


 \emph{ifilhandle}
 -- unique number generated by \emph{sfload}
 opcode to be used as an identifier for a SF2 file. Several SF2 files can be loaded and activated at the same time. 
\subsection*{Performance}


 \emph{sfilist}
 prints a list of all instruments of a previously loaded SF2 file to the console. 


  These opcodes only support the sample structure of SF2 files. The modulator structure of the SoundFont2 format is not supported in Csound. Any modulation or processing to the sample data is left to the Csound user, bypassing all restrictions forced by the SF2 standard. 
\subsection*{See Also}


 \emph{sfinstr}
, \emph{sfinstrm}
, \emph{sfload}
, \emph{sfpassign}
, \emph{sfplay}
, \emph{sfplaym}
, \emph{sfplist}
, \emph{sfpreset}

\subsection*{Credits}


 


 


\begin{tabular}{ccc}
Author: Gabriel Maldonado &Italy &May 2000

\end{tabular}



 


 New in Csound Version 4.07
%\hline 


\begin{comment}
\begin{tabular}{lcr}
Previous &Home &Next \\
setksmps &Up &sfinstr

\end{tabular}


\end{document}
\end{comment}
