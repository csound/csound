\begin{comment}
\documentclass[10pt]{article}
\usepackage{fullpage, graphicx, url}
\setlength{\parskip}{1ex}
\setlength{\parindent}{0ex}
\title{wgflute}
\begin{document}


\begin{tabular}{ccc}
The Alternative Csound Reference Manual & & \\
Previous & &Next

\end{tabular}

%\hline 
\end{comment}
\section{wgflute}
wgflute�--� Creates a tone similar to a flute. \subsection*{Description}


  Audio output is a tone similar to a flute, using a physical model developed from Perry Cook, but re-coded for Csound. 
\subsection*{Syntax}


 ar \textbf{wgflute}
 kamp, kfreq, kjet, iatt, idetk, kngain, kvibf, kvamp, ifn [, iminfreq] [, ijetrf] [, iendrf]
\subsection*{Initialization}


 \emph{iatt}
 -- time in seconds to reach full blowing pressure. 0.1 seems to correspond to reasonable playing. 


 \emph{idetk}
 -- time in seconds taken to stop blowing. 0.1 is a smooth ending 


 \emph{ifn}
 -- table of shape of vibrato, usually a sine table, created by a function 


 \emph{iminfreq}
 (optional) -- lowest frequency at which the instrument will play. If it is omitted it is taken to be the same as the initial kfreq. If \emph{iminfreq}
 is negative, initialization will be skipped. 


 \emph{ijetrf}
 (optional, default=0.5) -- amount of reflection in the breath jet that powers the flute. Default value is 0.5. 


 \emph{iendrf}
 (optional, default=0.5) -- reflection coefficient of the breath jet. Default value is 0.5. Both \emph{ijetrf}
 and \emph{iendrf}
 are used in the calculation of the pressure differential. 
\subsection*{Performance}


 \emph{kamp}
 -- Amplitude of note. 


 \emph{kfreq}
 -- Frequency of note played. While it can be varied in performance, I have not tried it. 


 \emph{kjet}
 -- a parameter controlling the air jet. Values should be positive, and about 0.3. The useful range is approximately 0.08 to 0.56. 


 \emph{kngain}
 -- amplitude of the noise component, about 0 to 0.5 


 \emph{kvibf}
 -- frequency of vibrato in Hertz. Suggested range is 0 to 12 


 \emph{kvamp}
 -- amplitude of the vibrato 
\subsection*{Examples}


  Here is an example of the wgflute opcode. It uses the files \emph{wgflute.orc}
 and \emph{wgflute.sco}
. 


 \textbf{Example 1. Example of the wgflute opcode.}

\begin{lstlisting}
/* wgflute.orc */
; Initialize the global variables.
sr = 44100
kr = 4410
ksmps = 10
nchnls = 1

; Instrument #1.
instr 1
  kamp = 31129.60
  kfreq = 440
  kjet = 0.32
  iatt = 0.1
  idetk = 0.1
  kngain = 0.15
  kvibf = 5.925
  kvamp = 0.05
  ifn = 1

  a1 wgflute kamp, kfreq, kjet, iatt, idetk, kngain, kvibf, kvamp, ifn
  out a1
endin
/* wgflute.orc */
        
\end{lstlisting}
\begin{lstlisting}
/* wgflute.sco */
; Table #1, a sine wave.
f 1 0 16384 10 1

; Play Instrument #1 for two seconds.
i 1 0 2
e
/* wgflute.sco */
        
\end{lstlisting}
\subsection*{Credits}


 


 


\begin{tabular}{ccc}
Author: John ffitch (after Perry Cook) &University of Bath, Codemist Ltd. &Bath, UK

\end{tabular}



 


 New in Csound version 3.47
%\hline 


\begin{comment}
\begin{tabular}{lcr}
Previous &Home &Next \\
wgclar &Up &wgpluck

\end{tabular}


\end{document}
\end{comment}
