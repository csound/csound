\begin{comment}
\documentclass[10pt]{article}
\usepackage{fullpage, graphicx, url}
\setlength{\parskip}{1ex}
\setlength{\parindent}{0ex}
\title{gain}
\begin{document}


\begin{tabular}{ccc}
The Alternative Csound Reference Manual & & \\
Previous & &Next

\end{tabular}

%\hline 
\end{comment}
\section{gain}
gain�--� Adjusts the amplitude audio signal according to a root-mean-square value. \subsection*{Description}


  Adjusts the amplitude audio signal according to a root-mean-square value. 
\subsection*{Syntax}


 ar \textbf{gain}
 asig, krms [, ihp] [, iskip]
\subsection*{Initialization}


 \emph{ihp}
 (optional, default=10) -- half-power point (in Hz) of a special internal low-pass filter. The default value is 10. 


 \emph{iskip}
 (optional, default=0) -- initial disposition of internal data space (see \emph{reson}
). The default value is 0. 
\subsection*{Performance}


 \emph{asig}
 -- input audio signal 


 \emph{gain}
 provides an amplitude modification of \emph{asig}
 so that the output \emph{ar}
 has rms power equal to \emph{krms}
. \emph{rms}
 and \emph{gain}
 used together (and given matching \emph{ihp}
 values) will provide the same effect as \emph{balance}
. 
\subsection*{Examples}


 


 
\begin{lstlisting}
asrc \emph{buzz}
    10000,440, sr/440, 1 ; band-limited pulse train
a1   \emph{reson}
   asrc, 1000,100       ; sent through
a2   \emph{reson}
   a1,3000,500          ; 2 filters
afin \emph{balance}
 a2, asrc             ; then balanced with source
        
\end{lstlisting}


 
\subsection*{See Also}


 \emph{balance}
, \emph{rms}

%\hline 


\begin{comment}
\begin{tabular}{lcr}
Previous &Home &Next \\
ftsr &Up &gauss

\end{tabular}


\end{document}
\end{comment}
