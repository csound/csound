\begin{comment}
\documentclass[10pt]{article}
\usepackage{fullpage, graphicx, url}
\setlength{\parskip}{1ex}
\setlength{\parindent}{0ex}
\title{vibr}
\begin{document}


\begin{tabular}{ccc}
The Alternative Csound Reference Manual & & \\
Previous & &Next

\end{tabular}

%\hline 
\end{comment}
\section{vibr}
vibr�--� Easier-to-use user-controllable vibrato. \subsection*{Description}


  Easier-to-use user-controllable vibrato. 
\subsection*{Syntax}


 kout \textbf{vibr}
 kAverageAmp, kAverageFreq, ifn
\subsection*{Initialization}


 \emph{ifn}
 -- Number of vibrato table. It normally contains a sine or a triangle wave. 
\subsection*{Performance}


 \emph{kAverageAmp}
 -- Average amplitude value of vibrato 


 \emph{kAverageFreq}
 -- Average frequency value of vibrato (in cps) 


 \emph{vibr}
 is an easier-to-use version of \emph{vibrato}
. It has the same generation-engine of \emph{vibrato}
, but the parameters corresponding to missing input arguments are hard-coded to default values. 
\subsection*{Examples}


  Here is an example of the vibr opcode. It uses the files \emph{vibr.orc}
 and \emph{vibr.sco}
. 


 \textbf{Example 1. Example of the vibr opcode.}

\begin{lstlisting}
/* vibr.orc */
; Initialize the global variables.
sr = 44100
kr = 4410
ksmps = 10
nchnls = 1

; Instrument #1.
instr 1
  ; Create a vibrato waveform.
  kaverageamp init 7500
  kaveragefreq init 5
  ifn = 1
  kvamp vibr kaverageamp, kaveragefreq, ifn

  ; Generate a tone including the vibrato.
  a1 oscili 10000+kvamp, 440, 2

  out a1
endin
/* vibr.orc */
        
\end{lstlisting}
\begin{lstlisting}
/* vibr.sco */
; Table #1, a sine wave for the vibrato.
f 1 0 256 10 1
; Table #1, a sine wave for the oscillator.
f 2 0 16384 10 1

; Play Instrument #1 for 2 seconds.
i 1 0 2
e
/* vibr.sco */
        
\end{lstlisting}
\subsection*{See Also}


 \emph{jitter}
, \emph{jitter2}
, \emph{vibrato}

\subsection*{Credits}


 Author: Gabriel Maldonado


 Example written by Kevin Conder.


 New in Version 4.15
%\hline 


\begin{comment}
\begin{tabular}{lcr}
Previous &Home &Next \\
vibes &Up &vibrato

\end{tabular}


\end{document}
\end{comment}
