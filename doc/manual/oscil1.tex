\begin{comment}
\documentclass[10pt]{article}
\usepackage{fullpage, graphicx, url}
\setlength{\parskip}{1ex}
\setlength{\parindent}{0ex}
\title{oscil1}
\begin{document}


\begin{tabular}{ccc}
The Alternative Csound Reference Manual & & \\
Previous & &Next

\end{tabular}

%\hline 
\end{comment}
\section{oscil1}
oscil1�--� Accesses table values by incremental sampling. \subsection*{Description}


  Accesses table values by incremental sampling. 
\subsection*{Syntax}


 kr \textbf{oscil1}
 idel, kamp, idur, ifn
\subsection*{Initialization}


 \emph{idel}
 -- delay in seconds before \emph{oscil1}
 incremental sampling begins. 


 \emph{idur}
 -- duration in seconds to sample through the \emph{oscil1}
 table just once. A zero or negative value will cause all initialization to be skipped. 


 \emph{ifn}
 -- function table number. \emph{tablei, oscil1i}
 require the extended guard point. 
\subsection*{Performance}


 \emph{kamp}
 -- amplitude factor. 


 \emph{oscil1}
 accesses values by sampling once through the function table at a rate determined by \emph{idur}
. For the first \emph{idel}
 seconds, the point of scan will reside at the first location of the table; it will then begin moving through the table at a constant rate, reaching the end in another \emph{idur}
 seconds; from that time on (i.e. after \emph{idel}
 + \emph{idur}
 seconds) it will remain pointing at the last location. Each value obtained from sampling is then multiplied by an amplitude factor \emph{kamp}
 before being written into the result. 
\subsection*{See Also}


 \emph{table}
, \emph{tablei}
, \emph{table3}
, \emph{oscil1i}
, \emph{osciln}

%\hline 


\begin{comment}
\begin{tabular}{lcr}
Previous &Home &Next \\
oscil &Up &oscil1i

\end{tabular}


\end{document}
\end{comment}
