\begin{comment}
\documentclass[10pt]{article}
\usepackage{fullpage, graphicx, url}
\setlength{\parskip}{1ex}
\setlength{\parindent}{0ex}
\title{fold}
\begin{document}


\begin{tabular}{ccc}
The Alternative Csound Reference Manual & & \\
Previous & &Next

\end{tabular}

%\hline 
\end{comment}
\section{fold}
fold�--� Adds artificial foldover to an audio signal. \subsection*{Description}


  Adds artificial foldover to an audio signal. 
\subsection*{Syntax}


 ar \textbf{fold}
 asig, kincr
\subsection*{Performance}


 \emph{asig}
 -- input signal 


 \emph{kincr}
 -- amount of foldover expressed in multiple of sampling rate. Must be $>$= 1 


 \emph{fold}
 is an opcode which creates artificial foldover. For example, when \emph{kincr}
 is equal to 1 with sr=44100, no foldover is added. When \emph{kincr}
 is set to 2, the foldover is equivalent to a downsampling to 22050, when it is set to 4, to 11025 etc. Fractional values of \emph{kincr}
 are possible, allowing a continuous variation of foldover amount. This can be used for a wide range of special effects. 
\subsection*{Examples}


  Here is an example of the fold opcode. It uses the files \emph{fold.orc}
 and \emph{fold.sco}
. 


 \textbf{Example 1. Example of the fold opcode.}

\begin{lstlisting}
/* fold.orc */
; Initialize the global variables.
sr = 44100
kr = 4410
ksmps = 10
nchnls = 1

; Instrument #1.
instr 1
  ; Use an ordinary sine wave.
  asig oscils 30000, 100, 1

  ; Vary the fold-over amount from 1 to 200.
  kincr line 1, p3, 200
  a1 fold asig, kincr

  out a1
endin
/* fold.orc */
        
\end{lstlisting}
\begin{lstlisting}
/* fold.sco */
; Play Instrument #1 for four seconds.
i 1 0 4
e
/* fold.sco */
        
\end{lstlisting}
\subsection*{Credits}


 


 


\begin{tabular}{ccc}
Author: Gabriel Maldonado &Italy &1999

\end{tabular}



 


 New in Csound version 3.56
%\hline 


\begin{comment}
\begin{tabular}{lcr}
Previous &Home &Next \\
fog &Up &follow

\end{tabular}


\end{document}
\end{comment}
