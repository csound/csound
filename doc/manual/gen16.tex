\begin{comment}
\documentclass[10pt]{article}
\usepackage{fullpage, graphicx, url}
\setlength{\parskip}{1ex}
\setlength{\parindent}{0ex}
\title{GEN16}
\begin{document}


\begin{tabular}{ccc}
The Alternative Csound Reference Manual & & \\
Previous & &Next

\end{tabular}

%\hline 
\end{comment}
\section{GEN16}
GEN16�--� Creates a table from a starting value to an ending value. \subsection*{Description}


  Creates a table from \emph{beg}
 value to \emph{end}
 value of \emph{dur}
 steps. 
\subsection*{Syntax}


 \textbf{f}
 \# time size 16 beg dur type end
\subsection*{Initialization}


 \emph{size}
 -- number of points in the table. Must be a power of 2 or a power-of-2 plus 1 (see \emph{f statement}
). The normal value is power-of-2 plus 1. 


 \emph{beg}
 -- starting value 


 \emph{dur}
 -- number of segments 


 \emph{type}
 -- if 0, a straight line is produced. If non-zero, then \emph{GEN16}
 creates the following curve, for \emph{dur}
 steps: 


 beg�+�(end�-�beg)�*�(1�-�exp(�i*type/(dur-1)�))�/�(1�-�exp(type))\\ 
 ������


 \emph{end}
 -- value after \emph{dur}
 segments 


 


\begin{tabular}{cc}
\textbf{Note}
 \\
� &

  If \emph{type}
 $>$ 0, there is a slowly rising, fast decaying (convex) curve, while if \emph{type}
 $<$ 0, the curve is fast rising, slowly decaying (concave). See also \emph{transeg}
. 


\end{tabular}

\subsection*{Credits}


 


 


\begin{tabular}{cccc}
Author: John ffitch &University of Bath, Codemist. Ltd. &Bath, UK &October, 2000

\end{tabular}



 


 New in Csound version 4.09
%\hline 


\begin{comment}
\begin{tabular}{lcr}
Previous &Home &Next \\
GEN15 &Up &GEN17

\end{tabular}


\end{document}
\end{comment}
