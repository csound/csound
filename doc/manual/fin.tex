\begin{comment}
\documentclass[10pt]{article}
\usepackage{fullpage, graphicx, url}
\setlength{\parskip}{1ex}
\setlength{\parindent}{0ex}
\title{fin}
\begin{document}


\begin{tabular}{ccc}
The Alternative Csound Reference Manual & & \\
Previous & &Next

\end{tabular}

%\hline 
\end{comment}
\section{fin}
fin�--� Read signals from a file at a-rate. \subsection*{Description}


  Read signals from a file at a-rate. 
\subsection*{Syntax}


 \textbf{fin}
 ifilename, iskipframes, iformat, ain1 [, ain2] [, ain3] [,...]
\subsection*{Initialization}


 \emph{ifilename}
 -- input file name (can be a string or a handle number generated by fiopen) 


 \emph{iskipframes}
 -- number of frames to skip at the start (every frame contains a sample of each channel) 


 \emph{iformat}
 -- a number specifying the input file format. 


 
\begin{itemize}
\item 

 0 - 32 bit floating points without header

\item 

 1 - 16 bit integers without header


\end{itemize}
\subsection*{Performance}


 \emph{fin}
 (file input) is the complement of \emph{fout}
: it reads a multichannel file to generate audio rate signals. At the present time no header is supported for the file format. The user must be sure that the number of channels of the input file is the same as the number of \emph{ainX}
 arguments. 
\subsection*{See Also}


 \emph{fini}
, \emph{fink}

\subsection*{Credits}


 


 


\begin{tabular}{ccc}
Author: Gabriel Maldonado &Italy &1999

\end{tabular}



 


 New in Csound version 3.56
%\hline 


\begin{comment}
\begin{tabular}{lcr}
Previous &Home &Next \\
filter2 &Up &fini

\end{tabular}


\end{document}
\end{comment}
