\begin{comment}
\documentclass[10pt]{article}
\usepackage{fullpage, graphicx, url}
\setlength{\parskip}{1ex}
\setlength{\parindent}{0ex}
\title{lowresx}
\begin{document}


\begin{tabular}{ccc}
The Alternative Csound Reference Manual & & \\
Previous & &Next

\end{tabular}

%\hline 
\end{comment}
\section{lowresx}
lowresx�--� Simulates layers of serially connected resonant lowpass filters. \subsection*{Description}


 \emph{lowresx}
 is equivalent to more layers of \emph{lowres}
 with the same arguments serially connected. 
\subsection*{Syntax}


 ar \textbf{lowresx}
 asig, kcutoff, kresonance [, inumlayer] [, iskip]
\subsection*{Initialization}


 \emph{inumlayer}
 -- number of elements in a \emph{lowresx}
 stack. Default value is 4. There is no maximum. 


 \emph{iskip}
 -- initial disposition of internal data space. A zero value will clear the space; a non-zero value will allow previous information to remain. The default value is 0. 
\subsection*{Performance}


 \emph{asig}
 -- input signal 


 \emph{kcutoff}
 -- filter cutoff frequency point 


 \emph{kresonance}
 -- resonance amount 


 \emph{lowresx}
 is equivalent to more layer of \emph{lowres}
 with the same arguments serially connected. Using a stack of a larger number of filters allows a sharper cutoff. This is faster than using a larger number of instances of \emph{lowres}
 in a Csound orchestra because only one initialization and k cycle are needed at time and the audio loop falls entirely inside the cache memory of processor. Based on an orchestra by Hans Mikelson 
\subsection*{Examples}


  Here is an example of the lowresx opcode. It uses the files \emph{lowresx.orc}
, \emph{lowresx.sco}
, and \emph{beats.wav}
. 


 \textbf{Example 1. Example of the lowresx opcode.}

\begin{lstlisting}
/* lowresx.orc */
; Initialize the global variables.
sr = 44100
kr = 4410
ksmps = 10
nchnls = 1

; Instrument #1 - play the sawtooth waveform through a 
; stack of filters.
instr 1
  ; Use a nice sawtooth waveform.
  asig vco 5000, 440, 1

  ; Vary the cutoff frequency from 30 to 300 Hz.
  kcutoff line 30, p3, 300
  kresonance = 3
  inumlayer = 2

  alr lowresx asig, kcutoff, kresonance, inumlayer

  ; It gets loud, so clip the output amplitude to 30,000.
  a1 clip alr, 1, 30000
  out a1
endin
/* lowresx.orc */
        
\end{lstlisting}
\begin{lstlisting}
/* lowresx.sco */
; Table #1, a sine wave for the vco opcode.
f 1 0 16384 10 1

; Play Instrument #1 for two seconds.
i 1 0 2
e
/* lowresx.sco */
        
\end{lstlisting}
\subsection*{See Also}


 \emph{lowres}

\subsection*{Credits}


 


 


\begin{tabular}{cc}
Author: Gabriel Maldonado (adapted by John ffitch) &Italy

\end{tabular}



 


 New in Csound version 3.49
%\hline 


\begin{comment}
\begin{tabular}{lcr}
Previous &Home &Next \\
lowres &Up &lpf18

\end{tabular}


\end{document}
\end{comment}
