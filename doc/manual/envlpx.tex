\begin{comment}
\documentclass[10pt]{article}
\usepackage{fullpage, graphicx, url}
\setlength{\parskip}{1ex}
\setlength{\parindent}{0ex}
\title{envlpx}
\begin{document}


\begin{tabular}{ccc}
The Alternative Csound Reference Manual & & \\
Previous & &Next

\end{tabular}

%\hline 
\end{comment}
\section{envlpx}
envlpx�--� Applies an envelope consisting of 3 segments. \subsection*{Description}


 \emph{envlpx}
 -- apply an envelope consisting of 3 segments: 


 
\begin{enumerate}
\item 

 stored function rise shape

\item 

 modified exponential pseudo steady state

\item 

 exponential decay


\end{enumerate}
\subsection*{Syntax}


 ar \textbf{envlpx}
 xamp, irise, idur, idec, ifn, iatss, iatdec [, ixmod]


 kr \textbf{envlpx}
 kamp, irise, idur, idec, ifn, iatss, iatdec [, ixmod]
\subsection*{Initialization}


 \emph{irise}
 -- rise time in seconds. A zero or negative value signifies no rise modification. 


 \emph{idur}
 -- overall duration in seconds. A zero or negative value will cause initialization to be skipped. 


 \emph{idec}
 -- decay time in seconds. Zero means no decay. An \emph{idec}
 $>$ \emph{idur}
 will cause a truncated decay. 


 \emph{ifn}
 -- function table number of stored rise shape with extended guard point. 


 \emph{iatss}
 -- attenuation factor, by which the last value of the \emph{envlpx}
 rise is modified during the note's pseudo steady state. A factor greater than 1 causes an exponential growth and a factor less than 1 creates an exponential decay. A factor of 1 will maintain a true steady state at the last rise value. Note that this attenuation is not by fixed rate (as in a piano), but is sensitive to a note's duration. However, if \emph{iatss}
 is negative (or if steady state $<$ 4 k-periods) a fixed attenuation rate of \emph{abs}
(\emph{iatss}
) per second will be used. 0 is illegal. 


 \emph{iatdec}
 -- attenuation factor by which the closing steady state value is reduced exponentially over the decay period. This value must be positive and is normally of the order of .01. A large or excessively small value is apt to produce a cutoff which is audible. A zero or negative value is illegal. 


 \emph{ixmod}
 (optional, between +- .9 or so) -- exponential curve modifier, influencing the steepness of the exponential trajectory during the steady state. Values less than zero will cause an accelerated growth or decay towards the target (e.g. \emph{subito piano}
). Values greater than zero will cause a retarded growth or decay. The default value is zero (unmodified exponential). 
\subsection*{Performance}


 \emph{kamp, xamp}
 -- input amplitude signal. 


  Rise modifications are applied for the first \emph{irise}
 seconds, and decay from time \emph{idur - idec}
. If these periods are separated in time there will be a steady state during which \emph{amp}
 will be modified by the first exponential pattern. If the rise and decay periods overlap then that will cause a truncated decay. If the overall duration \emph{idur}
 is exceeded in performance, the final decay will continue on in the same direction, tending asymptotically to zero. 
\subsection*{Examples}


  Here is an example of the envlpx opcode. It uses the files \emph{envlpx.orc}
 and \emph{envlpx.sco}
. 


 \textbf{Example 1. Example of the envlpx opcode.}

\begin{lstlisting}
/* envlpx.orc */
; Initialize the global variables.
sr = 44100
kr = 4410
ksmps = 10
nchnls = 1

; Instrument #1 - a simple instrument.
instr 1
  ; Set the amplitude.
  kamp init 20000
  ; Get the frequency from the fourth p-field.
  kcps = cpspch(p4)

  a1 vco kamp, kcps, 1
  out a1
endin

; Instrument #2 - instrument with an amplitude envelope.
instr 2
  kamp = 20000
  irise = 0.05
  idur = p3 - .01
  idec = 0.5
  ifn = 2
  iatss = 1
  iatdec = 0.01

  ; Create an amplitude envelope.
  kenv envlpx kamp, irise, idur, idec, ifn, iatss, iatdec

  ; Get the frequency from the fourth p-field.
  kcps = cpspch(p4)

  a1 vco kenv, kcps, 1
  out a1
endin
/* envlpx.orc */
        
\end{lstlisting}
\begin{lstlisting}
/* envlpx.sco */
; Table #1, a sine wave.
f 1 0 16384 10 1
; Table #2, a rising envelope.
f 2 0 129 -7 0 128 1

; Set the tempo to 120 beats per minute.
t 0 120

; Make sure the score plays for 33 seconds.
f 0 33

; Play a melody with Instrument #1.
; p4 = frequency in pitch-class notation.
i  1  0   1  8.04
i  1  1   1  8.04
i  1  2   1  8.05
i  1  3   1  8.07
i  1  4   1  8.07
i  1  5   1  8.05
i  1  6   1  8.04
i  1  7   1  8.02
i  1  8   1  8.00
i  1  9   1  8.00
i  1  10  1  8.02
i  1  11  1  8.04
i  1  12  2  8.04
i  1  14  2  8.02

; Repeat the melody with Instrument #2.
; p4 = frequency in pitch-class notation.
i  2  16  1  8.04
i  2  17  1  8.04
i  2  18  1  8.05
i  2  19  1  8.07
i  2  20  1  8.07
i  2  21  1  8.05
i  2  22  1  8.04
i  2  23  1  8.02
i  2  24  1  8.00
i  2  25  1  8.00
i  2  26  1  8.02
i  2  27  1  8.04
i  2  28  2  8.04
i  2  30  2  8.02
e
/* envlpx.sco */
        
\end{lstlisting}
\subsection*{See Also}


 \emph{envlpxr}
, \emph{linen}
, \emph{linenr}

\subsection*{Credits}


 Thanks goes to Luis Jure for pointing out a mistake with \emph{iatss}
.


 Example written by Kevin Conder.
%\hline 


\begin{comment}
\begin{tabular}{lcr}
Previous &Home &Next \\
endop &Up &envlpxr

\end{tabular}


\end{document}
\end{comment}
