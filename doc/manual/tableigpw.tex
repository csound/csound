\begin{comment}
\documentclass[10pt]{article}
\usepackage{fullpage, graphicx, url}
\setlength{\parskip}{1ex}
\setlength{\parindent}{0ex}
\title{tableigpw}
\begin{document}


\begin{tabular}{ccc}
The Alternative Csound Reference Manual & & \\
Previous & &Next

\end{tabular}

%\hline 
\end{comment}
\section{tableigpw}
tableigpw�--� Writes a table's guard point. \subsection*{Description}


  Writes a table's guard point. 
\subsection*{Syntax}


 \textbf{tableigpw}
 ifn
\subsection*{Initialization}


 \emph{ifn}
 -- Table number to be interrogated 
\subsection*{Performance}


 \emph{tableigpw}
 -- For writing the table's guard point, with the value which is in location 0. Does nothing if table does not exist. 


  Likely to be useful after manipulating a table with \emph{tablemix}
 or \emph{tablecopy.}

\subsection*{See Also}


 \emph{tablecopy}
, \emph{tablegpw}
, \emph{tablemix}
, \emph{tableicopy}
, \emph{tableimix}

\subsection*{Credits}


 


 


\begin{tabular}{ccc}
Author: Robin Whittle &Australia &May 1997

\end{tabular}



 
%\hline 


\begin{comment}
\begin{tabular}{lcr}
Previous &Home &Next \\
tableicopy &Up &tableikt

\end{tabular}


\end{document}
\end{comment}
