\begin{comment}
\documentclass[10pt]{article}
\usepackage{fullpage, graphicx, url}
\setlength{\parskip}{1ex}
\setlength{\parindent}{0ex}
\title{filesr}
\begin{document}


\begin{tabular}{ccc}
The Alternative Csound Reference Manual & & \\
Previous & &Next

\end{tabular}

%\hline 
\end{comment}
\section{filesr}
filesr�--� Returns the sample rate of a sound file. \subsection*{Description}


  Returns the sample rate of a sound file. 
\subsection*{Syntax}


 ir \textbf{filesr}
 ifilcod
\subsection*{Initialization}


 \emph{ifilcod}
 -- sound file to be queried 
\subsection*{Performance}


 \emph{filesr}
 returns the sample rate of the sound file \emph{ifilcod}
. 
\subsection*{Examples}


  Here is an example of the filesr opcode. It uses the files \emph{filesr.orc}
, \emph{filesr.sco}
, and \emph{mary.wav}
. 


 \textbf{Example 1. Example of the filesr opcode.}

\begin{lstlisting}
/* filesr.orc */
; Initialize the global variables.
sr = 44100
kr = 4410
ksmps = 10
nchnls = 1

; Instrument #1.
instr 1
  ; Print out the sampling rate of the 
  ; audio file "mary.wav".
  isr filesr "mary.wav"
  print isr
endin
/* filesr.orc */
        
\end{lstlisting}
\begin{lstlisting}
/* filesr.sco */
; Play Instrument #1 for 1 second.
i 1 0 1
e
/* filesr.sco */
        
\end{lstlisting}
 The audio file ``mary.wav'' was sampled at 44.1 KHz. So \emph{filesr}
's output should include a line like this: \begin{lstlisting}
instr 1:  isr = 44100.000
      
\end{lstlisting}
\subsection*{See Also}


 \emph{filelen}
, \emph{filenchnls}
, \emph{filepeak}

\subsection*{Credits}


 


 


\begin{tabular}{cc}
Author: Matt Ingalls &July 1999

\end{tabular}



 


 Example written by Kevin Conder.


 New in Csound version 3.57
%\hline 


\begin{comment}
\begin{tabular}{lcr}
Previous &Home &Next \\
filepeak &Up &filter2

\end{tabular}


\end{document}
\end{comment}
