\begin{comment}
\documentclass[10pt]{article}
\usepackage{fullpage, graphicx, url}
\setlength{\parskip}{1ex}
\setlength{\parindent}{0ex}
\title{unirand}
\begin{document}


\begin{tabular}{ccc}
The Alternative Csound Reference Manual & & \\
Previous & &Next

\end{tabular}

%\hline 
\end{comment}
\section{unirand}
unirand�--� Uniform distribution random number generator (positive values only). \subsection*{Description}


  Uniform distribution random number generator (positive values only). This is an x-class noise generator. 
\subsection*{Syntax}


 ar \textbf{unirand}
 krange


 ir \textbf{unirand}
 krange


 kr \textbf{unirand}
 krange
\subsection*{Performance}


 \emph{krange}
 -- the range of the random numbers (0 - \emph{krange}
). 


  For more detailed explanation of these distributions, see: 


 
\begin{enumerate}
\item 

 C. Dodge - T.A. Jerse 1985. Computer music. Schirmer books. pp.265 - 286

\item 

 D. Lorrain. A panoply of stochastic cannons. In C. Roads, ed. 1989. Music machine . Cambridge, Massachusetts: MIT press, pp. 351 - 379.


\end{enumerate}
\subsection*{Examples}


  Here is an example of the unirand opcode. It uses the files \emph{unirand.orc}
 and \emph{unirand.sco}
. 


 \textbf{Example 1. Example of the unirand opcode.}

\begin{lstlisting}
/* unirand.orc */
; Initialize the global variables.
sr = 44100
kr = 4410
ksmps = 10
nchnls = 1

; Instrument #1.
instr 1
  ; Generate a random number between 0 and 1.
  ; krange = 1

  i1 unirand 1

  print i1
endin
/* unirand.orc */
        
\end{lstlisting}
\begin{lstlisting}
/* unirand.sco */
; Play Instrument #1 for one second.
i 1 0 1
e
/* unirand.sco */
        
\end{lstlisting}
 Its output should include lines like this: \begin{lstlisting}
instr 1:  i1 = 0.840
      
\end{lstlisting}
\subsection*{See Also}


 \emph{betarand}
, \emph{bexprnd}
, \emph{cauchy}
, \emph{exprand}
, \emph{gauss}
, \emph{linrand}
, \emph{pcauchy}
, \emph{poisson}
, \emph{trirand}
, \emph{weibull}

\subsection*{Credits}


 


 


\begin{tabular}{ccc}
Author: Paris Smaragdis &MIT, Cambridge &1995

\end{tabular}



 


 Example written by Kevin Conder.
%\hline 


\begin{comment}
\begin{tabular}{lcr}
Previous &Home &Next \\
turnon &Up &upsamp

\end{tabular}


\end{document}
\end{comment}
