\begin{comment}
\documentclass[10pt]{article}
\usepackage{fullpage, graphicx, url}
\setlength{\parskip}{1ex}
\setlength{\parindent}{0ex}
\title{FLgroupEnd}
\begin{document}


\begin{tabular}{ccc}
The Alternative Csound Reference Manual & & \\
Previous & &Next

\end{tabular}

%\hline 
\end{comment}
\section{FLgroupEnd}
FLgroupEnd�--� Marks the end of a group of FLTK child widgets. \subsection*{Description}


  Marks the end of a group of FLTK child widgets. 
\subsection*{Syntax}


 \textbf{FLgroupEnd}

\subsection*{Performance}


  Containers are useful to format the graphic appearance of the widgets. The most important container is \emph{FLpanel}
, that actually creates a window. It can be filled with other containers and/or valuators or other kinds of widgets. 


  There are no k-rate arguments in containers. 
\subsection*{See Also}


 \emph{FLgroup}
, \emph{FLpack}
, \emph{FLpackEnd}
, \emph{FLpanel}
, \emph{FLpanelEnd}
, \emph{FLscroll}
, \emph{FLscrollEnd}
, \emph{FLtabs}
, \emph{FLtabsEnd}

\subsection*{Credits}


 Author: Gabriel Maldonado


 New in version 4.22
%\hline 


\begin{comment}
\begin{tabular}{lcr}
Previous &Home &Next \\
FLgroup &Up &FLhide

\end{tabular}


\end{document}
\end{comment}
