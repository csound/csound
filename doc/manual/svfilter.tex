\begin{comment}
\documentclass[10pt]{article}
\usepackage{fullpage, graphicx, url}
\setlength{\parskip}{1ex}
\setlength{\parindent}{0ex}
\title{svfilter}
\begin{document}


\begin{tabular}{ccc}
The Alternative Csound Reference Manual & & \\
Previous & &Next

\end{tabular}

%\hline 
\end{comment}
\section{svfilter}
svfilter�--� A resonant second order filter, with simultaneous lowpass, highpass and bandpass outputs. \subsection*{Description}


  Implementation of a resonant second order filter, with simultaneous lowpass, highpass and bandpass outputs. 
\subsection*{Syntax}


 alow, ahigh, aband \textbf{svfilter}
 asig, kcf, kq [, iscl]
\subsection*{Initialization}


 \emph{iscl}
 -- coded scaling factor, similar to that in \emph{reson}
. A non-zero value signifies a peak response factor of 1, i.e. all frequencies other than \emph{kcf}
 are attenuated in accordance with the (normalized) response curve. A zero value signifies no scaling of the signal, leaving that to some later adjustment (see \emph{balance}
). The default value is 0. 
\subsection*{Performance}


 \emph{svfilter}
 is a second order state-variable filter, with k-rate controls for cutoff frequency and Q. As Q is increased, a resonant peak forms around the cutoff frequency. \emph{svfilter}
 has simultaneous lowpass, highpass, and bandpass filter outputs; by mixing the outputs together, a variety of frequency responses can be generated. The state-variable filter, or ``multimode'' filter was a common feature in early analog synthesizers, due to the wide variety of sounds available from the interaction between cutoff, resonance, and output mix ratios. \emph{svfilter}
 is well suited to the emulation of ``analog'' sounds, as well as other applications where resonant filters are called for. 


 \emph{asig}
 -- Input signal to be filtered. 


 \emph{kcf}
 -- Cutoff or resonant frequency of the filter, measured in Hz. 


 \emph{kq}
 -- Q of the filter, which is defined (for bandpass filters) as bandwidth/cutoff. \emph{kq}
 should be in a range between 1 and 500. As \emph{kq}
 is increased, the resonance of the filter increases, which corresponds to an increase in the magnitude and ``sharpness'' of the resonant peak. When using \emph{svfilter}
 without any scaling of the signal (where \emph{iscl}
 is either absent or 0), the volume of the resonant peak increases as Q increases. For high values of Q, it is recommended that \emph{iscl}
 be set to a non-zero value, or that an external scaling function such as \emph{balance}
 is used. 


 \emph{svfilter}
 is based upon an algorithm in Hal Chamberlin's \emph{Musical Applications of Microprocessor}
s (Hayden Books, 1985). 
\subsection*{Examples}


  Here is an example of the svfilter opcode. It uses the files \emph{svfilter.orc}
 and \emph{svfilter.sco}
. 


 \textbf{Example 1. Example of the svfilter opcode.}

\begin{lstlisting}
/* svfilter.orc */
; Orchestra file for resonant filter sweep of a sawtooth-like waveform. 
; The seperate outputs of the filter are scaled by values from the score,
; and are mixed together.
sr = 44100
kr = 2205
ksmps = 20
nchnls = 1
  
instr 1
  
  idur     = p3
  ifreq    = p4
  iamp     = p5
  ilowamp  = p6              ; determines amount of lowpass output in signal
  ihighamp = p7              ; determines amount of highpass output in signal
  ibandamp = p8              ; determines amount of bandpass output in signal
  iq       = p9              ; value of q
  
  iharms   =        (sr*.4) / ifreq
  
  asig    gbuzz 1, ifreq, iharms, 1, .9, 1             ; Sawtooth-like waveform
  kfreq   linseg 1, idur * 0.5, 4000, idur * 0.5, 1     ; Envelope to control filter cutoff
  
  alow, ahigh, aband   svfilter asig, kfreq, iq
  
  aout1   =         alow * ilowamp
  aout2   =         ahigh * ihighamp
  aout3   =         aband * ibandamp
  asum    =         aout1 + aout2 + aout3
  kenv    linseg 0, .1, iamp, idur -.2, iamp, .1, 0     ; Simple amplitude envelope
          out asum * kenv
  
endin
/* svfilter.orc */
        
\end{lstlisting}
\begin{lstlisting}
/* svfilter.sco */
f1 0 8192 9 1 1 .25
  
i1  0 5 100 1000 1 0 0  5  ; lowpass sweep
i1  5 5 200 1000 1 0 0 30  ; lowpass sweep, octave higher, higher q
i1 10 5 100 1000 0 1 0  5  ; highpass sweep
i1 15 5 200 1000 0 1 0 30  ; highpass sweep, octave higher, higher q
i1 20 5 100 1000 0 0 1  5  ; bandpass sweep
i1 25 5 200 1000 0 0 1 30  ; bandpass sweep, octave higher, higher q
i1 30 5 200 2000 .4 .6  0  ; notch sweep - notch formed by combining highpass and lowpass outputs
e
/* svfilter.sco */
        
\end{lstlisting}
\subsection*{Credits}


 


 


\begin{tabular}{ccc}
Author: Sean Costello &Seattle, Washington &1999

\end{tabular}



 


 New in Csound version 3.55
%\hline 


\begin{comment}
\begin{tabular}{lcr}
Previous &Home &Next \\
sum &Up &table

\end{tabular}


\end{document}
\end{comment}
