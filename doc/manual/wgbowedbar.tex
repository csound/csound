\begin{comment}
\documentclass[10pt]{article}
\usepackage{fullpage, graphicx, url}
\setlength{\parskip}{1ex}
\setlength{\parindent}{0ex}
\title{wgbowedbar}
\begin{document}


\begin{tabular}{ccc}
The Alternative Csound Reference Manual & & \\
Previous & &Next

\end{tabular}

%\hline 
\end{comment}
\section{wgbowedbar}
wgbowedbar�--� A physical model of a bowed bar. \subsection*{Description}


  A physical model of a bowed bar, belonging to the Perry Cook family of waveguide instruments. 
\subsection*{Syntax}


 ar \textbf{wgbowedbar}
 kamp, kfreq, kpos, kbowpres, kgain [, iconst] [, itvel] [, ibowpos] [, ilow]
\subsection*{Initialization}


 \emph{iconst}
 (optional, default=0) -- an integration constant. Default is zero. 


 \emph{itvel}
 (optional, default=0) -- either 0 or 1. When \emph{ktvel}
 = 0, the bow velocity follows an ADSR style trajectory. When \emph{ktvel}
 = 1, the value of the bow velocity decays in an exponentially. 


 \emph{ibowpos}
 (optional, default=0) -- the position on the bow, which affects the bow velocity trajectory. 


 \emph{ilow}
 (optional, default=0) -- lowest frequency required 
\subsection*{Performance}


 \emph{kamp}
 -- amplitude of signal 


 \emph{kfreq}
 -- frequency of signal 


 \emph{kpos}
 -- position of the bow on the bar, in the range 0 to 1 


 \emph{kbowpres}
 -- pressure of the bow (as in \emph{wgbowed}
) 


 \emph{kgain}
 -- gain of filter. A value of about 0.809 is suggested. 
\subsection*{Examples}


  Here is an example of the wgbowedbar opcode. It uses the files \emph{wgbowedbar.orc}
 and \emph{wgbowedbar.sco}
. 


 \textbf{Example 1. Example of the wgbowedbar opcode.}

\begin{lstlisting}
/* wgbowedbar.orc */
; Initialize the global variables.
sr = 44100
kr = 4410
ksmps = 10
nchnls = 1

  instr 1
; pos      =         [0, 1]
; bowpress =         [1, 10]
; gain     =         [0.8, 1]
; intr     =         [0,1]
; trackvel =         [0, 1]
; bowpos   =         [0, 1]

  kb      line 0.5, p3, 0.1
  kp      line 0.6, p3, 0.7
  kc      line 1, p3, 1

  a1      wgbowedbar p4, cpspch(p5), kb, kp, 0.995, p6, 0

          out a1
          endin
/* wgbowedbar.orc */
        
\end{lstlisting}
\begin{lstlisting}
/* wgbowedbar.sco */
  i1      0  3 32000 7.00 0
  e
/* wgbowedbar.sco */
        
\end{lstlisting}
\subsection*{Credits}


 


 


\begin{tabular}{ccc}
Author: John ffitch (after Perry Cook) &University of Bath, Codemist Ltd. &Bath, UK

\end{tabular}



 


 New in Csound version 4.07
%\hline 


\begin{comment}
\begin{tabular}{lcr}
Previous &Home &Next \\
wgbow &Up &wgbrass

\end{tabular}


\end{document}
\end{comment}
