\begin{comment}
\documentclass[10pt]{article}
\usepackage{fullpage, graphicx, url}
\setlength{\parskip}{1ex}
\setlength{\parindent}{0ex}
\title{Expressions}
\begin{document}


\begin{tabular}{ccc}
The Alternative Csound Reference Manual & & \\
Previous &Syntax of the Orchestra &Next

\end{tabular}

%\hline 
\end{comment}
\section{Expressions}


  Expressions may be composed to any depth. Each part of an expression is evaluated at its own proper rate. For instance, if the terms within a sub-expression all change at the control rate or slower, the sub-expression will be evaluated only at the control rate; that result might then be used in an audio-rate evaluation. For example, in 


 
\begin{lstlisting}
k1 + \emph{abs}
(\emph{int}
(p5) + \emph{frac}
(p5) * 100/12 + \emph{sqrt}
(k1))
      
\end{lstlisting}


 


  the 100/12 would be evaluated at orch init, the p5 expressions evaluated at note i-time, and the remainder of the expression evaluated every k-period. The whole might occur in a unit generator argument position, or be part of an assignment statement. 
%\hline 


\begin{comment}
\begin{tabular}{lcr}
Previous &Home &Next \\
Constants and Variables &Up &Orchestra Header Statements

\end{tabular}


\end{document}
\end{comment}
