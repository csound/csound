\begin{comment}
\documentclass[10pt]{article}
\usepackage{fullpage, graphicx, url}
\setlength{\parskip}{1ex}
\setlength{\parindent}{0ex}
\title{fout}
\begin{document}


\begin{tabular}{ccc}
The Alternative Csound Reference Manual & & \\
Previous & &Next

\end{tabular}

%\hline 
\end{comment}
\section{fout}
fout�--� Outputs a-rate signals to an arbitrary number of channels. \subsection*{Description}


 \emph{fout}
 outputs \emph{N}
 a-rate signals to a specified file of \emph{N}
 channels. 
\subsection*{Syntax}


 \textbf{fout}
 ifilename, iformat, aout1 [, aout2, aout3,...,aoutN]
\subsection*{Initialization}


 \emph{ifilename}
 -- the output file's name (in double-quotes). 


 \emph{iformat}
 -- a flag to choose output file format: 


 
\begin{itemize}
\item 

 0 - 32-bit floating point samples without header (binary PCM multichannel file)

\item 

 1 - 16-bit integers without header (binary PCM multichannel file)

\item 

 2 - 16-bit integers with a header. The header type depends on the render format. The default header type is the IRCAM format. If the user chooses the AIFF format (using the \emph{-A flag}
), the header format will be a AIFF type. If the user chooses the WAV format (using the \emph{-W flag}
), the header format will be a WAV type.


\end{itemize}
\subsection*{Performance}


 \emph{aout1,... aoutN}
 -- signals to be written to the file 


 \emph{fout}
 (file output) writes samples of audio signals to a file with any number of channels. Channel number depends by the number of \emph{aoutN}
 variables (i.e. a mono signal with only an a-rate argument, a stereo signal with two a-rate arguments etc.) Maximum number of channels is fixed to 64. Multiple \emph{fout}
 opcodes can be present in the same instrument, referring to different files. 


  Notice that, unlike \emph{out}
, \emph{outs}
 and \emph{outq}
, \emph{fout}
 does not zero the audio variable so you must zero it after calling it. If polyphony is to be used, you can use \emph{vincr}
 and \emph{clear}
 opcodes for this task. 


  Notice that \emph{fout}
 and \emph{foutk}
 can use either a string containing a file pathname, or a handle-number generated by \emph{fiopen}
. Whereas, with \emph{fouti}
 and \emph{foutir}
, the target file can be only specified by means of a handle-number. 
\subsection*{Examples}


  Here is a simple example of the fout opcode. It uses the files \emph{fout.orc}
 and \emph{fout.sco}
. 


 \textbf{Example 1. Example of the fout opcode.}

\begin{lstlisting}
/* fout.orc */
; Initialize the global variables.
sr = 44100
kr = 4410
ksmps = 10
nchnls = 1

; Instrument #1.
instr 1
  iamp = 10000
  icps = 440
  iphs = 0

  ; Create an audio signal.
  asig oscils iamp, icps, iphs

  ; Write the audio signal to a headerless audio file 
  ; called "fout.raw".
  fout "fout.raw", 1, asig
endin
/* fout.orc */
        
\end{lstlisting}
\begin{lstlisting}
/* fout.sco */
; Play Instrument #1 for 2 seconds.
i 1 0 2
e
/* fout.sco */
        
\end{lstlisting}


  Here is an example of the fout opcode with a polyphonic score. It uses the files \emph{fout\_poly.orc}
, \emph{fout\_poly.sco}
 and \emph{beats.wav}
. 


 \textbf{Example 2. Example of the fout opcode with a polyphonic score.}

\begin{lstlisting}
/* fout_poly.orc */
; Initialize the global variables.
sr = 44100
kr = 44100
ksmps = 1
nchnls = 1

; Initialize the global audio signal.
gaudio init 0

; Instrument #1 - Play an audio file.
instr 1
  ; Generate an audio signal using 
  ; the audio file "beats.wav".
  asig soundin "beats.wav"

  ; Add this audio signal to the global one.
  vincr gaudio, asig
endin

; Instrument #2 - Create a basic tone.
instr 2
  iamp = 5000
  icps = 440
  iphs = 0

  ; Create an audio signal.
  asig oscils iamp, icps, iphs

  ; Add this audio signal to the global one.
  vincr gaudio, asig
endin

; Instrument #99 - Save the global signal to a file.
instr 99
  ; Write the global audio signal to a headerless 
  ; audio file called "fout_poly.raw".
  fout "fout_poly.raw", 1, gaudio

  ; Clear the global audio signal, preparing it 
  ; for the next round.
  clear gaudio
endin
/* fout_poly.orc */
        
\end{lstlisting}
\begin{lstlisting}
/* fout_poly.sco */
; Play Instrument #1 for two seconds.
i 1 0 2

; Play Instrument #2 every quarter-second.
i 2 0.00 0.1
i 2 0.25 0.1
i 2 0.50 0.1
i 2 0.75 0.1
i 2 1.00 0.1
i 2 1.25 0.1
i 2 1.50 0.1
i 2 1.75 0.1

; Make sure the global instrument, #99, is running
; during the entire performance (2 seconds).
i 99 0 2
e
/* fout_poly.sco */
        
\end{lstlisting}
\subsection*{See Also}


 \emph{fiopen}
, \emph{fouti}
, \emph{foutir}
, \emph{foutk}

\subsection*{Credits}


 


 


\begin{tabular}{ccc}
Author: Gabriel Maldonado &Italy &1999

\end{tabular}



 


 The simple example was written by Kevin Conder.


 New in Csound version 3.56


 October 2002. Added a note from Richard Dobson.
%\hline 


\begin{comment}
\begin{tabular}{lcr}
Previous &Home &Next \\
foscili &Up &fouti

\end{tabular}


\end{document}
\end{comment}
