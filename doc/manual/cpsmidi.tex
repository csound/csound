\begin{comment}
\documentclass[10pt]{article}
\usepackage{fullpage, graphicx, url}
\setlength{\parskip}{1ex}
\setlength{\parindent}{0ex}
\title{cpsmidi}
\begin{document}


\begin{tabular}{ccc}
The Alternative Csound Reference Manual & & \\
Previous & &Next

\end{tabular}

%\hline 
\end{comment}
\section{cpsmidi}
cpsmidi�--� Get the note number of the current MIDI event, expressed in cycles-per-second. \subsection*{Description}


  Get the note number of the current MIDI event, expressed in cycles-per-second. 
\subsection*{Syntax}


 icps \textbf{cpsmidi}

\subsection*{Performance}


  Get the note number of the current MIDI event, expressed in cycles-per-second units, for local processing. 
\subsection*{Examples}


  Here is an example of the cpsmidi opcode. It uses the files \emph{cpsmidi.orc}
 and \emph{cpsmidi.sco}
. 


 \textbf{Example 1. Example of the cpsmidi opcode.}

\begin{lstlisting}
/* cpsmidi.orc */
; Initialize the global variables.
sr = 44100
kr = 4410
ksmps = 10
nchnls = 1

; Instrument #1.
instr 1
  i1 cpsmidi

  print i1
endin
/* cpsmidi.orc */
        
\end{lstlisting}
\begin{lstlisting}
/* cpsmidi.sco */
; Play Instrument #1 for 12 seconds.
i 1 0 12
e
/* cpsmidi.sco */
        
\end{lstlisting}
\subsection*{See Also}


 \emph{aftouch}
, \emph{ampmidi}
, \emph{cpsmidib}
, \emph{cpstmid}
, \emph{midictrl}
, \emph{notnum}
, \emph{octmidi}
, \emph{octmidib}
, \emph{pchbend}
, \emph{pchmidi}
, \emph{pchmidib}
, \emph{veloc}

\subsection*{Credits}


 


 


\begin{tabular}{ccc}
Author: Barry L. Vercoe - Mike Berry &MIT - Mills &May 1997

\end{tabular}



 


 Example written by Kevin Conder.
%\hline 


\begin{comment}
\begin{tabular}{lcr}
Previous &Home &Next \\
cps2pch &Up &cpsmidib

\end{tabular}


\end{document}
\end{comment}
