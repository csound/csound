\begin{comment}
\documentclass[10pt]{article}
\usepackage{fullpage, graphicx, url}
\setlength{\parskip}{1ex}
\setlength{\parindent}{0ex}
\title{tableng}
\begin{document}


\begin{tabular}{ccc}
The Alternative Csound Reference Manual & & \\
Previous & &Next

\end{tabular}

%\hline 
\end{comment}
\section{tableng}
tableng�--� Interrogates a function table for length. \subsection*{Description}


  Interrogates a function table for length. 
\subsection*{Syntax}


 ir \textbf{tableng}
 ifn


 kr \textbf{tableng}
 kfn
\subsection*{Initialization}


 \emph{ifn}
 -- Table number to be interrogated 
\subsection*{Performance}


 \emph{kfn}
 -- Table number to be interrogated 


 \emph{tableng}
 returns the length of the specified table. This will be a power of two number in most circumstances. It will not show whether a table has a guardpoint or not. It seems this information is not available in the table's data structure. If the specified table is not found, then 0 will be returned. 


  Likely to be useful for setting up code for table manipulation operations, such as \emph{tablemix}
 and \emph{tablecopy}
. 
\subsection*{Examples}


  Here is an example of the tableng opcode. It uses the files \emph{tableng.orc}
 and \emph{tableng.sco}
. 


 \textbf{Example 1. Example of the tableng opcode.}

\begin{lstlisting}
/* tableng.orc */
; Initialize the global variables.
sr = 44100
kr = 4410
ksmps = 10
nchnls = 1

; Instrument #1.
instr 1
  ; Let's look at Table #1.
  ifn = 1
  ilen tableng ifn

  print ilen
endin
/* tableng.orc */
        
\end{lstlisting}
\begin{lstlisting}
/* tableng.sco */
; Table #1, a sine wave.
f 1 0 16384 10 1

; Play Instrument #1 for one second.
i 1 0 1
e
/* tableng.sco */
        
\end{lstlisting}
 The table is 16,384 samples long. So its output should include a line like this: \begin{lstlisting}
instr 1:  ilen = 16384.000
      
\end{lstlisting}
\subsection*{Credits}


 


 


\begin{tabular}{ccc}
Author: Robin Whittle &Australia &May 1997

\end{tabular}



 


 Example written by Kevin Conder.
%\hline 


\begin{comment}
\begin{tabular}{lcr}
Previous &Home &Next \\
tablemix &Up &tablera

\end{tabular}


\end{document}
\end{comment}
