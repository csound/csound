\begin{comment}
\documentclass[10pt]{article}
\usepackage{fullpage, graphicx, url}
\setlength{\parskip}{1ex}
\setlength{\parindent}{0ex}
\title{octave}
\begin{document}


\begin{tabular}{ccc}
The Alternative Csound Reference Manual & & \\
Previous & &Next

\end{tabular}

%\hline 
\end{comment}
\section{octave}
octave�--� Calculates a factor to raise/lower a frequency by a given amount of octaves. \subsection*{Description}


  Calculates a factor to raise/lower a frequency by a given amount of octaves. 
\subsection*{Syntax}


 \textbf{octave}
(x)


  This function works at a-rate, i-rate, and k-rate. 
\subsection*{Initialization}


 \emph{x}
 -- a value expressed in octaves. 
\subsection*{Performance}


  The value returned by the \emph{octave}
 function is a factor. You can multiply a frequency by this factor to raise/lower it by the given amount of octaves. 
\subsection*{Examples}


  Here is an example of the octave opcode. It uses the files \emph{octave.orc}
 and \emph{octave.sco}
. 


 \textbf{Example 1. Example of the octave opcode.}

\begin{lstlisting}
/* octave.orc */
; Initialize the global variables.
sr = 44100
kr = 4410
ksmps = 10
nchnls = 1

; Instrument #1.
instr 1
  ; The root note is A above middle-C (440 Hz)
  iroot = 440

  ; Raise the root note by two octaves.
  ioctaves = 2

  ; Calculate the new note.
  ifactor = octave(ioctaves)
  inew = iroot * ifactor

  ; Print out of all of the values.
  print iroot
  print ifactor
  print inew
endin
/* octave.orc */
        
\end{lstlisting}
\begin{lstlisting}
/* octave.sco */
; Play Instrument #1 for one second.
i 1 0 1
e
/* octave.sco */
        
\end{lstlisting}
 Its output should include lines like: \begin{lstlisting}
instr 1:  iroot = 440.000
instr 1:  ifactor = 4.000
instr 1:  inew = 1760.149
      
\end{lstlisting}
\subsection*{See Also}


 \emph{cent}
, \emph{db}
, \emph{semitone}

\subsection*{Credits}


 Example written by Kevin Conder.


 New in version 4.16
%\hline 


\begin{comment}
\begin{tabular}{lcr}
Previous &Home &Next \\
ntrpol &Up &octcps

\end{tabular}


\end{document}
\end{comment}
