\begin{comment}
\documentclass[10pt]{article}
\usepackage{fullpage, graphicx, url}
\setlength{\parskip}{1ex}
\setlength{\parindent}{0ex}
\title{lpinterp}
\begin{document}


\begin{tabular}{ccc}
The Alternative Csound Reference Manual & & \\
Previous & &Next

\end{tabular}

%\hline 
\end{comment}
\section{lpinterp}
lpslot, lpinterp�--� Computes a new set of poles from the interpolation between two analysis. \subsection*{Description}


  Computes a new set of poles from the interpolation between two analysis. 
\subsection*{Syntax}


 \textbf{lpinterp}
 islot1, islot2, kmix
\subsection*{Initialization}


 \emph{islot1}
 -- slot where the first analysis was stored 


 \emph{islot2}
 -- slot where the second analysis was stored 


 \emph{kmix}
 -- mix value between the two analysis. Should be between 0 and 1. 0 means analysis 1 only. 1 means analysis 2 only. Any value in between will produce interpolation between the filters. 


 \emph{lpinterp}
 computes a new set of poles from the interpolation between two analysis. The poles will be stored in the current \emph{lpslot}
 and used by the next \emph{lpreson}
 opcode. 
\subsection*{Examples}


  Here is a typical orc using the opcodes: 


 
\begin{lstlisting}
ipower \emph{init}
 50000  ; Define sound generator
ifreq  \emph{init}
 440 
asrc   \emph{buzz}
 ipower,ifreq,10,1
  
ktime  \emph{line}
 0,p3,p3          ; Define time lin
       \emph{lpslot}
 0              ; Read square data poles
krmsr,krmso,kerr,kcps \emph{lpread}
    ktime,"square.pol"                     
       \emph{lpslot}
 1              ; Read triangle data poles
krmsr,krmso,kerr,kcps \emph{lpread}
    ktime,"triangle.pol"
kmix   \emph{line}
 0,p3,1           ; Compute result of mixing
       \emph{lpinterp}
 0,1,kmix     ; and balance power
ares   \emph{lpreson}
 asrc
aout   \emph{balance}
 ares,asrc
       \emph{out}
 aout
        
\end{lstlisting}


 
\subsection*{See Also}


 \emph{lpslot}

\subsection*{Credits}


 Author: Gabriel Maldonado
%\hline 


\begin{comment}
\begin{tabular}{lcr}
Previous &Home &Next \\
lphasor &Up &lposcil

\end{tabular}


\end{document}
\end{comment}
