\begin{comment}
\documentclass[10pt]{article}
\usepackage{fullpage, graphicx, url}
\setlength{\parskip}{1ex}
\setlength{\parindent}{0ex}
\title{oscil3}
\begin{document}


\begin{tabular}{ccc}
The Alternative Csound Reference Manual & & \\
Previous & &Next

\end{tabular}

%\hline 
\end{comment}
\section{oscil3}
oscil3�--� A simple oscillator with cubic interpolation. \subsection*{Description}


  Table \emph{ifn}
 is incrementally sampled modulo the table length and the value obtained is multiplied by \emph{amp}
. 
\subsection*{Syntax}


 ar \textbf{oscil3}
 xamp, xcps, ifn [, iphs]


 kr \textbf{oscil3}
 kamp, kcps, ifn [, iphs]
\subsection*{Initialization}


 \emph{ifn}
 -- function table number. Requires a wrap-around guard point. 


 \emph{iphs}
 (optional) -- initial phase of sampling, expressed as a fraction of a cycle (0 to 1). A negative value will cause phase initialization to be skipped. The default value is 0. 
\subsection*{Performance}


 \emph{kamp, xamp}
 -- amplitude 


 \emph{kcps, xcps}
 -- frequency in cycles per second. 


 \emph{oscil3}
 is experimental, and is identical to \emph{oscili}
, except that it uses cubic interpolation. (New in Csound version 3.50.) 
\subsection*{Examples}


  Here is an example of the oscil3 opcode. It uses the files \emph{oscil3.orc}
 and \emph{oscil3.sco}
. 


 \textbf{Example 1. Example of the oscil3 opcode.}

\begin{lstlisting}
/* oscil3.orc */
; Initialize the global variables.
sr = 44100
kr = 4410
ksmps = 10
nchnls = 1

; Instrument #1 - a basic oscillator.
instr 1
  kamp = 10000
  kcps = 220
  ifn = 1

  a1 oscil kamp, kcps, ifn
  out a1
endin

; Instrument #2 - the basic oscillator with cubic interpolation.
instr 2
  kamp = 10000
  kcps = 220
  ifn = 1

  a1 oscil3 kamp, kcps, ifn
  out a1
endin
/* oscil3.orc */
        
\end{lstlisting}
\begin{lstlisting}
/* oscil3.sco */
; Table #1, a sine wave table with a small amount of data.
f 1 0 32 10 0 1

; Play Instrument #1, the basic oscillator, for 
; two seconds. This should sound relatively rough.
i 1 0 2

; Play Instrument #2, the cubic interpolated oscillator, for
; two seconds. This should sound relatively smooth.
i 2 2 2
e
/* oscil3.sco */
        
\end{lstlisting}
\subsection*{See Also}


 \emph{oscil}
, \emph{oscili}

\subsection*{Credits}


 Example written by Kevin Conder.
%\hline 


\begin{comment}
\begin{tabular}{lcr}
Previous &Home &Next \\
oscil1i &Up &oscili

\end{tabular}


\end{document}
\end{comment}
