\begin{comment}
\documentclass[10pt]{article}
\usepackage{fullpage, graphicx, url}
\setlength{\parskip}{1ex}
\setlength{\parindent}{0ex}
\title{portk}
\begin{document}


\begin{tabular}{ccc}
The Alternative Csound Reference Manual & & \\
Previous & &Next

\end{tabular}

%\hline 
\end{comment}
\section{portk}
portk�--� Applies portamento to a step-valued control signal. \subsection*{Description}


  Applies portamento to a step-valued control signal. 
\subsection*{Syntax}


 kr \textbf{portk}
 ksig, khtim [, isig]
\subsection*{Initialization}


 \emph{isig}
 (optional, default=0) -- initial (i.e. previous) value for internal feedback. The default value is 0. 
\subsection*{Performance}


 \emph{kr}
 -- the output signal at control-rate. 


 \emph{ksig}
 -- the input signal at control-rate. 


 \emph{khtim}
 -- half-time of the function in seconds. 


 \emph{portk}
 is like \emph{port}
 except the half-time can be varied at the control rate. 
\subsection*{See Also}


 \emph{areson}
, \emph{aresonk}
, \emph{atone}
, \emph{atonek}
, \emph{port}
, \emph{reson}
, \emph{resonk}
, \emph{tone}
, \emph{tonek}

%\hline 


\begin{comment}
\begin{tabular}{lcr}
Previous &Home &Next \\
port &Up &poscil

\end{tabular}


\end{document}
\end{comment}
