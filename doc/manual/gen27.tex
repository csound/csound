\begin{comment}
\documentclass[10pt]{article}
\usepackage{fullpage, graphicx, url}
\setlength{\parskip}{1ex}
\setlength{\parindent}{0ex}
\title{GEN27}
\begin{document}


\begin{tabular}{ccc}
The Alternative Csound Reference Manual & & \\
Previous & &Next

\end{tabular}

%\hline 
\end{comment}
\section{GEN27}
GEN27�--� Construct functions from segments of straight lines in breakpoint fashion. \subsection*{Description}


  Construct functions from segments of straight lines in breakpoint fashion. 
\subsection*{Syntax}


 \textbf{f}
 \# time size 27 x1 y1 x2 y2 x3 ...
\subsection*{Initialization}


 \emph{size }
 -- number of points in the table. Must be a power of 2 or power-of-2 plus 1 (see \emph{f statement}
). 


 \emph{x1, x2, x3,}
 etc. -- locations in table at which to attain the following y value. Must be in increasing order. If the last value is less than size, then the rest will be set to zero. Should not be negative but can be zero. 


 \emph{y1, y2, y3,}
, etc. -- Breakpoint values attained at the location specified by the preceding x value. 


 


\begin{tabular}{cc}
\textbf{Note}
 \\
� &

  If p4 is positive, functions are post-normalized (rescaled to a maximum absolute value of 1 after generation). A negative p4 will cause rescaling to be skipped. 


\end{tabular}

\subsection*{Examples}


 


 
\begin{lstlisting}
\emph{f}
 1 0 257 27 0 0 100 1 200 -1 256 0
        
\end{lstlisting}


 
 This describes a function which begins at 0, rises to 1 at the 100th table location, falls to -1, by the 200th location, and returns to 0 by the end of the table. The interpolation is linear. \subsection*{See Also}


 \emph{f statement}
, \emph{GEN25}

\subsection*{Credits}


 


 


\begin{tabular}{ccc}
Author: John ffitch &University of Bath/Codemist Ltd. &Bath, UK

\end{tabular}



 


 New in Csound version 3.49
%\hline 


\begin{comment}
\begin{tabular}{lcr}
Previous &Home &Next \\
GEN25 &Up &GEN28

\end{tabular}


\end{document}
\end{comment}
