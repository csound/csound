\begin{comment}
\documentclass[10pt]{article}
\usepackage{fullpage, graphicx, url}
\setlength{\parskip}{1ex}
\setlength{\parindent}{0ex}
\title{tablecopy}
\begin{document}


\begin{tabular}{ccc}
The Alternative Csound Reference Manual & & \\
Previous & &Next

\end{tabular}

%\hline 
\end{comment}
\section{tablecopy}
tablecopy�--� Simple, fast table copy opcode. \subsection*{Description}


  Simple, fast table copy opcode. 
\subsection*{Syntax}


 \textbf{tablecopy}
 kdft, ksft
\subsection*{Performance}


 \emph{kdft}
 -- Destination function table. 


 \emph{ksft}
 -- Number of source function table. 


 \emph{tablecopy}
 -- Simple, fast table copy opcode. Takes the table length from the destination table, and reads from the start of the source table. For speed reasons, does not check the source length - just copies regardless - in ``wrap'' mode. This may read through the source table several times. A source table with length 1 will cause all values in the destination table to be written to its value. 


 \emph{tablecopy}
 cannot read or write the guardpoint. To read it use \emph{table}
, with \emph{ndx}
 = the table length. Likewise use table write to write it. 


  To write the guardpoint to the value in location 0, use \emph{tablegpw}
. 


  This is primarily to change function tables quickly in a real-time situation. 
\subsection*{See Also}


 \emph{tablegpw}
, \emph{tablemix}
, \emph{tableicopy}
, \emph{tableigpw}
, \emph{tableimix}

\subsection*{Credits}


 


 


\begin{tabular}{ccc}
Author: Robin Whittle &Australia &May 1997

\end{tabular}



 
%\hline 


\begin{comment}
\begin{tabular}{lcr}
Previous &Home &Next \\
table3 &Up &tablegpw

\end{tabular}


\end{document}
\end{comment}
