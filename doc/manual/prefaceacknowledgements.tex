\begin{comment}
\documentclass[10pt]{article}
\usepackage{fullpage, graphicx, url}
\setlength{\parskip}{1ex}
\setlength{\parindent}{0ex}
\title{Acknowledgements}
\begin{document}


\begin{tabular}{ccc}
The Alternative Csound Reference Manual & & \\
Previous &Preface &Next

\end{tabular}

%\hline 
\end{comment}
\section{Acknowledgements}


  In addition to the core code developed by Barry L. Vercoe at M.I.T., a large part of the Csound code was modified, developed and extended by an independent group of programmers, composers and scientists. Copyright to this code is held by the respective authors: 


 \textbf{Table 1. Contributors}



\begin{tabular}{cccccccccccccccc}
Mike BerryRichard Karpen &Eli BrederVictor Lazzarini &Michael CaseyAllan Lee &Michael ClarkDavid Macintyre &Perry CookGabriel Maldonado &Sean CostelloMax Mathews &Richard DobsonHans Mikelson &Mark DolsonPeter Neub\"acker &Rasmus EkmanVille Pulkki &Dan EllisMarc Resibois &Tom ErbeParis Smaragdis &John ffitchRob Shaw &Bill GardnerGreg Sullivan &Matt IngallsBill Verplank &Istvan VargaRobin Whittle &Jean Pich\'ePeter Nix

\end{tabular}



  The official manual was compiled from the canonical Csound Manual sources maintained by John ffitch, Richard Boulanger, Jean Pich\'e, Peter Nix, and David M. Boothe. The Alternative Csound Reference Manual is maintained by Kevin Conder. 
%\hline 


\begin{comment}
\begin{tabular}{lcr}
Previous &Home &Next \\
Copyright Notice &Up &Why is this called the \emph{Alternative}
 Csound Reference Manual?

\end{tabular}


\end{document}
\end{comment}
