\begin{comment}
\documentclass[10pt]{article}
\usepackage{fullpage, graphicx, url}
\setlength{\parskip}{1ex}
\setlength{\parindent}{0ex}
\title{fmmetal}
\begin{document}


\begin{tabular}{ccc}
The Alternative Csound Reference Manual & & \\
Previous & &Next

\end{tabular}

%\hline 
\end{comment}
\section{fmmetal}
fmmetal�--� Uses FM synthesis to create a ``Heavy Metal'' sound. \subsection*{Description}


  Uses FM synthesis to create a ``Heavy Metal'' sound. It comes from a family of FM sounds, all using 4 basic oscillators and various architectures, as used in the TX81Z synthesizer. 
\subsection*{Syntax}


 ar \textbf{fmmetal}
 kamp, kfreq, kc1, kc2, kvdepth, kvrate, ifn1, ifn2, ifn3, ifn4, ivfn
\subsection*{Initialization}


  All these opcodes take 5 tables for initialization. The first 4 are the basic inputs and the last is the low frequency oscillator (LFO) used for vibrato. The last table should usually be a sine wave. 


  The initial waves should be: 


 
\begin{itemize}
\item 

 \emph{ifn1}
 -- sine wave

\item 

 \emph{ifn2}
 -- \emph{twopeaks.aiff}


\item 

 \emph{ifn3}
 -- \emph{twopeaks.aiff}


\item 

 \emph{ifn4}
 -- sine wave


\end{itemize}


 


\begin{tabular}{cc}
\textbf{Note}
 \\
� &

  The file ``twopeaks.aiff'' is also available at \emph{\url{ftp://ftp.cs.bath.ac.uk/pub/dream/documentation/sounds/modelling/}}
. 


\end{tabular}

\subsection*{Performance}


 \emph{kamp}
 -- Amplitude of note. 


 \emph{kfreq}
 -- Frequency of note played. 


 \emph{kc1, kc2}
 -- Controls for the synthesizer: 


 
\begin{itemize}
\item 

 \emph{kc1}
 -- Total mod index

\item 

 \emph{kc2}
 -- Crossfade of two modulators

\item 

 \emph{Algorithm}
 -- 3


\end{itemize}


 \emph{kvdepth}
 -- Vibrator depth 


 \emph{kvrate}
 -- Vibrator rate 
\subsection*{Examples}


  Here is an example of the fmmetal opcode. It uses the files \emph{fmmetal.orc}
, \emph{fmmetal.sco}
, and \emph{twopeaks.aiff}
. 


 \textbf{Example 1. Example of the fmmetal opcode.}

\begin{lstlisting}
/* fmmetal.orc */
; Initialize the global variables.
sr = 22050
kr = 2205
ksmps = 10
nchnls = 1

; Instrument #1.
instr 1
  kamp = 10000
  kfreq = 440
  kc1 = 6
  kc2 = 5
  kvdepth = 0
  kvrate = 0
  ifn1 = 1
  ifn2 = 2
  ifn3 = 2
  ifn4 = 1
  ivfn = 1

  a1 fmmetal kamp, kfreq, kc1, kc2, kvdepth, kvrate, ifn1, ifn2, ifn3, ifn4, ivfn
  out a1
endin
/* fmmetal.orc */
        
\end{lstlisting}
\begin{lstlisting}
/* fmmetal.sco */
; Table #1, a normal sine wave.
f 1 0 32768 10 1
; Table #2, the "twopeaks.aiff" audio file.
f 2 0 256 1 "twopeaks.aiff" 0 0 0 

; Play Instrument #1 for one second.
i 1 0 1
e
/* fmmetal.sco */
        
\end{lstlisting}
\subsection*{See Also}


 \emph{fmb3}
, \emph{fmbell}
, \emph{fmpercfl}
, \emph{fmrhode}
, \emph{fmwurlie}

\subsection*{Credits}


 


 


\begin{tabular}{ccc}
Author: John ffitch (after Perry Cook) &University of Bath, Codemist Ltd. &Bath, UK

\end{tabular}



 


 Example written by Kevin Conder.


 New in Csound version 3.47
%\hline 


\begin{comment}
\begin{tabular}{lcr}
Previous &Home &Next \\
fmbell &Up &fmpercfl

\end{tabular}


\end{document}
\end{comment}
