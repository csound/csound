\begin{comment}
\documentclass[10pt]{article}
\usepackage{fullpage, graphicx, url}
\setlength{\parskip}{1ex}
\setlength{\parindent}{0ex}
\title{timeinstk}
\begin{document}


\begin{tabular}{ccc}
The Alternative Csound Reference Manual & & \\
Previous & &Next

\end{tabular}

%\hline 
\end{comment}
\section{timeinstk}
timeinstk�--� Read absolute time in k-rate cycles. \subsection*{Description}


  Read absolute time, in k-rate cycles, since the start of an instance of an instrument. 
\subsection*{Syntax}


 kr \textbf{timeinstk}



 kr \textbf{timeinsts}

\subsection*{Performance}


 \emph{timeinstk}
 is for time in k-rate cycles. So with: 


 
\begin{lstlisting}
  \emph{sr}
    = 44100 
  \emph{kr}
    = 6300 
  \emph{ksmps}
 = 7
        
\end{lstlisting}


 
 then after half a second, the \emph{timek}
 opcode would report 3150. It will always report an integer. 

 \emph{timeinstk}
 produces a k-rate variable for output. There are no input parameters. 


 \emph{timeinstk}
 is similar to \emph{timek}
 except it returns the time since the start of this instance of the instrument. 
\subsection*{Examples}


  Here is an example of the timeinstk opcode. It uses the files \emph{timeinstk.orc}
 and \emph{timeinstk.sco}
. 


 \textbf{Example 1. Example of the timeinstk opcode.}

\begin{lstlisting}
/* timeinstk.orc */
; Initialize the global variables.
sr = 44100
kr = 4410
ksmps = 10
nchnls = 1

; Instrument #1.
instr 1
  ; Print out the value from timeinstk every half-second.
  k1 timeinstk
  printks "k1 = %f samples\\n", 0.5, k1
endin
/* timeinstk.orc */
        
\end{lstlisting}
\begin{lstlisting}
/* timeinstk.sco */
; Play Instrument #1 for two seconds.
i 1 0 2
e
/* timeinstk.sco */
        
\end{lstlisting}
 Its output should include lines like this: \begin{lstlisting}
k1 = 1.000000 samples
k1 = 2205.000000 samples
k1 = 4410.000000 samples
k1 = 6615.000000 samples
k1 = 8820.000000 samples
      
\end{lstlisting}
\subsection*{See Also}


 \emph{timeinsts}
, \emph{timek}
, \emph{times}

\subsection*{Credits}


 


 


\begin{tabular}{ccc}
Author: Robin Whittle &Australia &May 1997

\end{tabular}



 


 Example written by Kevin Conder.
%\hline 


\begin{comment}
\begin{tabular}{lcr}
Previous &Home &Next \\
tigoto &Up &timeinsts

\end{tabular}


\end{document}
\end{comment}
