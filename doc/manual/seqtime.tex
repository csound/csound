\begin{comment}
\documentclass[10pt]{article}
\usepackage{fullpage, graphicx, url}
\setlength{\parskip}{1ex}
\setlength{\parindent}{0ex}
\title{seqtime}
\begin{document}


\begin{tabular}{ccc}
The Alternative Csound Reference Manual & & \\
Previous & &Next

\end{tabular}

%\hline 
\end{comment}
\section{seqtime}
seqtime�--� Generates a trigger signal according to the values stored in a table. \subsection*{Description}


  Generates a trigger signal according to the values stored in a table. 
\subsection*{Syntax}


 ktrig\_out \textbf{seqtime}
 ktime\_unit, kstart, kloop, kinitndx, kfn\_times
\subsection*{Performance}


 \emph{ktrig\_out}
 -- output trigger signal 


 \emph{ktime\_unit}
 -- unit of measure of time, related to seconds. 


 \emph{kstart}
 -- start index of looped section 


 \emph{kloop}
 -- end index of looped section 


 \emph{kinitndx}
 -- initial index 


 


\begin{tabular}{cc}
\textbf{Note}
 \\
� &

  Although \emph{kinitndx}
 is listed as k-rate, it is in fact accessed only at init-time. So if you are using a k-rate argument, it must be assigned with \emph{init}
. 


\end{tabular}



 \emph{kfn\_times}
 -- number of table containing a sequence of times 


  This opcode handles timed-sequences of groups of values stored into a table. 


 \emph{seqtime}
 generates a trigger signal (a sequence of impulses, see also \emph{trigger}
 opcode), according to the values stored in the \emph{kfn\_times}
 table. This table should contain a series of delta-times (i.e. times beetween to adjacent events). The time units stored into table are expressed in seconds, but can be rescaled by means of ktime\_unit argument. The table can be filled with \emph{GEN02}
 or by means of an external text-file containing numbers, with \emph{GEN23}
. 


  It is possible to start the sequence from a value different than the first, by assigning to \emph{initndx}
 an index different than zero (which corresponds to the first value of the table). Normally the sequence is looped, and the start and end of loop can be adjusted by modifying \emph{kstart}
 and \emph{kloop}
 arguments. User must be sure that values of these arguments (as well as \emph{initndx}
) correspond to valid table numbers, otherwise Csound will crash (because no range-checking is implementeted). 


  It is possible to disable loop (one-shot mode) by assigning the same value both to \emph{kstart}
 and \emph{kloop}
 arguments. In this case, the last read element will be the one corresponding to the value of such arguments. Table can be read backward by assigning a negative \emph{kloop}
 value. It is possible to trigger two events almost at the same time (actually separated by a k-cycle) by giving a zero value to the corresponding delta-time. First element contained in the table should be zero, if the user intends to send a trigger impulse, it should come immediately after the orchestra instrument containing \emph{seqtime}
 opcode. 
\subsection*{Examples}


 


 \textbf{Example 1. Example of the seqtime opcode.}

\begin{lstlisting}
        instr   1
icps    cpsmidi
iamp    ampmidi 5000
ktrig   seqtime 1,       1,          10,      0,   1
trigseq ktrig, 0, 10, 0, 2, kdur, kampratio, kfreqratio
        schedkwhen      ktrig, -1, -1, 2, 0, kdur, kampratio*iamp, kfreqratio*icps
        endin


        instr  2
**** put here your intrument code *******
        out     a1
        endin
        
\end{lstlisting}
\subsection*{See Also}


 \emph{GEN02}
, \emph{GEN23}
, \emph{trigseq}

\subsection*{Credits}


 Author: Gabriel Maldonado


 November 2002. Added a note about the \emph{kinitndx}
 parameter, thanks to Rasmus Ekman.


 New in version 4.06
%\hline 


\begin{comment}
\begin{tabular}{lcr}
Previous &Home &Next \\
sensekey &Up &setctrl

\end{tabular}


\end{document}
\end{comment}
