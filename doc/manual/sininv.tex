\begin{comment}
\documentclass[10pt]{article}
\usepackage{fullpage, graphicx, url}
\setlength{\parskip}{1ex}
\setlength{\parindent}{0ex}
\title{sininv}
\begin{document}


\begin{tabular}{ccc}
The Alternative Csound Reference Manual & & \\
Previous & &Next

\end{tabular}

%\hline 
\end{comment}
\section{sininv}
sininv�--� Performs an arcsine function. \subsection*{Description}


  Returns the arcsine of \emph{x}
 (\emph{x}
 in radians). 
\subsection*{Syntax}


 \textbf{sininv}
(x) (no rate restriction)
\subsection*{Examples}


  Here is an example of the sininv opcode. It uses the files \emph{sininv.orc}
 and \emph{sininv.sco}
. 


 \textbf{Example 1. Example of the sininv opcode.}

\begin{lstlisting}
/* sininv.orc */
; Initialize the global variables.
sr = 44100
kr = 4410
ksmps = 10
nchnls = 1

; Instrument #1.
instr 1
  irad = 0.5
  i1 = sininv(irad)

  print i1
endin
/* sininv.orc */
        
\end{lstlisting}
\begin{lstlisting}
/* sininv.sco */
; Play Instrument #1 for one second.
i 1 0 1
e
/* sininv.sco */
        
\end{lstlisting}
 Its output should include a line like this: \begin{lstlisting}
instr 1:  i1 = 0.524
      
\end{lstlisting}
\subsection*{See Also}


 \emph{cos}
, \emph{cosh}
, \emph{cosinv}
, \emph{sin}
, \emph{sinh}
, \emph{tan}
, \emph{tanh}
, \emph{taninv}

\subsection*{Credits}


 Example written by Kevin Conder.
%\hline 


\begin{comment}
\begin{tabular}{lcr}
Previous &Home &Next \\
sinh &Up &sleighbells

\end{tabular}


\end{document}
\end{comment}
