\begin{comment}
\documentclass[10pt]{article}
\usepackage{fullpage, graphicx, url}
\setlength{\parskip}{1ex}
\setlength{\parindent}{0ex}
\title{pvsinfo}
\begin{document}


\begin{tabular}{ccc}
The Alternative Csound Reference Manual & & \\
Previous & &Next

\end{tabular}

%\hline 
\end{comment}
\section{pvsinfo}
pvsinfo�--� Get information from a PVOC-EX formatted source. \subsection*{Description}


  Get format information about fsrc, whether created by an opcode such as pvsanal, or obtained from a PVOCEX file by pvsfread. This information is available at init time, and can be used to set parameters for other pvs opcodes, and in particular for creating function tables (e.g. for pvsftw), or setting the number of oscillators for pvsadsyn. 
\subsection*{Syntax}


 ioverlap, inumbins, iwinsize, iformat \textbf{pvsinfo}
 fsrc
\subsection*{Initialization}


 \emph{ioverlap}
 -- The stream overlap size. 


 \emph{inumbins}
 -- The number of analysis bins (amplitude+frequency) in fsrc. The underlying FFT size is calculated as (inumbins -1) * 2. 


 \emph{iwinsize}
 -- The analysis window size. May be larger than the FFT size. 


 \emph{iformat}
 -- The analysis frame format. If fsrc is created by an opcode, iformat will always be 0, signifying amplitude+frequency. If fsrc is defined from a PVOC-EX file, iformat may also have the value 1 or 2 (amplitude+phase, complex). 
\subsection*{Examples}


 


 
\begin{lstlisting}
fin                pvsfread  "test.pvx"    ; import pvocex file
iovl,inb,iws,ifmt  pvsinfo   fin           ; get inumbins info
ifn                ftgen     0,0,inb,10,1  ; and create f-table
        
\end{lstlisting}


 
\subsection*{Credits}


 


 


\begin{tabular}{cc}
Author: Richard Dobson &August 2001 

\end{tabular}



 


 New in version 4.13
%\hline 


\begin{comment}
\begin{tabular}{lcr}
Previous &Home &Next \\
pvsftw &Up &pvsmaska

\end{tabular}


\end{document}
\end{comment}
