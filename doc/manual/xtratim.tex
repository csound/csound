\begin{comment}
\documentclass[10pt]{article}
\usepackage{fullpage, graphicx, url}
\setlength{\parskip}{1ex}
\setlength{\parindent}{0ex}
\title{xtratim}
\begin{document}


\begin{tabular}{ccc}
The Alternative Csound Reference Manual & & \\
Previous & &Next

\end{tabular}

%\hline 
\end{comment}
\section{xtratim}
xtratim�--� Extend the duration of real-time generated events. \subsection*{Description}


  Extend the duration of real-time generated events and handle their extra life (see also \emph{linenr}
). 
\subsection*{Syntax}


 \textbf{xtratim}
 iextradur
\subsection*{Initialization}


 \emph{iextradur}
 -- additional duration of current instrument instance 
\subsection*{Performance}


 \emph{xtratim}
 extends current MIDI-activated note duration of \emph{iextradur}
 seconds after the corresponding noteoff message has deactivated current note itself. This opcode has no output arguments. 


  This opcode is useful for implementing complex release-oriented envelopes. 
\subsection*{Examples}


 


 
\begin{lstlisting}
 \emph{instr}
 1 ;allows complex ADSR envelope with MIDI events
  inum \emph{notnum}

  icps \emph{cpsmidi}

  iamp \emph{ampmid}
i 4000
 ;
 ;------- complex envelope block ------
  \emph{xtratim}
 1 ;extra-time, i.e. release dur
  krel \emph{init}
 0
  krel \emph{release}
 ;outputs release-stage flag (0 or 1 values)
  if (krel  .5) \emph{kgoto}
 rel ;if in release-stage goto release section
 ;
 ;************ attack and sustain section ***********
  kmp1 \emph{linseg}
 0, .03, 1, .05, 1, .07, 0, .08, .5, 4, 1, 50, 1
  kmp = kmp1*iamp
   \emph{kgoto}
 done
 ;
 ;--------- release section --------
   rel:
  kmp2 \emph{linseg}
 1, .3, .2, .7, 0
  kmp = kmp1*kmp2*iamp
  done:
 ;------
  a1 \emph{oscili}
 kmp, icps, 1
  \emph{out}
 a1
 \emph{endin}

        
\end{lstlisting}


 
\subsection*{See Also}


 \emph{linenr}
, \emph{release}

\subsection*{Credits}


 


 


\begin{tabular}{cc}
Author: Gabriel Maldonado &Italy

\end{tabular}



 


 New in Csound version 3.47
%\hline 


\begin{comment}
\begin{tabular}{lcr}
Previous &Home &Next \\
xscanu &Up &xyin

\end{tabular}


\end{document}
\end{comment}
