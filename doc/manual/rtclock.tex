\begin{comment}
\documentclass[10pt]{article}
\usepackage{fullpage, graphicx, url}
\setlength{\parskip}{1ex}
\setlength{\parindent}{0ex}
\title{rtclock}
\begin{document}


\begin{tabular}{ccc}
The Alternative Csound Reference Manual & & \\
Previous & &Next

\end{tabular}

%\hline 
\end{comment}
\section{rtclock}
rtclock�--� Read the real time clock from the operating system. \subsection*{Description}


  Read the real-time clock from the operating system. 
\subsection*{Syntax}


 ir \textbf{rtclock}



 kr \textbf{rtclock}

\subsection*{Performance}


  Read the real-time clock from operating system. Under Windows, this changes only once per second. Under GNU/Linux, it ticks every microsecond. Performance under other systems varies. 
\subsection*{Examples}


  Here is an example of the rtclock opcode. It uses the files \emph{rtclock.orc}
 and \emph{rtclock.sco}
. 


 \textbf{Example 1. Example of the rtclock opcode.}

\begin{lstlisting}
/* rtclock.orc */
; Initialize the global variables.
sr = 44100
kr = 44100
ksmps = 1
nchnls = 1

; Instrument #1
instr 1
  ; Get the system time.
  k1 rtclock
  ; Print it once per second.
  printk 1, k1
endin
/* rtclock.orc */
        
\end{lstlisting}
\begin{lstlisting}
/* rtclock.sco */
; Play Instrument #1 for two seconds.
i 1 0 2
e
/* rtclock.sco */
        
\end{lstlisting}
 Its output should include lines like this: \begin{lstlisting}
 i   1 time     0.00002: 1018236096.00000
 i   1 time     1.00002: 1018236224.00000
      
\end{lstlisting}
\subsection*{Credits}


 Author: John ffitch


 Example written by Kevin Conder.


 New in version 4.10
%\hline 


\begin{comment}
\begin{tabular}{lcr}
Previous &Home &Next \\
rspline &Up &s16b14

\end{tabular}


\end{document}
\end{comment}
