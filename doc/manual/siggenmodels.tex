\begin{comment}
\documentclass[10pt]{article}
\usepackage{fullpage, graphicx, url}
\setlength{\parskip}{1ex}
\setlength{\parindent}{0ex}
\title{Models and Emulations}
\begin{document}


\begin{tabular}{ccc}
The Alternative Csound Reference Manual & & \\
Previous &Signal Generators &Next

\end{tabular}

%\hline 
\end{comment}
\section{Models and Emulations}


  The opcodes that model or emulate the sounds of other instruments are \emph{bamboo}
, \emph{cabasa}
, \emph{crunch}
, \emph{dripwater}
, \emph{gogobel}
, \emph{guiro}
, \emph{lorenz}
, \emph{mandol}
, \emph{marimba}
, \emph{moog}
, \emph{planet}
, \emph{sandpaper}
, \emph{sekere}
, \emph{shaker}
, \emph{sleighbells}
, \emph{stix}
, \emph{tambourine}
, \emph{vibes}
, and \emph{voice}
. 
%\hline 


\begin{comment}
\begin{tabular}{lcr}
Previous &Home &Next \\
Linear Predictive Coding (LPC) Resynthesis &Up &Phasors

\end{tabular}


\end{document}
\end{comment}
