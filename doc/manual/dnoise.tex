\begin{comment}
\documentclass[10pt]{article}
\usepackage{fullpage, graphicx, url}
\setlength{\parskip}{1ex}
\setlength{\parindent}{0ex}
\title{dnoise}
\begin{document}


\begin{tabular}{ccc}
The Alternative Csound Reference Manual & & \\
Previous & &Next

\end{tabular}

%\hline 
\end{comment}
\section{dnoise}
dnoise�--� Reduces noise in a file. \subsection*{Description}


  This is a noise reduction scheme using frequency-domain noise-gating. 
\subsection*{Syntax}


 \textbf{dnoise}
 [flags] -i noise\_ref\_file -o output\_soundfile input\_soundfile
\subsection*{Initialization}


  Dnoise specific flags: 


 
\begin{itemize}
\item 

 \emph{(no flag)}
 input soundfile to be denoised

\item 

 \emph{-i fname}
 input reference noise soundfile

\item 

 \emph{-o fname}
 output soundfile

\item 

 \emph{-N fnum}
 \# of bandpass filters (default: 1024)

\item 

 \emph{-w fovlp}
 filter overlap factor: \{0,1,(2),3\} DON'T USE \emph{-w}
 AND \emph{-M}


\item 

 \emph{-M awlen}
 analysis window length (default: N-1 unless \emph{-w}
 is specified)

\item 

 \emph{-L swlen}
 synthesis window length (default: M)

\item 

 \emph{-D dfac}
 decimation factor (default: M/8)

\item 

 \emph{-b btim}
 begin time in noise reference soundfile (default: 0)

\item 

 \emph{-B smpst}
 starting sample in noise reference soundfile (default: 0)

\item 

 \emph{-e etim}
 end time in noise reference soundfile (default: end of file)

\item 

 \emph{-E smpend}
 final sample in noise reference soundfile (default: end of file)

\item 

 \emph{-t thr}
 threshold above noise reference in dB (default: 30)

\item 

 \emph{-S gfact}
 sharpness of noise-gate turnoff, range: 1 to 5 (default: 1)

\item 

 \emph{-n numfrm}
 number of FFT frames to average over (default: 5)

\item 

 \emph{-m mingain}
 minimum gain of noise-gate when off in dB (default: -40)


\end{itemize}


  Soundfile format options: 


 
\begin{itemize}
\item 

 \emph{-A}
 AIFF format output

\item 

 \emph{-W}
 WAV format output

\item 

 \emph{-J}
 IRCAM format output

\item 

 \emph{-h}
 skip soundfile header (not valid for AIFF/WAV output)

\item 

 \emph{-8}
 8-bit unsigned char sound samples

\item 

 \emph{-c}
 8-bit signed\_char sound samples

\item 

 \emph{-a}
 alaw sound samples

\item 

 \emph{-u}
 ulaw sound samples

\item 

 \emph{-s}
 short\_int sound samples

\item 

 \emph{-l}
 long\_int sound samples

\item 

 \emph{-f}
 float sound samples. Floats also supported for WAV files. (New in Csound 3.47.)


\end{itemize}


  Additional options: 


 
\begin{itemize}
\item 

 \emph{-R}
 verbose - print status info

\item 

 \emph{-H [N]}
 print a heartbeat character at each soundfile write.

\item 

 \emph{-- fname}
 output to log file fname

\item 

 \emph{-V}
 verbose - print status info


\end{itemize}


 


\begin{tabular}{cc}
\textbf{Note}
 \\
� &

  DNOISE also looks at the environment variable SFOUTYP to determine soundfile output format. 


  The -i flag is used for a reference noise file (normally created from a short section of the denoised file, where only noise is audible). The input soundfile to be denoised can be given anywhere on the command line, without a flag. 


\end{tabular}

\subsection*{Performance}


  This is a noise reduction scheme using frequency-domain noise-gating. This should work best in the case of high signal-to-noise with hiss-type noise. 


  The algorithm is that suggested by Moorer \& Berger in ``Linear-Phase Bandsplitting: Theory and Applications'' presented at the 76th Convention 1984 October 8-11 New York of the Audio Engineering Society (preprint \#2132) except that it uses the Weighted Overlap-Add formulation for short-time Fourier analysis-synthesis in place of the recursive formulation suggested by Moorer \& Berger. The gain in each frequency bin is computed independently according to 


 gain�=�g0�+�(1-g0)�*�[avg�/�(avg�+�th*th*nref)]�\^{}�sh\\ 
 ������
 where \emph{avg}
 and \emph{nref}
 are the mean squared signal and noise respectively for the bin in question. (This is slightly different than in Moorer \& Berger.) 

  The critical parameters \emph{th}
 and \emph{g0}
 are specified in dB and internally converted to decimal values. The \emph{nref}
 values are computed at the start of the program on the basis of a noise\_soundfile (specified in the command line) which contains noise without signal. 


  The \emph{avg}
 values are computed over a rectangular window of m FFT frames looking both ahead and behind the current time. This corresponds to a temporal extent of m*D/R (which is typically (m*N/8)/R). The default settings of N, M, and D should be appropriate for most uses. A higher sample rate than 16 Khz might indicate a higher N. 
\subsection*{Credits}


 Author: Mark Dolson


 August 26, 1989


 Author: John ffitch


 December 30, 2000


 Updated by Rasmus Ekman on March 11, 2002.
%\hline 


\begin{comment}
\begin{tabular}{lcr}
Previous &Home &Next \\
File Conversion &Up &pvlook

\end{tabular}


\end{document}
\end{comment}
