\begin{comment}
\documentclass[10pt]{article}
\usepackage{fullpage, graphicx, url}
\setlength{\parskip}{1ex}
\setlength{\parindent}{0ex}
\title{timeinsts}
\begin{document}


\begin{tabular}{ccc}
The Alternative Csound Reference Manual & & \\
Previous & &Next

\end{tabular}

%\hline 
\end{comment}
\section{timeinsts}
timeinsts�--� Read absolute time in seconds. \subsection*{Description}


  Read absolute time, in seconds, since the start of an instance of an instrument. 
\subsection*{Syntax}


 kr \textbf{timeinsts}

\subsection*{Performance}


  Time in seconds is available with \emph{timeinsts}
. This would return 0.5 after half a second. 


 \emph{timeinsts}
 produces a k-rate variable for output. There are no input parameters. 


 \emph{timeinsts}
 is similar to \emph{times}
 except it returns the time since the start of this instance of the instrument. 
\subsection*{Examples}


  Here is an example of the timeinsts opcode. It uses the files \emph{timeinsts.orc}
 and \emph{timeinsts.sco}
. 


 \textbf{Example 1. Example of the timeinsts opcode.}

\begin{lstlisting}
/* timeinsts.orc */
; Initialize the global variables.
sr = 44100
kr = 4410
ksmps = 10
nchnls = 1

; Instrument #1.
instr 1
  ; Print out the value from timeinsts every half-second.
  k1 timeinsts
  printks "k1 = %f seconds\\n", 0.5, k1
endin
/* timeinsts.orc */
        
\end{lstlisting}
\begin{lstlisting}
/* timeinsts.sco */
; Play Instrument #1 for two seconds.
i 1 0 2
e
/* timeinsts.sco */
        
\end{lstlisting}
 Its output should include lines like this: \begin{lstlisting}
k1 = 0.000227 seconds
k1 = 0.500000 seconds
k1 = 1.000000 seconds
k1 = 1.500000 seconds
k1 = 2.000000 seconds
      
\end{lstlisting}
\subsection*{See Also}


 \emph{timeinstk}
, \emph{timek}
, \emph{times}

\subsection*{Credits}


 


 


\begin{tabular}{ccc}
Author: Robin Whittle &Australia &May 1997

\end{tabular}



 


 Example written by Kevin Conder.
%\hline 


\begin{comment}
\begin{tabular}{lcr}
Previous &Home &Next \\
timeinstk &Up &timek

\end{tabular}


\end{document}
\end{comment}
