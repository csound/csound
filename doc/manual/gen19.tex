\begin{comment}
\documentclass[10pt]{article}
\usepackage{fullpage, graphicx, url}
\setlength{\parskip}{1ex}
\setlength{\parindent}{0ex}
\title{GEN19}
\begin{document}


\begin{tabular}{ccc}
The Alternative Csound Reference Manual & & \\
Previous & &Next

\end{tabular}

%\hline 
\end{comment}
\section{GEN19}
GEN19�--� Generate composite waveforms made up of weighted sums of simple sinusoids. \subsection*{Description}


  These subroutines generate composite waveforms made up of weighted sums of simple sinusoids. The specification of each contributing partial requires 4 p-fields using \emph{GEN19}
. 
\subsection*{Syntax}


 \textbf{f}
 \# time size 19 pna stra phsa dcoa pnb strb phsb dcob ...
\subsection*{Initialization}


 \emph{size}
 -- number of points in the table. Must be a power of 2 or power-of-2 plus 1 (see \emph{f statement}
). 


 \emph{pna, pnb}
, etc. -- partial no. (relative to a fundamental that would occupy \emph{size}
 locations per cycle) of sinusoid a, sinusoid b, etc. Must be positive, but need not be a whole number, i.e., non-harmonic partials are permitted. Partials may be in any order. 


 \emph{stra, strb}
, etc. -- strength of partials \emph{pna, pnb}
, etc. These are relative strengths, since the composite waveform may be rescaled later. Negative values are permitted and imply a 180 degree phase shift. 


 \emph{phsa, phsb}
, etc. -- initial phase of partials \emph{pna, pnb,}
 etc., expressed in degrees. 


 \emph{dcoa, dcob}
, etc. -- DC offset of partials \emph{pna, pnb}
, etc. This is applied \emph{after}
 strength scaling, i.e. a value of 2 will lift a 2-strength sinusoid from range [-2,2] to range [0,4] (before later rescaling). 


 


\begin{tabular}{cc}
\textbf{Note}
 \\
� &

 


 
\begin{itemize}
\item 

  These subroutines generate stored functions as sums of sinusoids of different frequencies. The two major restrictions on \emph{GEN10}
 that the partials be harmonic and in phase do not apply to \emph{GEN09}
 or \emph{GEN19}
. 


  In each case the composite wave, once drawn, is then rescaled to unity if p4 was positive. A negative p4 will cause rescaling to be skipped. 


\end{itemize}


\end{tabular}

\subsection*{Examples}


  Here is a simple example of the GEN19 routine. It uses the files \emph{gen19.orc}
 and \emph{gen19.sco}
. It will generate a nice bell curve, here is its diagram: 


 \includegraphics[scale=1]{gen19} 


 Diagram of the waveform generated by GEN19.


 \textbf{Example 1. A simple example of the GEN19 routine.}

\begin{lstlisting}
/* gen19.orc */
; Initialize the global variables.
sr = 44100
kr = 4410
ksmps = 10
nchnls = 1

; Instrument #1.
instr 1
  ; Create an index over the length of our entire note.
  kcps init 1/p3
  kndx phasor kcps

  ; Read Table #1 with our index.
  ifn = 1
  ixmode = 1
  kval table kndx, ifn, ixmode

  ; Generate a sine waveform, use our Table #1 value to 
  ; vary its frequency by 100 Hz from its base frequency.
  ibasefreq = 440
  kfreq = kval * 100
  a1 oscil 20000, ibasefreq + kfreq, 2
  out a1
endin
/* gen19.orc */
        
\end{lstlisting}
\begin{lstlisting}
/* gen19.sco */
; Table #1: a bell curve (using GEN19).
f 1 0 16384 -19 1 1 260 1
; Table #2, a sine wave.
f 2 0 16384 10 1

; Play Instrument #1 for 3 seconds.
i 1 0 3
e
/* gen19.sco */
        
\end{lstlisting}
\subsection*{See Also}


 \emph{GEN09}
 and \emph{GEN10}

\subsection*{Credits}


 Example written by Kevin Conder
%\hline 


\begin{comment}
\begin{tabular}{lcr}
Previous &Home &Next \\
GEN18 &Up &GEN20

\end{tabular}


\end{document}
\end{comment}
