\begin{comment}
\documentclass[10pt]{article}
\usepackage{fullpage, graphicx, url}
\setlength{\parskip}{1ex}
\setlength{\parindent}{0ex}
\title{midichn}
\begin{document}


\begin{tabular}{ccc}
The Alternative Csound Reference Manual & & \\
Previous & &Next

\end{tabular}

%\hline 
\end{comment}
\section{midichn}
midichn�--� Returns the MIDI channel number from which the note was activated. \subsection*{Description}


 \emph{midichn}
 returns the MIDI channel number (1 - 16) from which the note was activated. In the case of score notes, it returns 0. 
\subsection*{Syntax}


 ichn \textbf{midichn}

\subsection*{Initialization}


 \emph{ichn}
 -- channel number. If the current note was activated from score, it is set to zero. 
\subsection*{Examples}


  Here is a simple example of the midichn opcode. It uses the files \emph{midichn.orc}
 and \emph{midichn.sco}
. 


 \textbf{Example 1. Example of the midichn opcode.}

\begin{lstlisting}
/* midichn.orc */
; Initialize the global variables.
sr = 44100
kr = 4410
ksmps = 10
nchnls = 1

; Instrument #1.
instr 1
  i1 midichn

  print i1
endin
/* midichn.orc */
        
\end{lstlisting}
\begin{lstlisting}
/* midichn.sco */
; Play Instrument #1 for 12 seconds.
i 1 0 12
e
/* midichn.sco */
        
\end{lstlisting}


  Here is an advanced example of the midichn opcode. It uses the files \emph{midichn\_advanced.mid}
, \emph{midichn\_advanced.orc}
, and \emph{midichn\_advanced.sco}
. 


  Don't forget that you must include the \emph{-F flag}
 when using an external MIDI file like ``midichn\_advanced.mid''. 


 


 \textbf{Example 2. An advanced example of the midichn opcode.}

\begin{lstlisting}
/* midichn_advanced.orc - written by Istvan Varga */
sr	=  44100
ksmps	=  10
nchnls	=  1

	massign  1, 1		; all channels use instr 1
	massign  2, 1
	massign  3, 1
	massign  4, 1
	massign  5, 1
	massign  6, 1
	massign  7, 1
	massign  8, 1
	massign  9, 1
	massign 10, 1
	massign 11, 1
	massign 12, 1
	massign 13, 1
	massign 14, 1
	massign 15, 1
	massign 16, 1

gicnt	=  0			; note counter

	instr 1

gicnt	=  gicnt + 1	; update note counter
kcnt	init gicnt	; copy to local variable
ichn	midichn		; get channel number
istime	times		; note-on time

	if (ichn > 0.5) goto l2		; MIDI note
	printks "note %.0f (time = %.2f) was activated from the score\\n", \
		3600, kcnt, istime
	goto l1
l2:
	printks "note %.0f (time = %.2f) was activated from channel %.0f\\n", \
		3600, kcnt, istime, ichn
l1:
	endin
/* midichn_advanced.orc - written by Istvan Varga */
        
\end{lstlisting}
\begin{lstlisting}
/* midichn_advanced.sco - written by Istvan Varga */
t 0 60
f 0 6 2 -2 0
i 1 1 0.5
i 1 4 0.5
e
/* midichn_advanced.sco - written by Istvan Varga */
        
\end{lstlisting}
 Its output should include lines like: \begin{lstlisting}
note 7 (time = 0.00) was activated from channel 4
note 8 (time = 0.00) was activated from channel 2
      
\end{lstlisting}
\subsection*{See Also}


 \emph{pgmassign}

\subsection*{Credits}


 


 


\begin{tabular}{cc}
Author: Istvan Varga &May 2002

\end{tabular}



 


 The simple example was written by Kevin Conder.


 New in version 4.20
%\hline 


\begin{comment}
\begin{tabular}{lcr}
Previous &Home &Next \\
midichannelaftertouch &Up &midicontrolchange

\end{tabular}


\end{document}
\end{comment}
