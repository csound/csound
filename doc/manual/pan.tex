\begin{comment}
\documentclass[10pt]{article}
\usepackage{fullpage, graphicx, url}
\setlength{\parskip}{1ex}
\setlength{\parindent}{0ex}
\title{pan}
\begin{document}


\begin{tabular}{ccc}
The Alternative Csound Reference Manual & & \\
Previous & &Next

\end{tabular}

%\hline 
\end{comment}
\section{pan}
pan�--� Distribute an audio signal amongst four channels. \subsection*{Description}


  Distribute an audio signal amongst four channels with localization control. 
\subsection*{Syntax}


 a1, a2, a3, a4 \textbf{pan}
 asig, kx, ky, ifn [, imode] [, ioffset]
\subsection*{Initialization}


 \emph{ifn}
 -- function table number of a stored pattern describing the amplitude growth in a speaker channel as sound moves towards it from an adjacent speaker. Requires extended guard-point. 


 \emph{imode}
 (optional) -- mode of the \emph{kx, ky}
 position values. 0 signifies raw index mode, 1 means the inputs are normalized (0 - 1). The default value is 0. 


 \emph{ioffset}
 (optional) -- offset indicator for \emph{kx, ky}
. 0 infers the origin to be at channel 3 (left rear); 1 requests an axis shift to the quadraphonic center. The default value is 0. 
\subsection*{Performance}


 \emph{pan}
 takes an input signal \emph{asig}
 and distributes it amongst four outputs (essentially quad speakers) according to the controls \emph{kx}
 and \emph{ky}
. For normalized input (mode=1) and no offset, the four output locations are in order: left-front at (0,1), right-front at (1,1), left-rear at the origin (0,0), and right-rear at (1,0). In the notation (\emph{kx}
, \emph{ky)}
, the coordinates \emph{kx}
 and \emph{ky}
, each ranging 0 - 1, thus control the 'rightness' and 'forwardness' of a sound location. 


  Movement between speakers is by amplitude variation, controlled by the stored function table \emph{ifn}
. As \emph{kx}
 goes from 0 to 1, the strength of the right-hand signals will grow from the left-most table value to the right-most, while that of the left-hand signals will progress from the right-most table value to the left-most. For a simple linear pan, the table might contain the linear function 0 - 1. A more correct pan that maintains constant power would be obtained by storing the first quadrant of a sinusoid. Since pan will scale and truncate \emph{kx}
 and \emph{ky}
 in simple table lookup, a medium-large table (say 8193) should be used. 


 \emph{kx, ky}
 values are not restricted to 0 - 1. A circular motion passing through all four speakers (inscribed) would have a diameter of root 2, and might be defined by a circle of radius R = root 1/2 with center at (.5,.5). \emph{kx, ky}
 would then come from Rcos(angle), Rsin(angle), with an implicit origin at (.5,.5) (i.e. \emph{ioffset}
 = 1). Unscaled raw values operate similarly. Sounds can thus be located anywhere in the polar or Cartesian plane; points lying outside the speaker square are projected correctly onto the square's perimeter as for a listener at the center. 
\subsection*{Examples}


 


 
\begin{lstlisting}
\emph{instr}
     1
  k1           \emph{phasor}
    1/p3                     ; fraction of circle
  k2           \emph{tablei}
    k1, 1, 1                 ; sin of angle (sinusoid in f1)
  k3           \emph{tablei}
    k1, 1, 1, .25, 1         ; cos of angle (sin offset 1/4 circle)
  a1           \emph{oscili}
    10000,440, 1             ; audio signal..
  a1,a2,a3,a4  \emph{pan}
       a1, k2/2, k3/2, 2, 1, 1  ; sent in a circle (f2=1st quad sin)
                                 
               \emph{outq}
 a1, a2, a3, a4
\emph{endin}

        
\end{lstlisting}


 
%\hline 


\begin{comment}
\begin{tabular}{lcr}
Previous &Home &Next \\
p &Up &pareq

\end{tabular}


\end{document}
\end{comment}
