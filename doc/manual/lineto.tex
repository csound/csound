\begin{comment}
\documentclass[10pt]{article}
\usepackage{fullpage, graphicx, url}
\setlength{\parskip}{1ex}
\setlength{\parindent}{0ex}
\title{lineto}
\begin{document}


\begin{tabular}{ccc}
The Alternative Csound Reference Manual & & \\
Previous & &Next

\end{tabular}

%\hline 
\end{comment}
\section{lineto}
lineto�--� Generate glissandos starting from a control signal. \subsection*{Description}


  Generate glissandos starting from a control signal. 
\subsection*{Syntax}


 kr \textbf{lineto}
 ksig, ktime
\subsection*{Performance}


 \emph{kr}
 -- Output signal. 


 \emph{ksig}
 -- Input signal. 


 \emph{ktime}
 -- Time length of glissando in seconds. 


 \emph{lineto}
 adds glissando (i.e. straight lines) to a stepped input signal (for example, produced by \emph{randh}
 or \emph{lpshold}
). It generates a straight line starting from previous step value, reaching the new step value in \emph{ktime}
 seconds. When the new step value is reached, such value is held until a new step occurs. Be sure that \emph{ktime}
 argument value is smaller than the time elapsed between two consecutive steps of the original signal, otherwise discontinuities will occur in output signal. 


  When used together with the output of \emph{lpshold}
 it emulates the glissando effect of old analog sequencers. 
\subsection*{See Also}


 \emph{tlineto}

\subsection*{Credits}


 Author: Gabriel Maldonado


 New in Version 4.13
%\hline 


\begin{comment}
\begin{tabular}{lcr}
Previous &Home &Next \\
linenr &Up &linrand

\end{tabular}


\end{document}
\end{comment}
