\begin{comment}
\documentclass[10pt]{article}
\usepackage{fullpage, graphicx, url}
\setlength{\parskip}{1ex}
\setlength{\parindent}{0ex}
\title{pitch}
\begin{document}


\begin{tabular}{ccc}
The Alternative Csound Reference Manual & & \\
Previous & &Next

\end{tabular}

%\hline 
\end{comment}
\section{pitch}
pitch�--� Tracks the pitch of a signal. \subsection*{Description}


  Using the same techniques as \emph{spectrum}
 and \emph{specptrk}
, pitch tracks the pitch of the signal in octave point decimal form, and amplitude in dB. 
\subsection*{Syntax}


 koct, kamp \textbf{pitch}
 asig, iupdte, ilo, ihi, idbthresh [, ifrqs] [, iconf] [, istrt] [, iocts] [, iq] [, inptls] [, irolloff] [, iskip]
\subsection*{Initialization}


 \emph{iupdte}
 -- length of period, in seconds, that outputs are updated 


 \emph{ilo}
, \emph{ihi}
 -- range in which pitch is detected, expressed in octave point decimal 


 \emph{idbthresh}
 -- amplitude, expressed in decibels, necessary for the pitch to be detected. Once started it continues until it is 6 dB down. 


 \emph{ifrqs}
 (optional) -- number of divisons of an octave. Default is 12 and is limited to 120. 


 \emph{iconf}
 (optional) -- the number of conformations needed for an octave jump. Default is 10. 


 \emph{istrt}
 (optional) -- starting pitch for tracker. Default value is (\emph{ilo}
 + \emph{ihi}
)/2. 


 \emph{iocts}
 (optional) -- number of octave decimations in spectrum. Default is 6. 


 \emph{iq}
 (optional) -- Q of analysis filters. Default is 10. 


 \emph{inptls}
 (optional) -- number of harmonics, used in matching. Computation time increases with the number of harmonics. Default is 4. 


 \emph{irolloff}
 (optional) -- amplitude rolloff for the set of filters expressed as fraction per octave. Values must be positive. Default is 0.6. 


 \emph{iskip}
 (optional) -- if non-zero, skips initialization. Default is 0. 
\subsection*{Performance}


 \emph{koct}
 -- The pitch output, given in the octave point decimal format. 


 \emph{kamp}
 -- The amplitude output. 


 \emph{pitch}
 analyzes the input signal, \emph{asig}
, to give a pitch/amplitude pair of outputs, for the strongest frequency in the signal. The value is updated every \emph{iupdte}
 seconds. 


  The number of partials and rolloff fraction can effect the pitch tracking, so some experimentation may be necessary. Suggested values are 4 or 5 harmonics, with rolloff 0.6, up to 10 or 12 harmonics with rolloff 0.75 for complex timbres, with a weak fundamental. 
\subsection*{Examples}


  Here is an example of the pitch opcode. It uses the files \emph{pitch.orc}
, \emph{pitch.sco}
 and \emph{mary.wav}
. 


 \textbf{Example 1. Example of the pitch opcode.}

\begin{lstlisting}
/* pitch.orc */
; Initialize the global variables.
sr = 44100
kr = 44100
ksmps = 1
nchnls = 1

; Instrument #1 - play an audio file without effects.
instr 1
  asig soundin "mary.wav"
  out asig
endin

; Instrument #2 - track the pitch of an audio file.
instr 2
  iupdte = 0.01
  ilo = 7
  ihi = 9
  idbthresh = 10
  ifrqs = 12
  iconf = 10
  istrt = 8

  asig soundin "mary.wav"

  ; Follow the audio file, get its pitch and amplitude.
  koct, kamp pitch asig, iupdte, ilo, ihi, idbthresh, ifrqs, iconf, istrt

  ; Re-synthesize the audio file with a different sounding waveform.
  kamp2 = kamp * 10
  kcps = cpsoct(koct)
  a1 oscil kamp2, kcps, 1

  out a1
endin
/* pitch.orc */
        
\end{lstlisting}
\begin{lstlisting}
/* pitch.sco */
; Table #1: A different sounding waveform.
f 1 0 32768 11 7 3 .7

; Play Instrument #1, the audio file, for three seconds.
i 1 0 3
; Play Instrument #2, the "re-synthesized" waveform, for three seconds.
i 2 3 3
e
/* pitch.sco */
        
\end{lstlisting}
\subsection*{Credits}


 


 


\begin{tabular}{cccc}
Author: John ffitch &University of Bath, Codemist Ltd. &Bath, UK &April 1999

\end{tabular}



 


 Example written by Kevin Conder.


 New in Csound version 3.54
%\hline 


\begin{comment}
\begin{tabular}{lcr}
Previous &Home &Next \\
pinkish &Up &pitchamdf

\end{tabular}


\end{document}
\end{comment}
