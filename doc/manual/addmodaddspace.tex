\begin{comment}
\documentclass[10pt]{article}
\usepackage{fullpage, graphicx, url}
\setlength{\parskip}{1ex}
\setlength{\parindent}{0ex}
\title{Additional Space}
\begin{document}


\begin{tabular}{ccc}
The Alternative Csound Reference Manual & & \\
Previous &Adding your own Cmodules to Csound &Next

\end{tabular}

%\hline 
\end{comment}
\section{Additional Space}


  Sometimes the space requirement of a module is too large to be part of a structure (upper limit 65535 bytes), or it is dependent on an i-arg value which is not known until initialization. Additional space can be dynamically allocated and properly managed by including the line 


 
\begin{lstlisting}
AUXCH      auxch;
        
\end{lstlisting}


 


  in the defined structure (*p), then using the following style of code in the init module: 


 
\begin{lstlisting}
if (p-auxch.auxp == NULL)
  auxalloc(npoints * sizeof(float), &p-auxch);
        
\end{lstlisting}


 


  The address of this auxiliary space is kept in a chain of such spaces belonging to this instrument, and is automatically managed while the instrument is being duplicated or garbage-collected during performance. The assignment 


 
\begin{lstlisting}
char *auxp = p-auxch.auxp;
        
\end{lstlisting}


 


  will find the allocated space for init-time and perf-time use. See the LINSEG structure in opcodes1.h and the code for lsgset() and klnseg() in opcodes1.c. 
%\hline 


\begin{comment}
\begin{tabular}{lcr}
Previous &Home &Next \\
Adding your own Cmodules to Csound &Up &File Sharing

\end{tabular}


\end{document}
\end{comment}
