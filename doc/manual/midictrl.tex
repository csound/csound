\begin{comment}
\documentclass[10pt]{article}
\usepackage{fullpage, graphicx, url}
\setlength{\parskip}{1ex}
\setlength{\parindent}{0ex}
\title{midictrl}
\begin{document}


\begin{tabular}{ccc}
The Alternative Csound Reference Manual & & \\
Previous & &Next

\end{tabular}

%\hline 
\end{comment}
\section{midictrl}
midictrl�--� Get the current value (0-127) of a specified MIDI controller. \subsection*{Description}


  Get the current value (0-127) of a specified MIDI controller. 
\subsection*{Syntax}


 ival \textbf{midictrl}
 inum [, imin] [, imax]


 kval \textbf{midictrl}
 inum [, imin] [, imax]
\subsection*{Initialization}


 \emph{inum}
 -- MIDI controller number (0-127) 


 \emph{imin, imax}
 -- set minimum and maximum limits on values obtained. 
\subsection*{Performance}


  Get the current value (0-127) of a specified MIDI controller. 
\subsection*{See Also}


 \emph{aftouch}
, \emph{ampmidi}
, \emph{cpsmidi}
, \emph{cpsmidib}
, \emph{notnum}
, \emph{octmidi}
, \emph{octmidib}
, \emph{pchbend}
, \emph{pchmidi}
, \emph{pchmidib}
, \emph{veloc}

\subsection*{Credits}


 


 


\begin{tabular}{ccc}
Author: Barry L. Vercoe - Mike Berry &MIT - Mills &May 1997

\end{tabular}



 


 Thanks goes to Rasmus Ekman for pointing out the correct MIDI channel and controller number ranges.
%\hline 


\begin{comment}
\begin{tabular}{lcr}
Previous &Home &Next \\
midicontrolchange &Up &mididefault

\end{tabular}


\end{document}
\end{comment}
