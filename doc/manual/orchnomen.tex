\begin{comment}
\documentclass[10pt]{article}
\usepackage{fullpage, graphicx, url}
\setlength{\parskip}{1ex}
\setlength{\parindent}{0ex}
\title{Nomenclature}
\begin{document}


\begin{tabular}{ccc}
The Alternative Csound Reference Manual & & \\
Previous &Syntax of the Orchestra &Next

\end{tabular}

%\hline 
\end{comment}
\section{Nomenclature}


  Throughout this document, opcodes are indicated in \emph{boldface}
 and their argument and result mnemonics, when mentioned in the text, are given in \emph{italics}
. Argument names are generally mnemonic (\emph{amp}
, \emph{phs}
), and the result is usually denoted by the letter \emph{r}
. Both are preceded by a type qualifier \emph{i, k, a,}
 or \emph{x}
 (e.g. \emph{kamp, iphs, ar}
). The prefix \emph{i}
 denotes scalar values valid at note init time; prefixes \emph{k}
 or \emph{a}
 denote control (scalar) and audio (vector) values, modified and referenced continuously throughout performance (i.e. at every control period while the instrument is active). Arguments are used at the prefix-listed times; results are created at their listed times, then remain available for use as inputs elsewhere. With few exceptions, argument rates may not exceed the rate of the result. The validity of inputs is defined by the following: 


 
\begin{itemize}
\item 

 arguments with prefix \emph{i}
 must be valid at init time;

\item 

 arguments with prefix \emph{k}
 can be either control or init values (which remain valid);

\item 

 arguments with prefix \emph{a}
 must be vector inputs;

\item 

 arguments with prefix \emph{x}
 may be either vector or scalar (the compiler will distinguish).


\end{itemize}


  All arguments, unless otherwise stated, can be expressions whose results conform to the above. Most opcodes (such as \emph{linen}
 and \emph{oscil}
) can be used in more than one mode, which one being determined by the prefix of the result symbol. 


  Thoughout this manual, the term ``opcode'' is used to indicate a command that usually produces an a-, k-, or i-rate output, and always forms the basis of a complete Csound orchestra statement. Items such as ``\emph{+}
`` or ``\emph{sin}
(x)'' or, ``( a \emph{$>$=}
 b \emph{?}
 c \emph{:}
 d)'' are called ``operators.'' 
%\hline 


\begin{comment}
\begin{tabular}{lcr}
Previous &Home &Next \\
Syntax of the Orchestra &Up &Orchestra Statement Types

\end{tabular}


\end{document}
\end{comment}
