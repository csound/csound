\begin{comment}
\documentclass[10pt]{article}
\usepackage{fullpage, graphicx, url}
\setlength{\parskip}{1ex}
\setlength{\parindent}{0ex}
\title{0dbfs}
\begin{document}


\begin{tabular}{ccc}
The Alternative Csound Reference Manual & & \\
Previous & &Next

\end{tabular}

%\hline 
\end{comment}
\section{0dbfs}
0dbfs�--� Sets the value of 0 decibels using full scale amplitude. \subsection*{Description}


  Sets the value of 0 decibels using full scale amplitude. 
\subsection*{Syntax}


 \textbf{0dbfs}
 = iarg
\subsection*{Initialization}


 \emph{iarg}
 -- the value of 0 decibels using full scale amplitude. 
\subsection*{Performance}


  The default is 32767, so all existing orcs \emph{should}
 work. 


  These calls should all work: 


 
\begin{lstlisting}
ipeak = 0dbfs
        
\end{lstlisting}


 


 
\begin{lstlisting}
asig oscil 0dbfs,freq,1
out  asig * 0.3 * 0dbfs
        
\end{lstlisting}


 
 and so on. 

  As for documentation: the usage should be obvious - the main thing is for people to start to code 0dbfs-relatively (and use the \emph{ampdb()}
 opcodes a lot more!), rather than use explicit sample values. 


  Floats written to a file, when \emph{0dbfs = 1}
, will in effect go through no range translation at all. So the nunbers in the file are exactly what the orc says they are. 


 


\begin{tabular}{cc}
\textbf{BIG NB}
 \\
� &

  All the main sample formats are supported, but I haven't got around to dealing with the char formats. Probably it's straight-forward... 


  I have tried to cover the main utils - adsyn,lpanal etc. But there are bound to be things missing, sorry. 


  Some of the parsing code is a bit grungy because I have a variable with a leading digit! 


\end{tabular}

\subsection*{Examples}


  Here is an example of the 0dbfs opcode. It uses the files \emph{0dbfs.orc}
 and \emph{0dbfs.sco}
. 


 \textbf{Example 1. Example of the 0dbfs opcode.}

\begin{lstlisting}
/* 0dbfs.orc */
; Initialize the global variables.
sr = 44100
kr = 4410
ksmps = 10
nchnls = 1

; Set the 0dbfs to the 16-bit maximum.
0dbfs = 32767

; Instrument #1.
instr 1
  ; Linearly increase the amplitude value "kamp" from 
  ; 0 to 1 over the duration defined by p3.
  kamp line 0, p3, 1

  ; Generate a basic tone using our amplitude value.
  a1 oscil kamp, 440, 1

  ; Multiply the basic tone (with its amplitude between 
  ; 0 and 1) by the full-scale 0dbfs value.
  out a1 * 0dbfs
endin
/* 0dbfs.orc */
        
\end{lstlisting}
\begin{lstlisting}
/* 0dbfs.sco */
; Table #1, a sine wave.
f 1 0 16384 10 1

; Play Instrument #1 for three seconds.
i 1 0 3
e
/* 0dbfs.sco */
        
\end{lstlisting}
\subsection*{Credits}


 


 


\begin{tabular}{cc}
Author: Richard Dobson &May 2002

\end{tabular}



 


 Example written by Kevin Conder.


 New in version 4.20
%\hline 


\begin{comment}
\begin{tabular}{lcr}
Previous &Home &Next \\
|| &Up &a

\end{tabular}


\end{document}
\end{comment}
