\begin{comment}
\documentclass[10pt]{article}
\usepackage{fullpage, graphicx, url}
\setlength{\parskip}{1ex}
\setlength{\parindent}{0ex}
\title{jitter2}
\begin{document}


\begin{tabular}{ccc}
The Alternative Csound Reference Manual & & \\
Previous & &Next

\end{tabular}

%\hline 
\end{comment}
\section{jitter2}
jitter2�--� Generates a segmented line with user-controllable random segments. \subsection*{Description}


  Generates a segmented line with user-controllable random segments. 
\subsection*{Syntax}


 kout \textbf{jitter2}
 ktotamp, kamp1, kcps1, kamp2, kcps2, kamp3, kcps3
\subsection*{Performance}


 \emph{ktotamp}
 -- Resulting amplitude of jitter2 


 \emph{kamp1}
 -- Amplitude of the first jitter component 


 \emph{kcps1}
 -- Speed of random variation of the first jitter component (expressed in cps) 


 \emph{kamp2}
 -- Amplitude of the second jitter component 


 \emph{kcps2}
 -- Speed of random variation of the second jitter component (expressed in cps) 


 \emph{kamp3}
 -- Amplitude of the third jitter component 


 \emph{kcps3}
 -- Speed of random variation of the third jitter component (expressed in cps) 


 \emph{jitter2}
 also generates a segmented line such as \emph{jitter}
, but in this case the result is similar to the sum of three \emph{randi}
 opcodes, each one with a different amplitude and frequency value (see \emph{randi}
 for more details), that can be varied at k-rate. Different effects can be obtained by varying the input arguments. 


 \emph{jitter2}
 can be used to make more natural and ``analog-sounding'' some static, dull sound. For best results, it is suggested to keep its amplitude moderate. 
\subsection*{Examples}


  Here is an example of the jitter2 opcode. It uses the files \emph{jitter2.orc}
 and \emph{jitter2.sco}
. 


 \textbf{Example 1. Example of the jitter2 opcode.}

\begin{lstlisting}
/* jitter2.orc */
; Initialize the global variables.
sr = 44100
kr = 4410
ksmps = 10
nchnls = 2

; Instrument #1 -- plain instrument.
instr 1
  aplain vco 20000, 220, 2, 0.83

  outs aplain, aplain
endin

; Instrument #2 -- instrument with jitter.
instr 2
  ; Create a signal modulated with the jitter2 opcode.
  ktotamp init 2
  kamp1 init 0.66
  kcps1 init 3
  kamp2 init 0.66
  kcps2 init 3
  kamp3 init 0.66
  kcps3 init 3
  kj jitter2 ktotamp, kamp1, kcps1, kamp2, kcps2, \
             kamp3, kcps3

  aplain vco 20000, 220, 2, 0.83
  ajitter vco 20000, 220+kj, 2, 0.83

  outs aplain, ajitter
endin
/* jitter2.orc */
        
\end{lstlisting}
\begin{lstlisting}
/* jitter2.sco */
; Table #1, a sine wave.
f 1 0 16384 10 1

; Play Instrument #1 for 3 seconds.
i 1 0 3
; Play Instrument #2 for 3 seconds.
i 2 3 3
e
/* jitter2.sco */
        
\end{lstlisting}
\subsection*{See Also}


 \emph{jitter}
, \emph{vibr}
, \emph{vibrato}

\subsection*{Credits}


 Author: Gabriel Maldonado


 Example written by Kevin Conder.


 New in Version 4.15
%\hline 


\begin{comment}
\begin{tabular}{lcr}
Previous &Home &Next \\
jitter &Up &jspline

\end{tabular}


\end{document}
\end{comment}
