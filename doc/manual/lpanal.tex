\begin{comment}
\documentclass[10pt]{article}
\usepackage{fullpage, graphicx, url}
\setlength{\parskip}{1ex}
\setlength{\parindent}{0ex}
\title{lpanal}
\begin{document}


\begin{tabular}{ccc}
The Alternative Csound Reference Manual & & \\
Previous & &Next

\end{tabular}

%\hline 
\end{comment}
\section{lpanal}
lpanal�--� Performs both linear predictive analysis on a soundfile. \subsection*{Description}


  Linear predictive analysis for the Csound \emph{lp generators}

\subsection*{Syntax}


 \textbf{csound -U lpanal}
 [flags] infilename outfilename


 \textbf{lpanal}
 [flags] infilename outfilename
\subsection*{Initialization}


 \emph{lpanal}
 performs both lpc and pitch-tracking analysis on a soundfile to produce a time-ordered sequence of \emph{frames}
 of control information suitable for Csound resynthesis. Analysis is conditioned by the control flags below. A space is optional between the flag and its value. 


 \emph{-a}
 -- [alternate storage] asks lpanal to write a file with filter poles values rather than the usual filter coefficient files. When \emph{lpread}
 / \emph{lpreson}
 are used with pole files, automatic stabilization is performed and the filter should not get wild. (This is the default in the Windows GUI) - Changed by Marc Resibois. 


 \emph{-s srate}
 -- sampling rate of the audio input file. This will over-ride the srate of the soundfile header, which otherwise applies. If neither is present, the default is 10000. 


 \emph{-c channel}
 -- channel number sought. The default is 1. 


 \emph{-b begin}
 -- beginning time (in seconds) of the audio segment to be analyzed. The default is 0.0 


 \emph{-d duration}
 -- duration (in seconds) of the audio segment to be analyzed. The default of 0.0 means to the end of the file. 


 \emph{-p npoles}
 -- number of poles for analysis. The default is 34, the maximum 50. 


 \emph{-h hopsize}
 -- hop size (in samples) between frames of analysis. This determines the number of frames per second (srate / hopsize) in the output control file. The analysis framesize is hopsize * 2 samples. The default is 200, the maximum 500. 


 \emph{-C string}
 -- text for the comments field of the lpfile header. The default is the null string. 


 \emph{-P mincps}
 -- lowest frequency (in Hz) of pitch tracking. -P0 means no pitch tracking. 


 \emph{-Q maxcps}
 -- highest frequency (in Hz) of pitch tracking. The narrower the pitch range, the more accurate the pitch estimate. The defaults are -P70, -Q200. 


 \emph{-v verbosity}
 -- level of terminal information during analysis. 


 
\begin{itemize}
\item 

 0 = none

\item 

 1 = verbose

\item 

 2 = debug


\end{itemize}
 The default is 0. \subsection*{Examples}


 


 
\begin{lstlisting}
\emph{lpanal}
 -a -p26 -d2.5 -P100 -Q400 audiofile.test lpfil22
        
\end{lstlisting}


 
 will analyze the first 2.5 seconds of file ``audiofile.test'', producing srate/200 frames per second, each containing 26-pole filter coefficients and a pitch estimate between 100 and 400 Hertz. Stabilized (\emph{-a}
) output will be placed in ``lpfil22'' in the current directory. \subsubsection*{File Format }


  Output is a file comprised of an identifiable header plus a set of frames of floating point analysis data. Each frame contains four values of pitch and gain information, followed by \emph{npoles}
 filter coefficients. The file is readable by Csound's \emph{lpread}
. 


 \emph{lpanal}
 is an extensive modification of Paul Lanksy's lpc analysis programs. 
%\hline 


\begin{comment}
\begin{tabular}{lcr}
Previous &Home &Next \\
hetro &Up &pvanal

\end{tabular}


\end{document}
\end{comment}
