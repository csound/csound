\begin{comment}
\documentclass[10pt]{article}
\usepackage{fullpage, graphicx, url}
\setlength{\parskip}{1ex}
\setlength{\parindent}{0ex}
\title{maxalloc}
\begin{document}


\begin{tabular}{ccc}
The Alternative Csound Reference Manual & & \\
Previous & &Next

\end{tabular}

%\hline 
\end{comment}
\section{maxalloc}
maxalloc�--� Limits the number of allocations of an instrument. \subsection*{Description}


  Limits the number of allocations of an instrument. 
\subsection*{Syntax}


 \textbf{maxalloc}
 insnum, icount
\subsection*{Initialization}


 \emph{insnum}
 -- instrument number 


 \emph{icount}
 -- number of instrument allocations 
\subsection*{Performance}


  All instances of \emph{maxalloc}
 must be defined in the header section, not in the instrument body. 
\subsection*{Examples}


  Here is an example of the maxalloc opcode. It uses the files \emph{maxalloc.orc}
 and \emph{maxalloc.sco}
. 


 \textbf{Example 1. Example of the maxalloc opcode.}

\begin{lstlisting}
/* maxalloc.orc */
; Initialize the global variables.
sr = 44100
kr = 4410
ksmps = 10
nchnls = 1

; Limit Instrument #1 to three instances.
maxalloc 1, 3
 
; Instrument #1
instr 1
  ; Generate a waveform, get the cycles per second from the 4th p-field.
  a1 oscil 6500, p4, 1
  out a1
endin
/* maxalloc.orc */
        
\end{lstlisting}
\begin{lstlisting}
/* maxalloc.sco */
; Just generate a nice, ordinary sine wave.
f 1 0 32768 10 1

; Play five instances of Instrument #1 for one second.
; Note that 4th p-field contains cycles per second.
i 1 0 1 220
i 1 0 1 440
i 1 0 1 880
i 1 0 1 1320
i 1 0 1 1760
e
/* maxalloc.sco */
        
\end{lstlisting}
 Its output should contain a message like this: \begin{lstlisting}
WARNING: cannot allocate last note because it exceeds instr maxalloc
      
\end{lstlisting}
\subsection*{See Also}


 \emph{cpuprc}
, \emph{prealloc}

\subsection*{Credits}


 


 


\begin{tabular}{ccc}
Author: Gabriel Maldonado &Italy &July 1999

\end{tabular}



 


 Example written by Kevin Conder.


 New in Csound version 3.57
%\hline 


\begin{comment}
\begin{tabular}{lcr}
Previous &Home &Next \\
massign &Up &mclock

\end{tabular}


\end{document}
\end{comment}
