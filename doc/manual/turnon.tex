\begin{comment}
\documentclass[10pt]{article}
\usepackage{fullpage, graphicx, url}
\setlength{\parskip}{1ex}
\setlength{\parindent}{0ex}
\title{turnon}
\begin{document}


\begin{tabular}{ccc}
The Alternative Csound Reference Manual & & \\
Previous & &Next

\end{tabular}

%\hline 
\end{comment}
\section{turnon}
turnon�--� Activate an instrument for an indefinite time. \subsection*{Description}


  Activate an instrument for an indefinite time. 
\subsection*{Syntax}


 \textbf{turnon}
 insnum [, itime]
\subsection*{Initialization}


 \emph{insnum}
 -- instrument number to be activated 


 \emph{itime}
 (optional, default=0) -- delay, in seconds, after which instrument \emph{insnum}
 will be activated. Default is 0. 
\subsection*{Performance}


 \emph{turnon}
 activates instrument \emph{insnum}
 after a delay of \emph{itime}
 seconds, or immediately if \emph{itime}
 is not specified. Instrument is active until explicitly turned off. (See \emph{turnoff}
.) 
%\hline 


\begin{comment}
\begin{tabular}{lcr}
Previous &Home &Next \\
turnoff &Up &unirand

\end{tabular}


\end{document}
\end{comment}
