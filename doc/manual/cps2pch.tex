\begin{comment}
\documentclass[10pt]{article}
\usepackage{fullpage, graphicx, url}
\setlength{\parskip}{1ex}
\setlength{\parindent}{0ex}
\title{cps2pch}
\begin{document}


\begin{tabular}{ccc}
The Alternative Csound Reference Manual & & \\
Previous & &Next

\end{tabular}

%\hline 
\end{comment}
\section{cps2pch}
cps2pch�--� Converts a pitch-class value into cycles-per-second for equal divisions of the octave. \subsection*{Description}


  Converts a pitch-class value into cycles-per-second (Hz) for equal divisions of the octave. 
\subsection*{Syntax}


 icps \textbf{cps2pch}
 ipch, iequal
\subsection*{Initialization}


 \emph{ipch}
 -- Input number of the form 8ve.pc, indicating an 'octave' and which note in the octave. 


 \emph{iequal}
 -- if positive, the number of equal intervals into which the 'octave' is divided. Must be less than or equal to 100. If negative, is the number of a table of frequency multipliers. 


 


%\begin{tabular}{cc}
\textbf{Note} \\ &

 


 
\begin{enumerate}
\item The following are essentially the same 


 ia��=��\emph{cpspch}
(8.02)\\ 
 ib�����\emph{cps2pch}
��8.02,�12\\ 
 ic�����\emph{cpsxpch}
��8.02,�12,�2,�1.02197503906\\ 
 ������������

\item These are opcodes not functions 

\item Negative values of \emph{ipch}
 are allowed. 


\end{enumerate}


%\end{tabular}

\subsection*{Examples}


  Here is an example of the cps2pch opcode. It uses the files \emph{cps2pch.orc}
 and \emph{cps2pch.sco}
. 


 \textbf{Example 1. Example of the cps2pch opcode.}

\begin{lstlisting}
/* cps2pch.orc */
; Initialize the global variables.
sr = 44100
kr = 4410
ksmps = 10
nchnls = 1

; Instrument #1.
instr 1
  ; Use a normal twelve-tone scale.
  ipch = 8.02
  iequal = 12

  icps cps2pch ipch, iequal

  print icps
endin
/* cps2pch.orc */
        
\end{lstlisting}
\begin{lstlisting}
/* cps2pch.sco */
; Play Instrument #1 for one second.
i 1 0 1
e
/* cps2pch.sco */
        
\end{lstlisting}
 Its output should include lines like this: \begin{lstlisting}
instr 1:  icps = 293.666
      
\end{lstlisting}


  Here is an example of the cps2pch opcode using a table of frequency multipliers. It uses the files \emph{cps2pch\_ftable.orc}
 and \emph{cps2pch\_ftable.sco}
. 


 \textbf{Example 2. Example of the cps2pch opcode using a table of frequency multipliers.}

\begin{lstlisting}
/* cps2pch_ftable.orc */
; Initialize the global variables.
sr = 44100
kr = 4410
ksmps = 10
nchnls = 1

; Instrument #1.
instr 1
  ipch = 8.02

  ; Use Table #1, a table of frequency multipliers.
  icps cps2pch ipch, -1

  print icps
endin
/* cps2pch_ftable.orc */
        
\end{lstlisting}
\begin{lstlisting}
/* cps2pch_ftable.sco */
; Table #1: a table of frequency multipliers.
; Creates a 10-note scale of unequal divisions.
f 1 0 16 -2 1 1.1 1.2 1.3 1.4 1.6 1.7 1.8 1.9

; Play Instrument #1 for one second.
i 1 0 1
e
/* cps2pch_ftable.sco */
        
\end{lstlisting}
 Its output should include lines like this: \begin{lstlisting}
instr 1:  icps = 313.951
      
\end{lstlisting}


  Here is an example of the cps2pch opcode using a 19ET scale. It uses the files \emph{cps2pch\_19et.orc}
 and \emph{cps2pch\_19et.sco}
. 


 \textbf{Example 3. Example of the cps2pch opcode using a 19ET scale.}

\begin{lstlisting}
/* cps2pch_19et.orc */
; Initialize the global variables.
sr = 44100
kr = 4410
ksmps = 10
nchnls = 1

; Instrument #1.
instr 1
  ; Use 19ET scale.
  ipch = 8.02
  iequal = 19

  icps cps2pch ipch, iequal

  print icps
endin
/* cps2pch_19et.orc */
        
\end{lstlisting}
\begin{lstlisting}
/* cps2pch_19et.sco */
; Play Instrument #1 for one second.
i 1 0 1
e
/* cps2pch_19et.sco */
        
\end{lstlisting}
 Its output should include lines like this: \begin{lstlisting}
instr 1:  icps = 281.429
      
\end{lstlisting}
\subsection*{See Also}

\emph{cpspch}, \emph{cpsxpch}

\subsection*{Credits}


 


 


\begin{tabular}{cccc}
Author: John ffitch &University of Bath/Codemist Ltd. &Bath, UK &1997

\end{tabular}



 


 


 


\begin{tabular}{ccc}
Author: Gabriel Maldonado &Italy &1998

\end{tabular}



 


 New in Csound version 3.492
%\hline 


\begin{comment}
\begin{tabular}{lcr}
Previous &Home &Next \\
cosinv &Up &cpsmidi

\end{tabular}


\end{document}
\end{comment}
