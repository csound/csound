\begin{comment}
\documentclass[10pt]{article}
\usepackage{fullpage, graphicx, url}
\setlength{\parskip}{1ex}
\setlength{\parindent}{0ex}
\title{lowres}
\begin{document}


\begin{tabular}{ccc}
The Alternative Csound Reference Manual & & \\
Previous & &Next

\end{tabular}

%\hline 
\end{comment}
\section{lowres}
lowres�--� Another resonant lowpass filter. \subsection*{Description}


 \emph{lowres}
 is a resonant lowpass filter. 
\subsection*{Syntax}


 ar \textbf{lowres}
 asig, kcutoff, kresonance [, iskip]
\subsection*{Initialization}


 \emph{iskip}
 -- initial disposition of internal data space. A zero value will clear the space; a non-zero value will allow previous information to remain. The default value is 0. 
\subsection*{Performance}


 \emph{asig}
 -- input signal 


 \emph{kcutoff}
 -- filter cutoff frequency point 


 \emph{kresonance}
 -- resonance amount 


 \emph{lowres}
 is a resonant lowpass filter derived from a Hans Mikelson orchestra. This implementation is much faster than implementing it in Csound language, and it allows \emph{kr}
 lower than \emph{sr}
. \emph{kcutoff}
 is not in Hz and \emph{kresonance}
 is not in dB, so experiment for the finding best results. 
\subsection*{Examples}


  Here is an example of the lowres opcode. It uses the files \emph{lowres.orc}
, \emph{lowres.sco}
 and \emph{beats.wav}
. 


 \textbf{Example 1. Example of the lowres opcode.}

\begin{lstlisting}
/* lowres.orc */
; Initialize the global variables.
sr = 44100
kr = 4410
ksmps = 10
nchnls = 1

; Instrument #1.
instr 1
  ; Use a nice sawtooth waveform.
  asig vco 5000, 440, 1

  ; Vary the cutoff frequency from 30 to 300 Hz.
  kcutoff line 30, p3, 300
  kresonance = 10

  ; Apply the filter.
  a1 lowres asig, kcutoff, kresonance 

  out a1
endin
/* lowres.orc */
        
\end{lstlisting}
\begin{lstlisting}
/* lowres.sco */
; Table #1, a sine wave for the vco opcode.
f 1 0 16384 10 1

; Play Instrument #1 for two seconds.
i 1 0 2
e
/* lowres.sco */
        
\end{lstlisting}
\subsection*{See Also}


 \emph{lowresx}

\subsection*{Credits}


 


 


\begin{tabular}{cc}
Author: Gabriel Maldonado (adapted by John ffitch) &Italy

\end{tabular}



 


 Example written by Kevin Conder.


 New in Csound version 3.49
%\hline 


\begin{comment}
\begin{tabular}{lcr}
Previous &Home &Next \\
lowpass2 &Up &lowresx

\end{tabular}


\end{document}
\end{comment}
