\begin{comment}
\documentclass[10pt]{article}
\usepackage{fullpage, graphicx, url}
\setlength{\parskip}{1ex}
\setlength{\parindent}{0ex}
\title{tempo}
\begin{document}


\begin{tabular}{ccc}
The Alternative Csound Reference Manual & & \\
Previous & &Next

\end{tabular}

%\hline 
\end{comment}
\section{tempo}
tempo�--� Apply tempo control to an uninterpreted score. \subsection*{Description}


  Apply tempo control to an uninterpreted score. 
\subsection*{Syntax}


 \textbf{tempo}
 ktempo, istartempo
\subsection*{Initialization}


 \emph{istartempo}
 -- initial tempo (in beats per minute). Typically 60. 
\subsection*{Performance}


 \emph{ktempo}
 -- The tempo to which the score will be adjusted. 


 \emph{tempo}
 allows the performance speed of Csound scored events to be controlled from within an orchestra. It operates only in the presence of the Csound \emph{-t}
 flag. When that flag is set, scored events will be performed from their uninterpreted p2 and p3 (beat) parameters, initially at the given command-line tempo. When a \emph{tempo}
 statement is activated in any instrument (\emph{ktempo}
 0.), the operating tempo will be adjusted to \emph{ktempo}
 beats per minute. There may be any number of \emph{tempo}
 statements in an orchestra, but coincident activation is best avoided. 
\subsection*{Examples}


  Here is an example of the tempo opcode. Remember, it only works if you use the \emph{-t}
 flag with Csound. The example uses the files \emph{tempo.orc}
 and \emph{tempo.sco}
. 


 \textbf{Example 1. Example of the tempo opcode.}

\begin{lstlisting}
/* tempo.orc */
; Initialize the global variables.
sr = 44100
kr = 4410
ksmps = 10
nchnls = 1

; Instrument #1.
instr 1
  ; If the fourth p-field is 1, increase the tempo.
  if (p4 == 1) kgoto speedup
    kgoto playit

speedup:
  ; Increase the tempo to 150 beats per minute.
  tempo 150, 60

playit:
  a1 oscil 10000, 440, 1
  out a1
endin
/* tempo.orc */
        
\end{lstlisting}
\begin{lstlisting}
/* tempo.sco */
; Table #1, a sine wave.
f 1 0 16384 10 1

; p4 = plays at a faster tempo (when p4=1).
; Play Instrument #1 at the normal tempo, repeat 3 times.
r3
i 1 00.00 00.10 0
i 1 00.25 00.10 0
i 1 00.50 00.10 0
i 1 00.75 00.10 0
s

; Play Instrument #1 at a faster tempo, repeat 3 times.
r3
i 1 00.00 00.10 1
i 1 00.25 00.10 1
i 1 00.50 00.10 1
i 1 00.75 00.10 1
s

e
/* tempo.sco */
        
\end{lstlisting}
\subsection*{See Also}


 \emph{tempoval}

\subsection*{Credits}


 Example written by Kevin Conder.
%\hline 


\begin{comment}
\begin{tabular}{lcr}
Previous &Home &Next \\
tempest &Up &tempoval

\end{tabular}


\end{document}
\end{comment}
