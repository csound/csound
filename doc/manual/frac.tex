\begin{comment}
\documentclass[10pt]{article}
\usepackage{fullpage, graphicx, url}
\setlength{\parskip}{1ex}
\setlength{\parindent}{0ex}
\title{frac}
\begin{document}


\begin{tabular}{ccc}
The Alternative Csound Reference Manual & & \\
Previous & &Next

\end{tabular}

%\hline 
\end{comment}
\section{frac}
frac�--� Returns the fractional part of a decimal number. \subsection*{Description}


  Returns the fractional part of \emph{x}
. 
\subsection*{Syntax}


 \textbf{frac}
(x) (init-rate or control-rate args only)


  where the argument within the parentheses may be an expression. Value converters perform arithmetic translation from units of one kind to units of another. The result can then be a term in a further expression. 
\subsection*{Examples}


  Here is an example of the frac opcode. It uses the files \emph{frac.orc}
 and \emph{frac.sco}
. 


 \textbf{Example 1. Example of the frac opcode.}

\begin{lstlisting}
/* frac.orc */
; Initialize the global variables.
sr = 44100
kr = 4410
ksmps = 10
nchnls = 1

; Instrument #1.
instr 1
  i1 = 16 / 5
  i2 = frac(i1)

  print i2
endin
/* frac.orc */
        
\end{lstlisting}
\begin{lstlisting}
/* frac.sco */
; Play Instrument #1 for one second.
i 1 0 1
e
/* frac.sco */
        
\end{lstlisting}
 Its output should include a line like this: \begin{lstlisting}
instr 1:  i2 = 0.200
      
\end{lstlisting}
\subsection*{See Also}


 \emph{abs}
, \emph{exp}
, \emph{int}
, \emph{log}
, \emph{log10}
, \emph{i}
, \emph{sqrt}

\subsection*{Credits}


 Example written by Kevin Conder.
%\hline 


\begin{comment}
\begin{tabular}{lcr}
Previous &Home &Next \\
fprints &Up &ftchnls

\end{tabular}


\end{document}
\end{comment}
