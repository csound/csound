\begin{comment}
\documentclass[10pt]{article}
\usepackage{fullpage, graphicx, url}
\setlength{\parskip}{1ex}
\setlength{\parindent}{0ex}
\title{p}
\begin{document}


\begin{tabular}{ccc}
The Alternative Csound Reference Manual & & \\
Previous & &Next

\end{tabular}

%\hline 
\end{comment}
\section{p}
p�--� Show the value in a given p-field. \subsection*{Description}


  Show the value in a given p-field. 
\subsection*{Syntax}


 \textbf{p}
(x) 


  This function works at i-rate and k-rate. 
\subsection*{Initialization}


 \emph{x}
 -- the number of the p-field. 
\subsection*{Performance}


  The value returned by the \emph{p}
 function is the value in a p-field. 
\subsection*{Examples}


  Here is an example of the p opcode. It uses the files \emph{p.orc}
 and \emph{p.sco}
. 


 \textbf{Example 1. Example of the p opcode.}

\begin{lstlisting}
/* p.orc */
; Initialize the global variables.
sr = 44100
kr = 4410
ksmps = 10
nchnls = 1

; Instrument #1.
instr 1
  ; Get the value in the fourth p-field, p4.
  i1 = p(4)

  print i1
endin
/* p.orc */
        
\end{lstlisting}
\begin{lstlisting}
/* p.sco */
; p4 = value to be printed.
; Play Instrument #1 for one second, p4 = 50.375.
i 1 0 1 50.375
e
/* p.sco */
        
\end{lstlisting}
 Its output should include lines like: \begin{lstlisting}
instr 1:  i1 = 50.375
      
\end{lstlisting}
\subsection*{Credits}


 Example written by Kevin Conder.
%\hline 


\begin{comment}
\begin{tabular}{lcr}
Previous &Home &Next \\
outz &Up &pan

\end{tabular}


\end{document}
\end{comment}
