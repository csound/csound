\begin{comment}
\documentclass[10pt]{article}
\usepackage{fullpage, graphicx, url}
\setlength{\parskip}{1ex}
\setlength{\parindent}{0ex}
\title{outic14}
\begin{document}


\begin{tabular}{ccc}
The Alternative Csound Reference Manual & & \\
Previous & &Next

\end{tabular}

%\hline 
\end{comment}
\section{outic14}
outic14�--� Sends 14-bit MIDI controller output at i-rate. \subsection*{Description}


  Sends 14-bit MIDI controller output at i-rate. 
\subsection*{Syntax}


 \textbf{outic14}
 ichn, imsb, ilsb, ivalue, imin, imax
\subsection*{Initialization}


 \emph{ichn}
 -- MIDI channel number (1-16) 


 \emph{imsb}
 -- most significant byte controller number when using 14-bit parameters (0-127) 


 \emph{ilsb}
 -- least significant byte controller number when using 14-bit parameters (0-127) 


 \emph{ivalue}
 -- floating point value 


 \emph{imin}
 -- minimum floating point value (converted in MIDI integer value 0) 


 \emph{imax}
 -- maximum floating point value (converted in MIDI integer value 16383 (14-bit)) 
\subsection*{Performance}


 \emph{outic14}
 (i-rate MIDI 14-bit controller output) sends a pair of controller messages. This opcode can drive 14-bit parameters on MIDI instruments that recognize them. The first control message contains the most significant byte of \emph{ivalue}
 argument while the second message contains the less significant byte. \emph{imsb}
 and \emph{ilsb}
 are the number of the most and less significant controller. 


  This opcode can drive a different value of a parameter for each note currently active. 


  It can scale an i-value floating-point argument according to the \emph{imin}
 and \emph{imax}
 values. For example, set \emph{imin}
 = 1.0 and \emph{imax}
 = 2.0. When the \emph{ivalue}
 argument receives a 2.0 value, the opcode will send a 127 value to the MIDI OUT device. When the \emph{ivalue}
 argument receives a 1.0 value, it will send a 0 value. i-rate opcodes send their message once during instrument initialization. 
\subsection*{See Also}


 \emph{outiat}
, \emph{outic}
, \emph{outipat}
, \emph{outipb}
, \emph{outipc}
, \emph{outkat}
, \emph{outkc14}
, \emph{outkc}
, \emph{outkpat}
, \emph{outkpb}
, \emph{outkpc}

\subsection*{Credits}


 


 


\begin{tabular}{cc}
Author: Gabriel Maldonado &Italy

\end{tabular}



 


 New in Csound version 3.47


 Thanks goes to Rasmus Ekman for pointing out the correct MIDI channel and controller number ranges.
%\hline 


\begin{comment}
\begin{tabular}{lcr}
Previous &Home &Next \\
outic &Up &outipat

\end{tabular}


\end{document}
\end{comment}
