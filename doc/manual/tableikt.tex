\begin{comment}
\documentclass[10pt]{article}
\usepackage{fullpage, graphicx, url}
\setlength{\parskip}{1ex}
\setlength{\parindent}{0ex}
\title{tableikt}
\begin{document}


\begin{tabular}{ccc}
The Alternative Csound Reference Manual & & \\
Previous & &Next

\end{tabular}

%\hline 
\end{comment}
\section{tableikt}
tableikt�--� Provides k-rate control over table numbers. \subsection*{Description}


  k-rate control over table numbers. 


  The standard Csound opcode \emph{tablei}
, when producing a k- or a-rate result, can only use an init-time variable to select the table number. \emph{tableikt}
 accepts k-rate control as well as i-time. In all other respects they are similar to the original opcodes. 
\subsection*{Syntax}


 ar \textbf{tableikt}
 xndx, kfn [, ixmode] [, ixoff] [, iwrap]


 kr \textbf{tableikt}
 kndx, kfn [, ixmode] [, ixoff] [, iwrap]
\subsection*{Initialization}


 \emph{ixmode}
 -- if 0, \emph{xndx}
 and \emph{ixoff}
 ranges match the length of the table. if non-zero \emph{xndx}
 and \emph{ixoff}
 have a 0 to 1 range. Default is 0 


 \emph{ixoff}
 -- if 0, total index is controlled directly by \emph{xndx,}
 ie. the indexing starts from the start of the table. If non-zero, start indexing from somewhere else in the table. Value must be positive and less than the table length (\emph{ixmode}
 = 0) or less than 1 (\emph{ixmode}
 not equal to 0). Default is 0. 


 \emph{iwrap}
 -- if \emph{iwrap}
 = 0, \emph{Limit mode}
: when total index is below 0, then final index is 0.Total index above table length results in a final index of the table length - high out of range total indexes stick at the upper limit of the table. If \emph{iwrap}
 not equal to 0, \emph{Wrap mode}
: total index is wrapped modulo the table length so that all total indexes map into the table. For instance, in a table of length 8, \emph{xndx}
 = 5 and \emph{ixoff}
 = 6 gives a total index of 11, which wraps to a final index of 3. Default is 0. 
\subsection*{Performance}


 \emph{kndx}
 -- Index into table, either a positive number range 


 \emph{xndx}
 -- matching the table length (\emph{ixmode}
 = 0) or a 0 to 1 range (\emph{ixmode}
 not equal to 0) 


 \emph{kfn}
 -- Table number. Must be $>$= 1. Floats are rounded down to an integer. If a table number does not point to a valid table, or the table has not yet been loaded (\emph{GEN01}
) then an error will result and the instrument will be de-activated. 


 


\begin{tabular}{cc}
Caution &\textbf{Caution with k-rate table numbers}
 \\
� &

  At k-rate, if a table number of $<$ 1 is given, or the table number points to a non-existent table, or to one which has a length of 0 (it is to be loaded from a file later) then an error will result and the instrument will be deactivated. \emph{kfn}
 must be initialized at the appropriate rate using \emph{init}
. Attempting to load an i-rate value into \emph{kfn}
 will result in an error. 


\end{tabular}

\subsection*{See Also}


 \emph{tablekt}

\subsection*{Credits}


 


 


\begin{tabular}{ccc}
Author: Robin Whittle &Australia &May 1997

\end{tabular}



 
%\hline 


\begin{comment}
\begin{tabular}{lcr}
Previous &Home &Next \\
tableigpw &Up &tableimix

\end{tabular}


\end{document}
\end{comment}
