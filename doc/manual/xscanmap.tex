\begin{comment}
\documentclass[10pt]{article}
\usepackage{fullpage, graphicx, url}
\setlength{\parskip}{1ex}
\setlength{\parindent}{0ex}
\title{xscanmap}
\begin{document}


\begin{tabular}{ccc}
The Alternative Csound Reference Manual & & \\
Previous & &Next

\end{tabular}

%\hline 
\end{comment}
\section{xscanmap}
xscanmap�--� Allows the position and velocity of a node in a scanned process to be read. \subsection*{Description}


  Allows the position and velocity of a node in a scanned process to be read. 
\subsection*{Syntax}


 kpos, kvel \textbf{xscanmap}
 iscan, kamp, kvamp [, iwhich]
\subsection*{Initialization}


 \emph{iscan}
 -- which scan process to read 


 \emph{iwhich}
 (optional) -- which node to sense. The default is 0. 
\subsection*{Performance}


 \emph{kamp}
 -- amount to amplify the \emph{kpos}
 value. 


 \emph{kvamp}
 -- amount to amplify the \emph{kvel}
 value. 


  The internal state of a node is read. This includes its position and velocity. They are amplified by the \emph{kamp}
 and \emph{kvamp}
 values. 
\subsection*{Credits}


 Author: John ffitch


 New in version 4.20
%\hline 


\begin{comment}
\begin{tabular}{lcr}
Previous &Home &Next \\
xout &Up &xscansmap

\end{tabular}


\end{document}
\end{comment}
