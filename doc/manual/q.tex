\begin{comment}
\documentclass[10pt]{article}
\usepackage{fullpage, graphicx, url}
\setlength{\parskip}{1ex}
\setlength{\parindent}{0ex}
\title{q Statement}
\begin{document}


\begin{tabular}{ccc}
The Alternative Csound Reference Manual & & \\
Previous & &Next

\end{tabular}

%\hline 
\end{comment}
\section{q Statement}
q statement�--� This statement may be used to quiet an instrument. \subsection*{Description}


  This statement may be used to quiet an instrument. 
\subsection*{Syntax}


 \textbf{q}
 p1 p2 p3
\subsection*{Performance}


 \emph{p1}
 -- Instrument number to mute/unmute. 


 \emph{p2}
 -- Action time of function generation (or destruction) in beats. 


 \emph{p3}
 -- determines whether the instrument is muted/unmuted. The value of 0 means the instrument is muted, other values mean it is unmuted. 


  Note that this does not affect instruments that are already running at time \emph{p2}
. It blocks any attempt to start one afterwards. 
%\hline 


\begin{comment}
\begin{tabular}{lcr}
Previous &Home &Next \\
n Statement &Up &r Statement (Repeat Statement)

\end{tabular}


\end{document}
\end{comment}
