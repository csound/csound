\begin{comment}
\documentclass[10pt]{article}
\usepackage{fullpage, graphicx, url}
\setlength{\parskip}{1ex}
\setlength{\parindent}{0ex}
\title{Orchestra Statement Types}
\begin{document}


\begin{tabular}{ccc}
The Alternative Csound Reference Manual & & \\
Previous &Syntax of the Orchestra &Next

\end{tabular}

%\hline 
\end{comment}
\section{Orchestra Statement Types}


  An orchestra program in Csound is comprised of \emph{orchestra header statements}
 which set various global parameters, followed by a number of \emph{instrument blocks}
 representing different instrument types. An instrument block, in turn, is comprised of \emph{ordinary statements}
 that set values, control the logical flow, or invoke the various signal processing subroutines that lead to audio output. 


 An \emph{orchestra header statement}
 operates once only, at orchestra setup time. It is most commonly an assignment of some value to a \emph{global reserved symbol}
 , e.g. sr = 20000. All orchestra header statements belong to a pseudo instrument 0, an \emph{init}
 pass of which is run prior to all other instruments at score time 0. Any \emph{ordinary statement}
 can serve as an orchestra header statement, eg. gifreq = cpspch(8.09) provided it is an init-time only operation. 


 An \emph{ordinary statement}
 runs at either init time or performance time or both. Operations which produce a result formally run at the rate of that result (that is, at init time for i-rate results; at performance time for k- and a-rate results), with the sole exception of the \emph{init}
 opcode. Most generators and modifiers, however, produce signals that depend not only on the instantaneous value of their arguments but also on some preserved internal state. These performance-time units therefore have an implicit init-time component to set up that state. The run time of an operation which produces no result is apparent in the opcode. 


 Arguments are values that are sent to an operation. Most arguments will accept arithmetic expressions composed of constants, variables, reserved symbols, value converters, arithmetic operations, and conditional values. 
%\hline 


\begin{comment}
\begin{tabular}{lcr}
Previous &Home &Next \\
Nomenclature &Up &Constants and Variables

\end{tabular}


\end{document}
\end{comment}
