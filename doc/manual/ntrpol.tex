\begin{comment}
\documentclass[10pt]{article}
\usepackage{fullpage, graphicx, url}
\setlength{\parskip}{1ex}
\setlength{\parindent}{0ex}
\title{ntrpol}
\begin{document}


\begin{tabular}{ccc}
The Alternative Csound Reference Manual & & \\
Previous & &Next

\end{tabular}

%\hline 
\end{comment}
\section{ntrpol}
ntrpol�--� Calculates the weighted mean value of two input signals. \subsection*{Description}


  Calculates the weighted mean value (i.e. linear interpolation) of two input signals 
\subsection*{Syntax}


 ar \textbf{ntrpol}
 asig1, asig2, kpoint [, imin] [, imax]


 ir \textbf{ntrpol}
 isig1, isig2, ipoint [, imin] [, imax]


 kr \textbf{ntrpol}
 ksig1, ksig2, kpoint [, imin] [, imax]
\subsection*{Initialization}


 \emph{imin}
 -- minimum xpoint value (optional, default 0) 


 \emph{imax}
 -- maximum xpoint value (optional, default 1) 
\subsection*{Performance}


 \emph{xsig1}
, \emph{xsig2}
 -- input signals 


 \emph{xpoint}
 -- interpolation point between the two values 


 \emph{ntrpol}
 opcode outputs the linear interpolation between two input values. \emph{xpoint}
 is the distance of evaluation point from the first value. With the default values of \emph{imin}
 and \emph{imax}
, (0 and 1) a zero value indicates no distance from the first value and the maximum distance from the second one. With a 0.5 value, \emph{ntrpol}
 will output the mean value of the two inputs, indicating the exact half point between \emph{xsig1}
 and \emph{xsig2}
. A 1 value indicates the maximum distance from the first value and no distance from the second one. The range of \emph{xpoint}
 can be also defined with \emph{imin}
 and \emph{imax}
 to make its management easier. 


  These opcodes are useful for crossfading two signals. 
\subsection*{Credits}


 


 


\begin{tabular}{ccc}
Author: Gabriel Maldonado &Italy &October 1998

\end{tabular}



 


 New in Csound version 3.49
%\hline 


\begin{comment}
\begin{tabular}{lcr}
Previous &Home &Next \\
nstrnum &Up &octave

\end{tabular}


\end{document}
\end{comment}
