\begin{comment}
\documentclass[10pt]{article}
\usepackage{fullpage, graphicx, url}
\setlength{\parskip}{1ex}
\setlength{\parindent}{0ex}
\title{voice}
\begin{document}


\begin{tabular}{ccc}
The Alternative Csound Reference Manual & & \\
Previous & &Next

\end{tabular}

%\hline 
\end{comment}
\section{voice}
voice�--� An emulation of a human voice. \subsection*{Description}


  An emulation of a human voice. 
\subsection*{Syntax}


 ar \textbf{voice}
 kamp, kfreq, kphoneme, kform, kvibf, kvamp, ifn, ivfn
\subsection*{Initialization}


 \emph{ifn}
, \emph{ivfn}
 -- two table numbers containing the carrier waveform and the vibrato waveform. The files \emph{impuls20.aiff}
, \emph{ahh.aiff}
, \emph{eee.aiff}
, or \emph{ooo.aiff}
 are suitable for the first of these, and a sine wave for the second. These files are available from \emph{\url{ftp://ftp.cs.bath.ac.uk/pub/dream/documentation/sounds/modelling/}}
. 
\subsection*{Performance}


 \emph{kamp}
 -- Amplitude of note. 


 \emph{kfreq}
 -- Frequency of note played. It can be varied in performance. 


 \emph{kphoneme}
 -- an integer in the range 0 to 16, which select the formants for the sounds: 


 
\begin{itemize}
\item 

 ``eee'', ``ihh'', ``ehh'', ``aaa'', 

\item 

 ``ahh'', ``aww'', ``ohh'', ``uhh'', 

\item 

 ``uuu'', ``ooo'', ``rrr'', ``lll'', 

\item 

 ``mmm'', ``nnn'', ``nng'', ``ngg''. 


\end{itemize}


  At present the phonemes 


 
\begin{itemize}
\item 

 ``fff'', ``sss'', ``thh'', ``shh'', 

\item 

 ``xxx'', ``hee'', ``hoo'', ``hah'', 

\item 

 ``bbb'', ``ddd'', ``jjj'', ``ggg'', 

\item 

 ``vvv'', ``zzz'', ``thz'', ``zhh''


\end{itemize}
 are not available (!) 

 \emph{kform}
 -- Gain on the phoneme. values 0.0 to 1.2 recommended. 


 \emph{kvibf}
 -- frequency of vibrato in Hertz. Suggested range is 0 to 12 


 \emph{kvamp}
 -- amplitude of the vibrato 
\subsection*{Examples}


  Here is an example of the voice opcode. It uses the files \emph{voice.orc}
, \emph{voice.sco}
, and \emph{impuls20.aiff}
. 


 \textbf{Example 1. Example of the voice opcode.}

\begin{lstlisting}
/* voice.orc */
; Initialize the global variables.
sr = 22050
kr = 2205
ksmps = 10
nchnls = 1

; Instrument #1.
instr 1
  kamp = 3
  kfreq = 0.8
  kphoneme = 6
  kform = 0.488
  kvibf = 0.04
  kvamp = 1
  ifn = 1
  ivfn = 2

  av voice kamp, kfreq, kphoneme, kform, kvibf, kvamp, ifn, ivfn

  ; It tends to get loud, so clip voice's amplitude at 30,000.
  a1 clip av, 2, 30000
  out a1
endin
/* voice.orc */
        
\end{lstlisting}
\begin{lstlisting}
/* voice.sco */
; Table #1, an audio file for the carrier waveform.
f 1 0 256 1 "impuls20.aiff" 0 0 0
; Table #2, a sine wave for the vibrato waveform.
f 2 0 256 10 1

; Play Instrument #1 for a half-second.
i 1 0 0.5
e
/* voice.sco */
        
\end{lstlisting}
\subsection*{Credits}


 


 


\begin{tabular}{ccc}
Author: John ffitch (after Perry Cook) &University of Bath, Codemist Ltd. &Bath, UK

\end{tabular}



 


 Example written by Kevin Conder.


 New in Csound version 3.47
%\hline 


\begin{comment}
\begin{tabular}{lcr}
Previous &Home &Next \\
vlowres &Up &vpvoc

\end{tabular}


\end{document}
\end{comment}
